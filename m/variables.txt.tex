
\question El alcance de una variable es
\choice el rango de valores que puede tomar.
\choice el tiempo transcurrido entre su creación y su desaparición.
\correctchoice el conjunto de líneas de código desde donde es visible la variable.

\question Una variable local es aquella que aparece definida
\correctchoice dentro de una función.
\choice fuera de una función.
\choice fuera de todas las funciones.

\question Una variable externa es aquella que aparece definida
\choice fuera de una función pero dentro de una segunda función.
\choice fuera de un bloque.
\correctchoice fuera de toda función.
\choice dentro de una función.

\question Una variable local puede ser usada
\choice desde toda la unidad de traducción.
\correctchoice desde dentro de la función donde se la declara.
\choice desde las funciones que aparecen con posterioridad a su declaración.
\choice en todos los casos anteriores.

\question Una variable externa puede ser usada
\choice desde toda la unidad de traducción.
\choice desde dentro de la función donde se la declara.
\correctchoice desde las funciones que aparecen con posterioridad a su declaración.
\choice en todos los casos anteriores.

\question Si una función declara una variable local con el mismo nombre que una externa, los usos de esa variable dentro de la función se referirán a
\correctchoice la variable local.
\choice la variable externa.
\choice depende de la implementación.

\question Una variable local vive durante
\choice toda la ejecución del programa.
\correctchoice la ejecución de la función donde se la declara.
\choice la compilación del programa.
\choice la ejecución de las funciones que aparecen con posterioridad a su declaración

\question Cambiar la clase de almacenamiento de una variable implica afectar 
\correctchoice cuándo aparece y desaparece.
\choice el tamaño de los objetos de datos que contiene.
\choice el ámbito de la variable dentro de la unidad de traducción.
\choice todo lo anterior.

\question La clase de almacenamiento por defecto de las variables locales es
\correctchoice \code{auto}.
\choice \code{static}.
\choice \code{register}.
\choice \code{extern}.

\question La clase de almacenamiento por defecto de las variables externas es
\choice \code{auto}.
\correctchoice \code{static}.
\choice \code{register}.
\choice \code{extern}.

\question Una variable con clase de almacenamiento \code{static} 
\choice se crea estáticamente al ejecutarse la función donde se la define.
\correctchoice se crea estáticamente al cargarse el programa en memoria.
\choice se crea estáticamente al iniciarse la ejecución de \code{main()}.

\question Una variable con clase de almacenamiento \code{static} 
\correctchoice se inicializa con ceros al inicio del programa.
\choice no se inicializa y contiene basura.
\choice se inicializa con ceros al ejecutarse la función donde se la define.

\question Una variable con clase de almacenamiento \code{auto}
\choice se inicializa con ceros al inicio del programa.
\correctchoice no se inicializa y contiene basura.
\choice se inicializa con ceros al ejecutarse la función donde se la define.

\question Las variables locales que \quotes{recuerdan la historia} son
\choice las declaradas \code{auto}.
\correctchoice las declaradas \code{static}.
\choice las declaradas \code{register}.
\choice ninguna de las anteriores.

\question Si un objeto se declara \code{static}
\choice se hace visible desde otras unidades de traducción.
\correctchoice se impide que se vea desde otras unidades de traducción.
\choice se impide que se vea desde otras funciones que aquella donde se lo define.

\question La declaración \code{extern} para una variable
\choice crea el objeto de datos correspondiente.
\choice equivale a una definición de la variable.
\choice indica la unidad de traducción donde está definida la variable.
\correctchoice solamente enuncia el tipo y nombre de la variable.

\question Una declaración \code{extern} puede corresponderse
\correctchoice con una variable externa en otra unidad de traducción.
\choice con una variable local en otra unidad de traducción.
\choice ninguna de las anteriores.

\question Una declaración \code{static} puede corresponderse
\choice con una variable externa en otra unidad de traducción.
\choice con una variable local en otra unidad de traducción.
\correctchoice ninguna de las anteriores.

\question Una variable \code{const} 
\choice debe ser optimizada.
\choice no debe ser optimizada.
\choice puede ser modificada sólo por funciones en la misma unidad de traducción.
\correctchoice no puede ser modificada.

\question Una variable \code{volatile} 
\choice debe ser optimizada.
\correctchoice no debe ser optimizada.
\choice puede ser modificada sólo por funciones en la misma unidad de traducción.
\choice no puede ser modificada.
