
\question ¿De qué tipo es la función siguiente?<br><code>float z(int p, short q) \
\correctchoice 1;<br>  float h\=2;<br>  return g;<br>\}{
\choice int
\correctchoice float
\choice short
\choice double

\question ¿Qué ocurre con el parámetro <emph>b</emph> en el cuerpo de la función siguiente?<br><code>int fun(int a) \
\correctchoice 2 * b;<br>  return b;<br>  \}</code><br>{
\choice El código no compila porque falta declarar el parámetro b
\choice Se devuelve el valor de b que es basura por ser variable local
\correctchoice Se devuelve b si b es una global declarada más arriba

\question ¿Cuál sería el prototipo más plausible para la función q() si su uso legal es como el siguiente?<br><code>float p, r; int s;<br>r \= q(p,s) / 2;</code><br>
\correctchoice q(p,s) / 2;</code><br>{
\choice r \= q(1, 2);
\correctchoice q(1, 2);
\correctchoice float q(float x, int y);
\choice float q(int x, int y);
\choice int q(float x, float y);

\question ¿Cuál sería el prototipo más plausible para la función t() si su uso legal es como el siguiente?<br><code>double w;<br>w\=t(5e1, 2L);</code><br>
\correctchoice t(5e1, 2L);</code><br>{
\choice long t();
\choice double t(int x, int y);
\correctchoice double t(double x, long y);
\choice long t(double x, double y);

\question ¿Con este prototipo, cuál es el parámetro cuyo tipo no es correcto en la invocación de la función?<br><code>void fun1(long x, double y, int g, char h);<br>fun2(500, 1.02e3, -12, 9);</code><br>
\correctchoice x
\choice y
\choice g
\choice h

\question ¿Con este prototipo, cuál es el parámetro cuyo tipo no es correcto en la invocación de la función?<br><code>void fun2(char a, unsigned b, int c, double d);<br>fun2('2', 100, 100, 100);</code><br>
\choice a
\choice b
\choice c
\correctchoice d

\question ¿Con quién está relacionado el problema en estas líneas?<br><code>void fun3(int e, unsigned short f, long int g, signed char h);<br>a \= fun3(1, 1, 1, 1);</code><br>
\correctchoice fun3(1, 1, 1, 1);</code><br>{
\choice Con e
\choice Con f
\choice Con g
\choice Con h
\correctchoice Con a
