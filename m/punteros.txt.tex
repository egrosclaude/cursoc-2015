
\question El sentido de la expresión &p puede traducirse como
\choice Lo apuntado por p
\correctchoice La dirección de p
\choice El arreglo comenzado por p
\choice *p
\choice La cadena apuntada por p

\question La expresión *p puede traducirse como
\choice Multiplicar por p
\correctchoice Lo apuntado por p
\choice La dirección de p
\choice p

\question El sentido de la expresión &(*p) puede traducirse como
\choice La dirección de p
\choice Lo apuntado por p
\choice Lo apuntado por la dirección de p
\correctchoice La dirección de lo apuntado por p

\question La expresión &(*p) equivale al valor de 
\choice *p
\choice &p
\correctchoice p

\question La expresión *(&p) equivale al valor de 
\choice *p
\choice &p
\correctchoice p

\question Con la declaración <br><code>char *k;</code><br>k será
\choice un carácter
\correctchoice un puntero a carácter
\choice la dirección de un carácter
\choice todas las anteriores

\question Con la declaración <br><code> char k;</code><br>k será
\correctchoice un carácter 
\choice un puntero a carácter
\choice la dirección de un carácter
\choice todas las anteriores

\question Con la declaración <br><code> int j;</code><br>la expresión &j será
\choice un puntero
\correctchoice la dirección de j
\choice lo apuntado por j
\choice un arreglo

\question Con la declaración <br><code> int *j;</code><br>la expresión *j será
\choice un puntero
\choice la dirección de j
\correctchoice lo apuntado por j
\choice un arreglo

\question La expresión p-q, si p y q son punteros a char, vale
\choice La cantidad de bytes entre las direcciones apuntadas por p y q
\choice La diferencia entre las direcciones apuntadas por p y q
\choice La cantidad de bytes que hace falta desplazarse desde la dirección apuntada por p para llegar a la dirección apuntada por q
\correctchoice Todas las anteriores
\choice Ninguna de las anteriores

\question La expresión g-f, si g y f son punteros a long, vale
\choice La cantidad de bytes entre las direcciones apuntadas por g y f
\correctchoice La cantidad de longs que caben entre las direcciones apuntadas por g y f
\choice La diferencia entre los longs apuntados por g y por f
\choice Todas las anteriores
\choice Ninguna de las anteriores

\question Con la declaración <br><code>char *s \= "abcdef";</code><br> construimos
\correctchoice "abcdef";</code><br> construimos{
\choice Un arreglo
\choice Un puntero a una cadena terminada en "0"
\correctchoice Un puntero a una cadena terminada en '\0'
\choice Un puntero a un carácter '\0'
\choice Un puntero nulo

\question El puntero nulo es igual a
\choice "0"
\correctchoice (char *)0
\choice (char) "0"
\choice '0'

\question Con la declaración <br><code>char *a[] \= \
\correctchoice \{"alfa", "beta", "gamma" \};</code><br>se tiene que a[1] equivale a {
\correctchoice la dirección de la cadena "beta"
\choice la dirección de la cadena "alfa"
\choice la letra 'l' dentro de la cadena "alfa"
\choice la letra 'b' dentro de la cadena "beta"
\choice ninguna de las anteriores

\question Con la declaración <br><code>char *a[] \= \
\correctchoice \{"alfa", "beta", "gamma" \};</code><br>se tiene que *a[1] equivale a {
\choice la dirección de la cadena "beta"
\choice la dirección de la cadena "alfa"
\choice la letra 'l' dentro de la cadena "alfa"
\correctchoice la letra 'b' dentro de la cadena "beta"
\choice ninguna de las anteriores

\question Con la declaración <br><code>char *a[] \= \
\correctchoice \{"alfa", "beta", "gamma" \};</code><br>la letra 't' dentro de la cadena "beta" se puede escribir como {
\choice a[1][2]
\choice *(a[1]+2)
\choice *(*(a+1)+2)
\correctchoice todas las anteriores
\choice ninguna de las anteriores
