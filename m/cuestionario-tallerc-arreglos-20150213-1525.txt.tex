
\question ::Cambiar la clase de almacenamiento de una variable implica afectar::Cambiar la clase de almacenamiento de una variable implica afectar
\correctchoice Cuándo aparece y desaparece# 
\choice El tamaño de los objetos de datos que contiene# 
\choice El ámbito de la variable dentro de la unidad de traducción# 
\choice Todo lo anterior# 

\question ::Con la declaración  char k;k será::Con la declaración <br><code> char k;</code><br>k será
\correctchoice un carácter# 
\choice un puntero a carácter# 
\choice la dirección de un carácter# 
\choice todas las anteriores# 

\question ::Con la declaración  int *j;la expresión *j será::Con la declaración <br><code> int *j;</code><br>la expresión *j será
\choice un puntero# 
\choice la dirección de j# 
\correctchoice lo apuntado por j# 
\choice un arreglo# 

\question ::Con la declaración  int j;la expresión &j será::Con la declaración <br><code> int j;</code><br>la expresión &j será
\choice un puntero# 
\correctchoice la dirección de j# 
\choice lo apuntado por j# 
\choice un arreglo# 

\question // question: 1118  name: Con la declaración char *a[] = 
\correctchoice {"alfa", "beta", "gamma" };la letra 't' dentro de la cadena "beta" se puede escribir como 

\question ::Con la declaración char *a[] \= \
\correctchoice \{"alfa", "beta", "gamma" \};la letra 't' dentro de la cadena "beta" se puede escribir como::Con la declaración <br><code>char *a[] \= \{"alfa", "beta", "gamma" \};</code><br>la letra 't' dentro de la cadena "beta" se puede escribir como{
\choice a[1][2]# 
\choice *(a[1]+2)# 
\choice *(*(a+1)+2)# 
\correctchoice todas las anteriores# 
\choice ninguna de las anteriores# 

\question // question: 1117  name: Con la declaración char *a[] = 
\correctchoice {"alfa", "beta", "gamma" };se tiene que *a[1] equivale a 

\question ::Con la declaración char *a[] \= \
\correctchoice \{"alfa", "beta", "gamma" \};se tiene que *a[1] equivale a::Con la declaración <br><code>char *a[] \= \{"alfa", "beta", "gamma" \};</code><br>se tiene que *a[1] equivale a{
\choice la dirección de la cadena "beta"# 
\choice la dirección de la cadena "alfa"# 
\choice la letra 'l' dentro de la cadena "alfa"# 
\correctchoice la letra 'b' dentro de la cadena "beta"# 
\choice ninguna de las anteriores# 

\question // question: 1116  name: Con la declaración char *a[] = 
\correctchoice {"alfa", "beta", "gamma" };se tiene que a[1] equivale a 

\question ::Con la declaración char *a[] \= \
\correctchoice \{"alfa", "beta", "gamma" \};se tiene que a[1] equivale a::Con la declaración <br><code>char *a[] \= \{"alfa", "beta", "gamma" \};</code><br>se tiene que a[1] equivale a{
\correctchoice la dirección de la cadena "beta"# 
\choice la dirección de la cadena "alfa"# 
\choice la letra 'l' dentro de la cadena "alfa"# 
\choice la letra 'b' dentro de la cadena "beta"# 
\choice ninguna de las anteriores# 

\question ::Con la declaración char *k;k será::Con la declaración <br><code>char *k;</code><br>k será
\choice un carácter# 
\correctchoice un puntero a carácter# 
\choice la dirección de un carácter# 
\choice todas las anteriores# 
\correctchoice "abcdef"; construimos 

\question ::Con la declaración char *s \= "abcdef"; construimos::Con la declaración <br><code>char *s \= "abcdef";</code><br> construimos
\correctchoice "abcdef"; construimos::Con la declaración <br><code>char *s \= "abcdef";</code><br> construimos{
\choice Un arreglo# 
\choice Un puntero a una cadena terminada en "0"# 
\correctchoice Un puntero a una cadena terminada en '\0'# 
\choice Un puntero a un carácter '\0'# 
\choice Un puntero nulo# 

\question // question: 1085  name: Con la declaración del arreglo que sigue, ¿cuál de las sentencias es incorrecta? long trece[12] = 
\correctchoice { 1, 5, 20L, 35}; 

\question ::Con la declaración del arreglo que sigue, ¿cuál de las sentencias es incorrecta? long trece[12] \= \
\correctchoice \{ 1, 5, 20L, 35\};::Con la declaración del arreglo que sigue, ¿cuál de las sentencias es incorrecta? <br><code>long trece[12] \= \{ 1, 5, 20L, 35\};</code><br>{
\choice trece[1]++;# 
\correctchoice trece[12]--;# 
\choice trece[1] \= trece[0];# 
\correctchoice trece[0];# 
\choice trece[11] \= 20L;# 
\correctchoice 20L;# 

\question ::Cuando existe una condición de overflow\:::Cuando existe una condición de <emph>overflow</emph>\:
\choice El programa aborta con mensaje de error# 
\choice El programa es terminado por el subsistema de protección del sistema operativo# 
\correctchoice El valor se trunca# 
\choice El valor se redondea# 
\choice La variable vuelve a cero# 
\correctchoice 1; la operación a  

\question ::Dada la declaración unsigned char a\=1; la operación a ::Dada la declaración <code>unsigned char a\=1;</code> la operación a <<\= a tiene como resultado
\correctchoice 1; la operación a ::Dada la declaración <code>unsigned char a\=1;</code> la operación a <<\= a tiene como resultado{
\choice 0# 
\choice 1# 
\correctchoice 2# 
\choice 255# 
\choice 127# 

\question ::El alcance de una variable es::El alcance de una variable es
\choice El rango de valores que puede tomar# 
\choice El tiempo transcurrido entre su creación y su desaparición# 
\correctchoice El conjunto de líneas de código desde donde es visible la variable# 

\question ::El efecto de las directivas de preprocesador abarca::El efecto de las directivas de preprocesador abarca
\choice La función donde están declaradas# 
\correctchoice La unidad de traducción# 
\choice El proyecto de programación# 
\choice El bloque donde están declaradas# 

\question ::El lenguaje C::El lenguaje C
\choice No realiza recolección de basura pero sí controles de tiempo de ejecución# 
\choice No realiza controles de tiempo de ejecución pero sí recolección de basura# 
\choice Realiza ambas cosas# 
\correctchoice Ninguna de las dos cosas# 

\question ::El lenguaje C pertenece al paradigma::El lenguaje C pertenece al paradigma
\choice Lógico# 
\correctchoice Procedural# 
\choice Funcional# 
\choice Orientado a objetos# 

\question ::El mapa de memoria del programa comprende::El mapa de memoria del programa comprende
\choice Dos regiones estáticas y dos dinámicas# 
\choice Cuatro regiones en total# 
\choice Regiones de texto, de datos estáticos, de heap y de stack# 
\correctchoice Todo lo anterior# 

\question ::El operador ^ en C significa::El operador ^ en C significa
\choice exponenciación en base 10# 
\choice exponenciación en base <emph>e</emph># 
\correctchoice or exclusivo de bits# 
\choice or lógico exclusivo# 

\question ::El pasaje de argumentos a funciones en C se hace::El pasaje de argumentos a funciones en C se hace
\correctchoice por valor# 
\choice por referencia# 
\choice por nombre# 

\question ::El preprocesador facilita::El preprocesador facilita
\choice El mantenimiento# 
\choice La legibilidad# 
\choice La expresividad# 
\correctchoice Todas las anteriores# 

\question ::El preprocesador interviene::El preprocesador interviene
\choice Después de la compilación del código# 
\correctchoice Antes de la compilación del código# 

\question ::El preprocesador promueve::El preprocesador promueve
\correctchoice La legibilidad# 
\choice La redundancia# 
\choice La rapidez de ejecución# 
\choice Todas las anteriores# 

\question ::El principal objetivo de diseño de quienes crearon el C era::El principal objetivo de diseño de quienes crearon el C era
\choice Posibilidad de acceder a los recursos de hardware# 
\choice Portabilidad del compilador# 
\choice Eficiencia del código generado# 
\correctchoice Todas las anteriores# 

\question ::El problema de la expansión errónea de las macros se soluciona::El problema de la expansión errónea de las macros se soluciona
\choice Rodeando los argumentos entre signos <># 
\choice Rodeando los argumentos con corchetes# 
\choice Poniendo la macro completa entre comillas# 
\correctchoice Rodeando los argumentos con paréntesis# 

\question ::El puntero nulo es igual a::El puntero nulo es igual a
\choice "0"# 
\correctchoice (char *)0# 
\choice (char) "0"# 
\choice '0'# 

\question ::El resultado de preprocesar la siguiente macro\:  \#define FUNCION(x) 3*x+1 aplicada al argumento 2+1 será::[html]El resultado de preprocesar la siguiente macro\: <br /> <br /><code>\#define FUNCION(x) 3*x+1</code><br /><br /> aplicada al argumento <emph>2+1</emph> será
\choice 3*3+1
\correctchoice 3*2+1+1
\choice 7
\choice 8

\question ::El sentido de la expresión &(*p) puede traducirse como::El sentido de la expresión &(*p) puede traducirse como
\choice La dirección de p# 
\choice Lo apuntado por p# 
\choice Lo apuntado por la dirección de p# 
\correctchoice La dirección de lo apuntado por p# 

\question ::El sentido de la expresión &p puede traducirse como::El sentido de la expresión &p puede traducirse como
\choice Lo apuntado por p# 
\correctchoice La dirección de p# 
\choice El arreglo comenzado por p# 
\choice *p# 
\choice La cadena apuntada por p# 

\question ::El signo default de las variables de tipo int y de tipo char es, respectivamente,::El signo default de las variables de tipo int y de tipo char es, respectivamente,
\choice signed y unsigned# 
\correctchoice signed y dependiente de la implementación# 
\choice unsigned y signed# 
\choice dependiente de la implementación y unsigned# 
\choice dependiente de la implementación en ambos casos# 

\question ::El tipo de las expresiones::El tipo de las expresiones
\choice Es asignado por el compilador# 
\choice Es asignado por el usuario# 
\correctchoice Las dos anteriores# 
\choice Es asignado por el linkeditor# 
\choice No puede ser modificado por el usuario# 

\question ::El utilitario make genera::El utilitario <emph>make</emph> genera
\choice archivos objeto# 
\choice ejecutables# 
\choice bibliotecas# 
\correctchoice todo lo anterior# 

\question ::El valor máximo de un unsigned char suele estar en el orden de::El valor máximo de un unsigned char suele estar en el orden de
\choice las decenas# 
\correctchoice los cientos# 
\choice los miles# 
\choice las decenas de miles# 
\choice los millones# 

\question ::Elija dos constantes correctamente escritas\:::Elija dos constantes correctamente escritas\:
\correctchoice 0xFFU y 0XABL# 
\choice 0ABU y 010# 
\choice 0x10 y -0dB# 
\choice Todas las anteriores# 

\question ::En la declaración siguiente, ¿cuál es el tipo básico interpretado por el compilador?  unsigned short byte;::En la declaración siguiente, ¿cuál es el tipo básico interpretado por el compilador? <br> <br><code>unsigned short byte;</code><br>
\choice char# 
\correctchoice int# 
\choice long# 

\question ::Indicar cuál de las expresiones tiene valor lógico falso\:::Indicar cuál de las expresiones tiene valor lógico falso\:
\choice a \=\= a# 
\correctchoice \= a# 
\choice 2 * a - a# 
\correctchoice a \= 0# 
\choice 1 !\= 0# 
\correctchoice 0# 

\question ::La Biblioteca Standard de C::La <emph>Biblioteca Standard</emph> de C
\choice Provee funciones para todas las necesidades# 
\choice Está escrita por el usuario# 
\correctchoice No provee funciones para todas las necesidades# 

\question ::La cadena 'ABC0x25' es::La cadena 'ABC0x25' es
\choice Una constante decimal# 
\choice Una constante hexadecimal# 
\choice Una constante string# 
\choice Una constante de carácter# 
\correctchoice Ninguna de las anteriores# 

\question ::La clase de almacenamiento por defecto de las variables externas es::La clase de almacenamiento por defecto de las variables externas es
\choice auto# 
\correctchoice static# 
\choice register# 
\choice extern# 

\question ::La clase de almacenamiento por defecto de las variables locales es::La clase de almacenamiento por defecto de las variables locales es
\correctchoice auto# 
\choice static# 
\choice register# 
\choice extern# 

\question ::La constante de carácter '0' en un sistema basado en ASCII tiene el valor decimal::La constante de carácter '0' en un sistema basado en ASCII tiene el valor decimal
\choice 0# 
\correctchoice 48# 
\choice "0"# 

\question ::La constante de carácter '\0' en un sistema basado en ASCII tiene el valor decimal::La constante de carácter '\0' en un sistema basado en ASCII tiene el valor decimal
\correctchoice 0# 
\choice 48# 
\choice "0"# 

\question ::La constante string "A\103" se leerá una vez impresa como::La constante string "A\103" se leerá una vez impresa como
\choice AA# 
\choice A103# 
\choice A\103# 
\correctchoice A\103# 

\question ::La declaración extern para una variable::La declaración extern para una variable
\choice crea el objeto de datos correspondiente# 
\choice equivale a una definición de la variable# 
\choice indica la unidad de traducción donde está definida la variable# 
\correctchoice solamente enuncia el tipo y nombre de la variable# 

\question ::La directiva correcta para crear una macro que devuelva el doble de su argumento es::[html]La directiva correcta para crear una macro que devuelva el doble de su argumento es
\choice \#DOBLE(x) 2*x
\choice \#define DOBLE 2*x
\correctchoice \#define DOBLE(x) 2*(x)# define DOBLE(x) 2*(x)
\choice \#define DOBLE(x) 2*(x);
\choice \#define DOBLE(x) 2 * (x)

\question ::La expresión &(*p) equivale al valor de::La expresión &(*p) equivale al valor de
\choice *p# 
\choice &p# 
\correctchoice p# 
\correctchoice = b) ? c : d vale 

\question ::La expresión (a \=\= b) ? c \: d vale::La expresión <code>(a \=\= b) ? c \: d</code> vale
\correctchoice \= b) ? c \: d vale::La expresión <code>(a \=\= b) ? c \: d</code> vale{
\choice a si a es igual a b# 
\choice b si c es distinto de d# 
\choice c si c es igual a d# 
\correctchoice d si a es distinto de b# 

\question ::La expresión *(&p) equivale al valor de::La expresión *(&p) equivale al valor de
\choice *p# 
\choice &p# 
\correctchoice p# 

\question ::La expresión *p puede traducirse como::La expresión *p puede traducirse como
\choice Multiplicar por p# 
\correctchoice Lo apuntado por p# 
\choice La dirección de p# 
\choice p# 

\question ::La expresión g-f, si g y f son punteros a long, vale::La expresión g-f, si g y f son punteros a long, vale
\choice La cantidad de bytes entre las direcciones apuntadas por g y f# 
\correctchoice La cantidad de longs que caben entre las direcciones apuntadas por g y f# 
\choice La diferencia entre los longs apuntados por g y por f# 
\choice Todas las anteriores# 
\choice Ninguna de las anteriores# 

\question ::La expresión p-q, si p y q son punteros a char, vale::La expresión p-q, si p y q son punteros a char, vale
\choice La cantidad de bytes entre las direcciones apuntadas por p y q# 
\choice La diferencia entre las direcciones apuntadas por p y q# 
\choice La cantidad de bytes que hace falta desplazarse desde la dirección apuntada por p para llegar a la dirección apuntada por q# 
\correctchoice Todas las anteriores# 
\choice Ninguna de las anteriores# 
\correctchoice 0x07 equivale a: 

\question ::La operación a &\= 0x07 equivale a\:::La operación <code>a &\= 0x07</code> equivale a\:
\correctchoice 0x07 equivale a\:::La operación <code>a &\= 0x07</code> equivale a\:{
\choice dividir a por 7# 
\choice dividir a por 8# 
\choice tomar el resto de dividir a por 7# 
\correctchoice tomar el resto de dividir a por 8# 
\choice restarle 8 a a# 
\choice restarle 7 a a# 
\correctchoice 2 equivale a: 

\question ::La operación a >>\= 2 equivale a\:::La operación <code>a >>\= 2</code> equivale a\:
\correctchoice 2 equivale a\:::La operación <code>a >>\= 2</code> equivale a\:{
\choice dividir a por dos# 
\correctchoice dividir a por cuatro# 
\choice multiplicar a por dos# 
\choice multiplicar a por cuatro# 

\question ::La primera definición oficial del lenguaje fue dada por Kernighan y Ritchie en::La primera definición oficial del lenguaje fue dada por Kernighan y Ritchie en
\choice 1975# 
\correctchoice 1978# 
\choice 1983# 
\choice 1988# 

\question ::La región de pila almacena::La región de pila almacena
\correctchoice las variables locales# 
\choice las variables estáticas# 
\choice las estructuras de datos dinámicas# 
\choice el código del programa# 
\correctchoice a % 2; puede escribirse también 

\question ::La sentencia a \= a % 2; puede escribirse también::La sentencia <code>a \= a % 2;</code> puede escribirse también
\correctchoice a % 2; puede escribirse también::La sentencia <code>a \= a % 2;</code> puede escribirse también{
\choice a%%;# 
\correctchoice a %\= 2;# 
\choice a \=% 2;# 
\correctchoice % 2;# 
\choice a%2;# 
\correctchoice 2) ? 3 : 4); imprime: 

\question ::La sentencia printf("%d ", (1 !\= 2) ? 3 \: 4); imprime\:::[html]La sentencia <code><br /></code>\n<p style\="font-family\: courier new,courier,monospace;"><code>printf(&quot;%d&quot;, (1 !\= 2) ? 3 \: 4);</code></p> imprime\:
\correctchoice 2) ? 3 \: 4); imprime\:::[html]La sentencia <code><br /></code>\n<p style\="font-family\: courier new,courier,monospace;"><code>printf(&quot;%d&quot;, (1 !\= 2) ? 3 \: 4);</code></p> imprime\:{
\choice 1
\choice 2
\correctchoice 3
\choice 4

\question ::Las directivas condicionales consideran un segmento de texto::Las directivas condicionales consideran un segmento de texto
\choice Sólo si la compilación resulta exitosa# 
\correctchoice Sólo si la condición resulta exitosa# 

\question ::Las directivas de preprocesador::Las directivas de preprocesador
\choice Están contenidas en el lenguaje C# 
\choice Son variables de texto# 
\correctchoice No pertenecen al lenguaje C# 
\choice Son palabras reservadas# 
\choice Son funciones de C# 

\question ::Las herramientas del ciclo de compilación comprenden::Las herramientas del ciclo de compilación comprenden
\choice compilador y linkeditor# 
\correctchoice editor, compilador, linkeditor y bibliotecario# 
\choice compilador y biblioteca standard# 

\question ::Las palabras reservadas de C son::Las palabras reservadas de C son
\choice Muchas# 
\correctchoice Pocas# 
\choice Exactamente las de entrada/salida# 
\choice Exactamente tantas como las de Pascal# 

\question ::Las variables locales que "recuerdan la historia" son::Las variables locales que "recuerdan la historia" son
\choice las declaradas auto# 
\correctchoice las declaradas static# 
\choice las declaradas register# 
\choice ninguna de las anteriores# 

\question ::Los headers::Los headers
\choice Son escritos por el usuario# 
\choice Vienen con el compilador# 
\correctchoice Todas las anteriores# 
\choice Ninguna de las anteriores# 

\question ::Los headers que definen funciones::Los headers que definen funciones
\choice Son recomendables# 
\choice Son imprescindibles# 
\correctchoice No son recomendables# 
\choice Son recomendables pero no imprescindibles# 

\question ::Los programas en C son portables porque::Los programas en C son portables porque
\choice Se lo dotó de control de tipos de datos# 
\correctchoice Los tipos de datos no tienen un tamaño definido por el lenguaje# 
\choice Los tamaños de los tipos de datos son idénticos en todas las implementaciones# 
\choice Se lo basó en una única arquitectura# 
\correctchoice 1; b = 0; if(a = b)   b = a; las variables a y b valen respectivamente 

\question ::Luego de ejecutar las sentencias a \= 1; b \= 0; if(a \= b)   b \= a; las variables a y b valen respectivamente::[html]Luego de ejecutar las sentencias <code><br />a \= 1; b \= 0; <br />if(a \= b)<br /> b \= a;</code> <br />las variables a y b valen respectivamente
\correctchoice 1; b \= 0; if(a \= b)   b \= a; las variables a y b valen respectivamente::[html]Luego de ejecutar las sentencias <code><br />a \= 1; b \= 0; <br />if(a \= b)<br /> b \= a;</code> <br />las variables a y b valen respectivamente{
\correctchoice 0 y 0
\choice 0 y 2
\choice 2 y 2
\choice ninguno de los anteriores
\correctchoice 1; b = 2; if(a = b)   b = a; las variables a y b valen respectivamente 

\question ::Luego de ejecutar las sentencias a \= 1; b \= 2; if(a \= b)   b \= a; las variables a y b valen respectivamente::[html]Luego de ejecutar las sentencias <code><br />a \= 1; b \= 2; <br />if(a \= b)<br /> b \= a;</code> <br />las variables a y b valen respectivamente
\correctchoice 1; b \= 2; if(a \= b)   b \= a; las variables a y b valen respectivamente::[html]Luego de ejecutar las sentencias <code><br />a \= 1; b \= 2; <br />if(a \= b)<br /> b \= a;</code> <br />las variables a y b valen respectivamente{
\choice 1 y 1
\choice 1 y 2
\correctchoice 2 y 2
\choice ninguno de los anteriores
\correctchoice 1; a=++c; a y c valen respectivamente 

\question ::Luego de ejecutar las sentencias c\=1; a\=++c; a y c valen respectivamente::Luego de ejecutar las sentencias <code>c\=1; a\=++c;</code> a y c valen respectivamente
\correctchoice 1; a\=++c; a y c valen respectivamente::Luego de ejecutar las sentencias <code>c\=1; a\=++c;</code> a y c valen respectivamente{
\choice 1 y 1# 
\choice 1 y 2# 
\correctchoice 2 y 2# 
\choice 2 y 1# 
\choice ninguna de las anteriores# 
\correctchoice 1; a=--c; a += c++; a y c valen respectivamente 

\question ::Luego de ejecutar las sentencias c\=1; a\=--c; a +\= c++; a y c valen respectivamente::[html]Luego de ejecutar las sentencias <code><br />c\=1; a\=--c; a +\= c++;</code> <br />a y c valen respectivamente
\correctchoice 1; a\=--c; a +\= c++; a y c valen respectivamente::[html]Luego de ejecutar las sentencias <code><br />c\=1; a\=--c; a +\= c++;</code> <br />a y c valen respectivamente{
\choice 1 y 1
\choice 1 y 2
\choice 2 y 2
\choice 2 y 1
\correctchoice ninguna de las anteriores
\correctchoice 1; a=c++; a y c valen respectivamente 

\question ::Luego de ejecutar las sentencias c\=1; a\=c++; a y c valen respectivamente::[html]Luego de ejecutar las sentencias <code><br />c\=1; a\=c++;</code> <br />a y c valen respectivamente
\correctchoice 1; a\=c++; a y c valen respectivamente::[html]Luego de ejecutar las sentencias <code><br />c\=1; a\=c++;</code> <br />a y c valen respectivamente{
\choice 1 y 1
\correctchoice 1 y 2
\choice 2 y 2
\choice 2 y 1
\choice ninguna de las anteriores

\question ::Normalmente los headers contienen::Normalmente los headers contienen
\correctchoice Declaraciones de variables y funciones# 
\choice Definiciones de variables y funciones# 
\choice Prototipos de directivas# 
\choice Inclusión de archivos fuente# 
\choice Todas las anteriores# 

\question ::Si a es int y b es long, ¿cuál es el tipo de la expresión siguiente?a / b::Si a es int y b es long, ¿cuál es el tipo de la expresión siguiente?<br><code>a / b</code><br>
\correctchoice long# 
\choice int# 
\choice float# 

\question ::Si a y b son char que tienen el máximo valor posible para los chars, el tipo de la expresión a * b es\:::Si a y b son char que tienen el máximo valor posible para los chars, el tipo de la expresión <br><code>a * b</code><br> es\:
\choice unsigned char# 
\choice unsigned# 
\choice int# 
\correctchoice char# 

\question ::Si a y b son enteros, para obtener el valor de su cociente con decimales se debe escribir::Si a y b son enteros, para obtener el valor de su cociente con decimales se debe escribir
\choice a % b;# 
\choice a / b;# 
\correctchoice (float) a / b;# 
\choice float (a) / b;# 
\choice (float)(a / b);# 

\question ::Si un objeto se declara static::Si un objeto se declara static
\choice Se hace visible desde otras unidades de traducción# 
\correctchoice Se impide que se vea desde otras unidades de traducción# 
\choice Se impide que se vea desde otras funciones que aquella donde se lo define# 

\question ::Si un signed char vale 127 y se le suma 1\:::Si un <emph>signed char</emph> vale 127 y se le suma 1\:
\choice queda en 0# 
\choice queda en -127# 
\choice queda en 128# 
\correctchoice queda en -128# 
\choice queda en -1# 

\question ::Si un unsigned int vale 0 y se le resta 1,::Si un unsigned int vale 0 y se le resta 1,
\choice queda en 0# 
\choice queda en -1# 
\correctchoice queda en el valor del máximo entero sin signo# 
\choice queda en 65535# 
\choice queda en 32768# 

\question ::Si una función declara una variable local con el mismo nombre que una externa, los usos de esa variable dentro de la función se referirán a::Si una función declara una variable local con el mismo nombre que una externa, los usos de esa variable dentro de la función se referirán a
\correctchoice La variable local# 
\choice La variable externa# 
\choice Depende de la implementación# 

\question ::Un objeto de datos es::Un objeto de datos es
\choice Un tipo de datos# 
\choice Un tipo de datos abstracto# 
\choice Una variable# 
\correctchoice Un espacio de almacenamiento para contener valores# 

\question ::Un objeto de datos es ocupado::Un objeto de datos es ocupado
\choice Al terminar la ejecución del programa# 
\correctchoice Al calcular cada subexpresión# 
\choice Al inicio de cada función# 

\question ::Una constante de carácter correctamente escrita entre las siguientes es\:::Una constante de carácter correctamente escrita entre las siguientes es\:
\choice '0xAB'# 
\choice "A"# 
\correctchoice 'a'# 
\choice 265# 
\choice Todas las anteriores# 

\question ::Una declaración extern puede corresponderse::Una declaración extern puede corresponderse
\correctchoice con una variable externa en otra unidad de traducción# 
\choice con una variable local en otra unidad de traducción# 
\choice ninguna de las anteriores# 

\question ::Una declaración signed indica que la variable contendrá::Una declaración <emph>signed</emph> indica que la variable contendrá
\choice un número negativo o cero# 
\choice un número positivo o negativo# 
\choice un número no negativo# 
\correctchoice un número negativo, positivo o cero# 

\question ::Una declaración static puede corresponderse::Una declaración static puede corresponderse
\choice con una variable externa en otra unidad de traducción# 
\choice con una variable local en otra unidad de traducción# 
\correctchoice ninguna de las anteriores# 

\question ::Una variable con clase de almacenamiento auto::Una variable con clase de almacenamiento auto
\choice se inicializa con ceros al inicio del programa# 
\correctchoice no se inicializa y contiene basura# 
\choice se inicializa con ceros al ejecutarse la función donde se la define# 

\question ::Una variable con clase de almacenamiento static::Una variable con clase de almacenamiento static
\choice se crea estáticamente al ejecutarse la función donde se la define# 
\correctchoice se crea estáticamente al cargarse el programa en memoria# 
\choice se crea estáticamente al iniciarse la ejecución de main()# 

\question ::Una variable con clase de almacenamiento static::Una variable con clase de almacenamiento static
\correctchoice se inicializa con ceros al inicio del programa# 
\choice no se inicializa y contiene basura# 
\choice se inicializa con ceros al ejecutarse la función donde se la define# 

\question ::Una variable const::Una variable const
\choice debe ser optimizada# 
\choice no debe ser optimizada# 
\choice puede ser modificada sólo por funciones en la misma unidad de traducción# 
\correctchoice no puede ser modificada# 

\question ::Una variable externa es aquella que aparece definida::Una variable externa es aquella que aparece definida
\choice Fuera de una función pero dentro de una segunda función# 
\choice Fuera de un bloque# 
\correctchoice Fuera de toda función# 
\choice Dentro de una función# 

\question ::Una variable externa puede ser usada::Una variable externa puede ser usada
\choice desde toda la unidad de traducción# 
\choice desde dentro de la función donde se la declara# 
\correctchoice desde las funciones que aparecen con posterioridad a su declaración# 
\choice en todos los casos anteriores# 

\question ::Una variable local es aquella que aparece definida::Una variable local es aquella que aparece definida
\correctchoice Dentro de una función# 
\choice Fuera de una función# 
\choice Fuera de todas las funciones# 

\question ::Una variable local puede ser usada::Una variable local puede ser usada
\choice desde toda la unidad de traducción# 
\correctchoice desde dentro de la función donde se la declara# 
\choice desde las funciones que aparecen con posterioridad a su declaración# 
\choice en todos los casos anteriores# 

\question ::Una variable local vive::Una variable local vive
\choice Durante toda la ejecución del programa# 
\correctchoice Durante la ejecución de la función donde se la declara# 
\choice Durante la compilación del programa# 
\choice Durante la ejecución de las funciones que aparecen con posterioridad a su declaración# 

\question ::¿Con este prototipo, cuál es el parámetro cuyo tipo no es correcto en la invocación de la función?void fun1(long x, double y, int g, char h);fun2(500, 1.02e3, -12, 9);::¿Con este prototipo, cuál es el parámetro cuyo tipo no es correcto en la invocación de la función?<br><code>void fun1(long x, double y, int g, char h);<br>fun2(500, 1.02e3, -12, 9);</code><br>
\correctchoice x# 
\choice y# 
\choice g# 
\choice h# 

\question ::¿Con este prototipo, cuál es el parámetro cuyo tipo no es correcto en la invocación de la función?void fun2(char a, unsigned b, int c, double d);fun2('2', 100, 100, 100);::¿Con este prototipo, cuál es el parámetro cuyo tipo no es correcto en la invocación de la función?<br><code>void fun2(char a, unsigned b, int c, double d);<br>fun2('2', 100, 100, 100);</code><br>
\choice a# 
\choice b# 
\choice c# 
\correctchoice d# 
\correctchoice fun3(1, 1, 1, 1); 

\question ::¿Con quién está relacionado el problema en estas líneas?void fun3(int e, unsigned short f, long int g, signed char h);a \= fun3(1, 1, 1, 1);::¿Con quién está relacionado el problema en estas líneas?<br><code>void fun3(int e, unsigned short f, long int g, signed char h);<br>a \= fun3(1, 1, 1, 1);</code><br>
\correctchoice fun3(1, 1, 1, 1);::¿Con quién está relacionado el problema en estas líneas?<br><code>void fun3(int e, unsigned short f, long int g, signed char h);<br>a \= fun3(1, 1, 1, 1);</code><br>{
\choice Con e# 
\choice Con f# 
\choice Con g# 
\choice Con h# 
\correctchoice Con a# 

\question ::¿Cuál de estas declaraciones es la correcta?::¿Cuál de estas declaraciones es la correcta?

\question 	= char palabra[]\=\
\correctchoice char palabra[]\=\{'n', 'u', 'e', 'v', 'o'\};# 

\question 	~char palabra \= \
\choice char palabra \= \{'n', 'u', 'e', 'v', 'o'\};# 
\correctchoice \{'n', 'u', 'e', 'v', 'o'\};# 
\choice char palabra \= 'nuevo';# 
\correctchoice 'nuevo';# 

\question 	~char palabra[4] \= \
\choice char palabra[4] \= \{'n', 'u', 'e', 'v', 'o'\};# 
\correctchoice \{'n', 'u', 'e', 'v', 'o'\};# 

\question ::¿Cuál de estos segmentos de programa es incorrecto?::¿Cuál de estos segmentos de programa es incorrecto?
\choice int alfa[3]; alfa[2++] \= 8;# 
\correctchoice 8;# 

\question 	= int alfa[3]; alfa \= \
\correctchoice int alfa[3]; alfa \= \{ 1, 2, 8 \};# 
\choice int alfa[3]; c \= alfa[0]++;# 
\correctchoice alfa[0]++;# 
\choice int alfa[3]; alfa[1] \= alfa[2];# 
\correctchoice alfa[2];# 

\question ::¿Cuál de las declaraciones siguientes es correcta?::¿Cuál de las declaraciones siguientes es correcta?
\correctchoice long size;# 
\choice double float a;# 
\choice unsigned long integer p;# 
\choice LONG alfa;# 

\question ::¿Cuál de las declaraciones siguientes es incorrecta?::¿Cuál de las declaraciones siguientes es incorrecta?
\choice int i,j,k;# 
\choice char uvw;# 
\correctchoice unsigned a, short b;# 
\choice unsigned long int integer;# 

\question ::¿Cuál de las declaraciones siguientes no es correcta?::¿Cuál de las declaraciones siguientes <emph>no</emph> es correcta?
\choice char byte;# 
\choice unsigned char integer;# 
\correctchoice unsigned double a;# 
\choice long UNO;# 
\choice long int eme;# 

\question ::¿Cuál de las reglas no es válida?::¿Cuál de las reglas <emph>no es</emph> válida?
\choice Toda expresión cuyo valor aritmético es 0 tiene valor lógico falso# 
\choice Toda expresión cuyo valor lógico es falso tiene valor aritmético 0# 
\choice Toda expresión cuyo valor aritmético es 1 tiene valor lógico verdadero# 
\correctchoice Toda expresión cuyo valor lógico es verdadero tiene valor aritmético 1# 

\question ::¿Cuál es la declaración correcta para un arreglo de nueve caracteres?::¿Cuál es la declaración correcta para un arreglo de nueve caracteres?
\choice int chars[9];# 
\choice alfa\=char[9];# 
\correctchoice char[9];# 
\choice char[9] alfa;# 
\correctchoice char alfa[9];# 

\question ::¿Cuál es la directiva correcta para incluir un header llamado beta.h situado en el directorio donde se está realizando la compilación?::[html]¿Cuál es la directiva correcta para incluir un header llamado <emph>beta.h</emph> situado en el directorio donde se está realizando la compilación?
\choice \#define <beta.h>
\choice \#include <beta.h>
\correctchoice \#include "beta.h"# include &quot;beta.h&quot;

\question ::¿Cuál es la directiva de preprocesador correcta si queremos definir un símbolo ALFA con valor 1?::[html]¿Cuál es la directiva de preprocesador correcta si queremos definir un símbolo ALFA con valor 1?
\choice \#ALFA \= 1
\correctchoice 1
\choice \#define ALFA \= 1
\correctchoice 1
\correctchoice \#define ALFA 1# define ALFA 1
\choice \#define 1 ALFA

\question ::¿Cuál es la directiva de preprocesador correcta si queremos incluir el header de biblioteca standard stdio.h?::[html]¿Cuál es la directiva de preprocesador correcta si queremos incluir el header de biblioteca standard <emph>stdio.h</emph>?
\choice \#include stdio.h
\choice \#include <stdio>
\correctchoice \#include <stdio.h># include <stdio.h></stdio.h>
\choice Cualquiera de las anteriores

\question ::¿Cuál es la regla verdadera para los tamaños de los objetos de datos?::¿Cuál es la regla verdadera para los tamaños de los objetos de datos?
\choice Un long es mayor que un short# 
\choice Un int es menor que un long# 
\choice Un short no es menor que un long# 
\correctchoice Un short no es mayor que un long# 
\correctchoice q(p,s) / 2; 

\question ::¿Cuál sería el prototipo más plausible para la función q() si su uso legal es como el siguiente?float p, r; int s;r \= q(p,s) / 2;::¿Cuál sería el prototipo más plausible para la función q() si su uso legal es como el siguiente?<br><code>float p, r; int s;<br>r \= q(p,s) / 2;</code><br>
\correctchoice q(p,s) / 2;::¿Cuál sería el prototipo más plausible para la función q() si su uso legal es como el siguiente?<br><code>float p, r; int s;<br>r \= q(p,s) / 2;</code><br>{
\choice r \= q(1, 2);# 
\correctchoice q(1, 2);# 
\correctchoice float q(float x, int y);# 
\choice float q(int x, int y);# 
\choice int q(float x, float y);# 
\correctchoice t(5e1, 2L); 

\question ::¿Cuál sería el prototipo más plausible para la función t() si su uso legal es como el siguiente?double w;w\=t(5e1, 2L);::¿Cuál sería el prototipo más plausible para la función t() si su uso legal es como el siguiente?<br><code>double w;<br>w\=t(5e1, 2L);</code><br>
\correctchoice t(5e1, 2L);::¿Cuál sería el prototipo más plausible para la función t() si su uso legal es como el siguiente?<br><code>double w;<br>w\=t(5e1, 2L);</code><br>{
\choice long t();# 
\choice double t(int x, int y);# 
\correctchoice double t(double x, long y);# 
\choice long t(double x, double y);# 

\question // question: 1066  name: ¿Cuántas X imprimen estas líneas?c=3; do 
\correctchoice 3; do { printf("X"); } while(--c); 

\question ::¿Cuántas X imprimen estas líneas?c\=3; do \
\correctchoice 3; do \{ printf("X"); \} while(--c);::[html]¿Cuántas X imprimen estas líneas?<code><br />c\=3; do \{ printf(&quot;X&quot;); \} while(--c);</code>{
\choice 1
\choice 2
\correctchoice 3
\choice 4

\question // question: 1065  name: ¿Cuántas X imprimen estas líneas?c=3; do 
\correctchoice 3; do { printf("X"); } while(c--); 

\question ::¿Cuántas X imprimen estas líneas?c\=3; do \
\correctchoice 3; do \{ printf("X"); \} while(c--);::[html]¿Cuántas X imprimen estas líneas?<br /><code>c\=3; do \{ printf(&quot;X&quot;); \} while(c--);</code>{
\choice 1
\choice 2
\choice 3
\correctchoice 4
\correctchoice 3; while(--c) printf("X"); 

\question ::¿Cuántas X imprimen estas líneas?c\=3; while(--c) printf("X");::[html]¿Cuántas X imprimen estas líneas?<code><br />c\=3; <br />while(--c) <br /> printf(&quot;X&quot;); </code>
\correctchoice 3; while(--c) printf("X");::[html]¿Cuántas X imprimen estas líneas?<code><br />c\=3; <br />while(--c) <br /> printf(&quot;X&quot;); </code>{
\choice 1
\correctchoice 2
\choice 3
\choice 4
\correctchoice 3; while(c--) printf("X"); 

\question ::¿Cuántas X imprimen estas líneas?c\=3; while(c--) printf("X");::[html]¿Cuántas X imprimen estas líneas?<br /><code>c\=3; <br />while(c--) <br /> printf(&quot;X&quot;); </code>
\correctchoice 3; while(c--) printf("X");::[html]¿Cuántas X imprimen estas líneas?<br /><code>c\=3; <br />while(c--) <br /> printf(&quot;X&quot;); </code>{
\choice 1
\choice 2
\correctchoice 3
\choice 4

\question // question: 1083  name: ¿Cuántos elementos tiene este arreglo?  int trece[12] = 
\correctchoice { 1, 3, 5 }; 

\question ::¿Cuántos elementos tiene este arreglo?  int trece[12] \= \
\correctchoice \{ 1, 3, 5 \};::¿Cuántos elementos tiene este arreglo? <br><code> int trece[12] \= \{ 1, 3, 5 \};</code><br>{
\choice 3# 
\choice 11# 
\correctchoice 12# 
\choice 13# 

\question // question: 1084  name: ¿Cuántos elementos tiene este arreglo?  int trece[] = 
\correctchoice { 1, 3, 5 }; 

\question ::¿Cuántos elementos tiene este arreglo?  int trece[] \= \
\correctchoice \{ 1, 3, 5 \};::¿Cuántos elementos tiene este arreglo? <br><code> int trece[] \= \{ 1, 3, 5 \};</code><br>{
\correctchoice 3# 
\choice 11# 
\choice 12# 
\choice 13# 

\question ::¿Cuántos elementos tiene este arreglo? long trece[12];::¿Cuántos elementos tiene este arreglo? <br><code>long trece[12];</code><br>
\choice 11# 
\correctchoice 12# 
\choice 13# 

\question // question: 1051  name: ¿De qué tipo es la función siguiente?float z(int p, short q) 
\correctchoice 1;  float h=2;  return g;} 

\question ::¿De qué tipo es la función siguiente?float z(int p, short q) \
\correctchoice 1;  float h\=2;  return g;\}::¿De qué tipo es la función siguiente?<br><code>float z(int p, short q) \{<br>  double g\=1;<br>  float h\=2;<br>  return g;<br>\}{
\choice int# 
\correctchoice float# 
\choice short# 
\choice double# 

\question // question: 1052  name: ¿Qué ocurre con el parámetro b en el cuerpo de la función siguiente?int fun(int a) 
\correctchoice 2 * b;  return b;  } 

\question ::¿Qué ocurre con el parámetro b en el cuerpo de la función siguiente?int fun(int a) \
\correctchoice 2 * b;  return b;  \}::¿Qué ocurre con el parámetro <emph>b</emph> en el cuerpo de la función siguiente?<br><code>int fun(int a) \{<br>  a \= 2 * b;<br>  return b;<br>  \}</code><br>{
\choice El código no compila porque falta declarar el parámetro b# 
\choice Se devuelve el valor de b que es basura por ser variable local# 
\correctchoice Se devuelve b si b es una global declarada más arriba# 

\question // question: 1062  name: ¿Qué resultado final tiene la variable a si inicialmente a, c y d valen 1?switch(c) 
\correctchoice a+d;  case 2: a=a-d;} 

\question ::¿Qué resultado final tiene la variable a si inicialmente a, c y d valen 1?switch(c) \
\correctchoice a+d;  case 2\: a\=a-d;\}::[html]¿Qué resultado final tiene la variable a si inicialmente a, c y d valen 1?<code><br />switch(c) \{<br /> case 1\: a\=a+d;<br /> case 2\: a\=a-d;<br />\}</code>{
\correctchoice 1
\choice 2
\choice 3
\correctchoice 1;if(1)    a=2; 

\question ::¿Qué resultado final tiene la variable a?a\=1;if(1)    a\=2;::[html]¿Qué resultado final tiene la variable <emph>a</emph>?<code><br />a\=1;<br />if(1)<br /> a\=2;</code>
\correctchoice 1;if(1)    a\=2;::[html]¿Qué resultado final tiene la variable <emph>a</emph>?<code><br />a\=1;<br />if(1)<br /> a\=2;</code>{
\choice 1
\correctchoice 2
\choice ninguna de las anteriores
\correctchoice 1;if(b=0)    a=2; 

\question ::¿Qué resultado final tiene la variable a?a\=1;if(b\=0)    a\=2;::[html]¿Qué resultado final tiene la variable <emph>a</emph>?<code><br />a\=1;<br />if(b\=0)<br /> a\=2;</code>
\correctchoice 1;if(b\=0)    a\=2;::[html]¿Qué resultado final tiene la variable <emph>a</emph>?<code><br />a\=1;<br />if(b\=0)<br /> a\=2;</code>{
\correctchoice 1
\choice 2
\choice depende del valor de b
\choice ninguna de las anteriores
\correctchoice 1;if(b==2);    a=2; 

\question ::¿Qué resultado final tiene la variable a?a\=1;if(b\=\=2);    a\=2;::[html]¿Qué resultado final tiene la variable <emph>a</emph>?<br /><code>a\=1;<br />if(b\=\=2);<br /> a\=2;</code>
\correctchoice 1;if(b\=\=2);    a\=2;::[html]¿Qué resultado final tiene la variable <emph>a</emph>?<br /><code>a\=1;<br />if(b\=\=2);<br /> a\=2;</code>{
\choice 1
\correctchoice 2
\choice depende del valor de b
\choice ninguna de las anteriores
\correctchoice 3;if(b==1)    a=2;else if(b==2) a=3; else a=4; 

\question ::¿Qué resultado final tiene la variable a?b\=3;if(b\=\=1)    a\=2;else if(b\=\=2) a\=3; else a\=4;::[html]¿Qué resultado final tiene la variable <emph>a</emph>?<code><br />b\=3;<br />if(b\=\=1)<br /> a\=2;else if(b\=\=2) a\=3; else a\=4;</code>
\correctchoice 3;if(b\=\=1)    a\=2;else if(b\=\=2) a\=3; else a\=4;::[html]¿Qué resultado final tiene la variable <emph>a</emph>?<code><br />b\=3;<br />if(b\=\=1)<br /> a\=2;else if(b\=\=2) a\=3; else a\=4;</code>{
\choice 2
\choice 3
\correctchoice 4
\choice no está definido

\question // question: 1063  name: ¿Qué resultado final tiene la variable b si inicialmente b, c y d valen 1?switch(c) 
\correctchoice b+d;  case 2: b=b-d;  default:          b=0;} 

\question ::¿Qué resultado final tiene la variable b si inicialmente b, c y d valen 1?switch(c) \
\correctchoice b+d;  case 2\: b\=b-d;  default\:          b\=0;\}::[html]¿Qué resultado final tiene la variable b si inicialmente b, c y d valen 1?<code><br />switch(c) \{<br /> case 1\: b\=b+d;<br /> case 2\: b\=b-d;<br /> default\:<br /> b\=0;<br />\}</code>{
\correctchoice 0
\choice 1
\choice 2
\choice 3
\correctchoice 1;for(i=0; i 

\question ::¿Qué resultado final tiene la variable c?c\=1;for(i\=0; i::[html]¿Qué resultado final tiene la variable c?<br /><code>c\=1;<br />for(i\=0; i&lt;5; i++);<br /> for(j\=0; j&lt;2; j++)<br /> c++;</code>
\correctchoice 1;for(i\=0; i::[html]¿Qué resultado final tiene la variable c?<br /><code>c\=1;<br />for(i\=0; i&lt;5; i++);<br /> for(j\=0; j&lt;2; j++)<br /> c++;</code>{
\choice 1
\choice 2
\correctchoice 3
\choice 6
\choice 13
