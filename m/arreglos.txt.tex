
\question ¿Cuántos elementos tiene este arreglo? <br><code>long trece[12];</code><br>
\choice 11
\correctchoice 12
\choice 13

\question ¿Cuál es la declaración correcta para un arreglo de nueve caracteres?
\choice int chars[9];
\choice alfa\=char[9];
\correctchoice char[9];
\choice char[9] alfa;
\correctchoice char alfa[9];

\question ¿Cuántos elementos tiene este arreglo? <br><code> int trece[12] \= \
\correctchoice \{ 1, 3, 5 \};</code><br>{
\choice 3
\choice 11
\correctchoice 12
\choice 13

\question ¿Cuántos elementos tiene este arreglo? <br><code> int trece[] \= \
\correctchoice \{ 1, 3, 5 \};</code><br>{
\correctchoice 3
\choice 11
\choice 12
\choice 13

\question Con la declaración del arreglo que sigue, ¿cuál de las sentencias es incorrecta? <br><code>long trece[12] \= \
\correctchoice \{ 1, 5, 20L, 35\};</code><br>{
\choice trece[1]++;
\correctchoice trece[12]--;
\choice trece[1] \= trece[0];
\correctchoice trece[0];
\choice trece[11] \= 20L;
\correctchoice 20L;

\question ¿Cuál de estos segmentos de programa es incorrecto?
\choice int alfa[3]; alfa[2++] \= 8;
\correctchoice 8;

\question = int alfa[3]; alfa \= \
\correctchoice int alfa[3]; alfa \= \{ 1, 2, 8 \};
\choice int alfa[3]; c \= alfa[0]++;
\correctchoice alfa[0]++;
\choice int alfa[3]; alfa[1] \= alfa[2];
\correctchoice alfa[2];
