
\question El operador ^ en C significa
\choice exponenciación en base 10
\choice exponenciación en base <emph>e</emph>
\correctchoice or exclusivo de bits
\choice or lógico exclusivo

\question Luego de ejecutar las sentencias <br><code>c\=1; a\=c++;</code><br> a y c valen respectivamente
\correctchoice 1; a\=c++;</code><br> a y c valen respectivamente{
\choice 1 y 1
\correctchoice 1 y 2
\choice 2 y 2
\choice 2 y 1
\choice ninguna de las anteriores

\question Luego de ejecutar las sentencias <br><code>c\=1; a\=++c;</code><br> a y c valen respectivamente
\correctchoice 1; a\=++c;</code><br> a y c valen respectivamente{
\choice 1 y 1
\choice 1 y 2
\correctchoice 2 y 2
\choice 2 y 1
\choice ninguna de las anteriores

\question Luego de ejecutar las sentencias <br><code>c\=1; a\=--c; a +\= c++;</code><br> a y c valen respectivamente
\correctchoice 1; a\=--c; a +\= c++;</code><br> a y c valen respectivamente{
\choice 1 y 1
\choice 1 y 2
\choice 2 y 2
\choice 2 y 1
\correctchoice ninguna de las anteriores

\question La sentencia <br><code>a \= a % 2;</code><br> puede escribirse también
\correctchoice a % 2;</code><br> puede escribirse también{
\choice a%%;
\correctchoice a %\= 2;
\choice a \=% 2;
\correctchoice % 2;
\choice a%2;

\question Luego de ejecutar las sentencias <br><code>a \= 1; b \= 2; if(a \= b) b \= a;</code><br> las variables a y b valen respectivamente 
\correctchoice 1; b \= 2; if(a \= b) b \= a;</code><br> las variables a y b valen respectivamente {
\choice 1 y 1
\choice 1 y 2
\correctchoice 2 y 2
\choice ninguno de los anteriores

\question Luego de ejecutar las sentencias <br><code>a \= 1; b \= 0; if(a \= b) b \= a;</code><br> las variables a y b valen respectivamente 
\correctchoice 1; b \= 0; if(a \= b) b \= a;</code><br> las variables a y b valen respectivamente {
\correctchoice 0 y 0
\choice 0 y 2
\choice 2 y 2
\choice ninguno de los anteriores

\question ¿Cuál de las reglas <emph>no es</emph> válida?
\choice Toda expresión cuyo valor aritmético es 0 tiene valor lógico falso
\choice Toda expresión cuyo valor lógico es falso tiene valor aritmético 0
\choice Toda expresión cuyo valor aritmético es 1 tiene valor lógico verdadero
\correctchoice Toda expresión cuyo valor lógico es verdadero tiene valor aritmético 1

\question Indicar cuál de las expresiones tiene valor lógico falso:
\choice a \=\= a
\correctchoice \= a
\choice 2 * a - a
\correctchoice a \= 0
\choice 1 !\= 0
\correctchoice 0

\question La operación <br><code>a &\= 0x07</code><br> equivale a:
\correctchoice 0x07</code><br> equivale a:{
\choice dividir a por 7
\choice dividir a por 8
\choice tomar el resto de dividir a por 7
\correctchoice tomar el resto de dividir a por 8
\choice restarle 8 a a
\choice restarle 7 a a

\question La operación <br><code>a >>\= 2</code><br> equivale a:
\correctchoice 2</code><br> equivale a:{
\choice dividir a por dos
\correctchoice dividir a por cuatro
\choice multiplicar a por dos
\choice multiplicar a por cuatro

\question Dada la declaración <br><code>unsigned char a\=1;</code><br> la operación a <<\= a tiene como resultado
\correctchoice 1;</code><br> la operación a <<\= a tiene como resultado{
\choice 0
\choice 1
\correctchoice 2
\choice 255
\choice 127

\question La expresión <br><code>(a \=\= b) ? c : d</code><br> vale
\correctchoice \= b) ? c : d</code><br> vale{
\choice a si a es igual a b
\choice b si c es distinto de d
\choice c si c es igual a d
\correctchoice d si a es distinto de b

\question La sentencia <br><code>printf("%d\n", (1 !\= 2) ? 3 : 4);</code><br> imprime:
\correctchoice 2) ? 3 : 4);</code><br> imprime:{
\choice 1
\choice 2
\correctchoice 3
\choice 4
