\documentclass[a4paper,11pt]{article}
\usepackage[T1]{fontenc}
\usepackage[utf8]{inputenc}
\usepackage{lmodern}

\title{}
\author{}



\newcounter{question}
\newif\ifinchoices
\inchoicesfalse
\newenvironment{questions}{%
  \list{\thequestion.}%
  {%
    \usecounter{question}%
    \def\question{\inchoicesfalse\item}%
    \settowidth{\leftmargin}{10.\hskip\labelsep}%
    \labelwidth\leftmargin\advance\labelwidth-\labelsep
  }%
}
{%
  \endlist
}%

\newcounter{choice}
\renewcommand\thechoice{\Alph{choice}}
\newcommand\choicelabel{\thechoice.}
\def\choice{%
  \ifinchoices
    % Do nothing
  \else
    \startchoices
  \fi
  \refstepcounter{choice}%
  \ifnum\value{choice}>1\relax
  \penalty -50\hskip 1em plus 1em\relax
  \fi
  \choicelabel
  \nobreak\enskip
}% choice
\def\CorrectChoice{%
  \choice
  \addanswer{\thequestion}{\thechoice}%
}
\let\correctchoice\CorrectChoice

\newcommand{\startchoices}{%
  \inchoicestrue
  \setcounter{choice}{0}%
  \par % Uncomment this to have choices always start a new line
  % If we're continuing the paragraph containing the question,
  % then leave a bit of space before the first choice:
  \ifvmode\else\enskip\fi
}%

\newbox\allanswers
\setbox\allanswers=\hbox{}
\newcommand{\addanswer}[2]{%
  \global\setbox\allanswers=\hbox{\unhbox\allanswers #1.~#2\quad}%
}
\newcommand{\showanswers}{%
  % \vfill
%   \begin{center}
%     Answers
%   \end{center}
   \noindent\unhbox\allanswers
}



%--------------------------------------------------------------------
\begin{document}

\section {Preguntas}

\begin{questions}
\question El principal objetivo de diseño de quienes crearon el C era
\choice Posibilidad de acceder a los recursos de hardware
\choice Portabilidad del compilador
\choice Eficiencia del código generado
\correctchoice Todas las anteriores

\question La primera definición oficial del lenguaje fue dada por Kernighan y Ritchie en
\choice 1975
\correctchoice 1978
\choice 1983
\choice 1988

\question Las palabras reservadas de C son
\choice Muchas
\correctchoice Pocas
\choice Exactamente las de entrada/salida
\choice Exactamente tantas como las de Pascal

\question La <emph>Biblioteca Standard</emph> de C
\choice Provee funciones para todas las necesidades
\choice Está escrita por el usuario
\correctchoice No provee funciones para todas las necesidades

\question El lenguaje C
\choice No realiza recolección de basura pero sí controles de tiempo de ejecución
\choice No realiza controles de tiempo de ejecución pero sí recolección de basura
\choice Realiza ambas cosas
\correctchoice Ninguna de las dos cosas

\question Los programas en C son portables porque
\choice Se lo dotó de control de tipos de datos
\correctchoice Los tipos de datos no tienen un tamaño definido por el lenguaje
\choice Los tamaños de los tipos de datos son idénticos en todas las implementaciones
\choice Se lo basó en una única arquitectura

\question El pasaje de argumentos a funciones en C se hace
\correctchoice por valor
\choice por referencia
\choice por nombre

\question El lenguaje C pertenece al paradigma
\choice Lógico
\correctchoice Procedural
\choice Funcional
\choice Orientado a objetos

\question Las herramientas del ciclo de compilación comprenden
\choice compilador y linkeditor
\correctchoice editor, compilador, linkeditor y bibliotecario
\choice compilador y biblioteca standard

\question El utilitario <emph>make</emph> genera
\choice archivos objeto
\choice ejecutables
\choice bibliotecas
\correctchoice todo lo anterior

\question El mapa de memoria del programa comprende
\choice Dos regiones estáticas y dos dinámicas
\choice Cuatro regiones en total
\choice Regiones de texto, de datos estáticos, de heap y de stack 
\correctchoice Todo lo anterior

\question La región de pila almacena
\correctchoice las variables locales
\choice las variables estáticas
\choice las estructuras de datos dinámicas
\choice el código del programa

\end{questions}

\section {Respuestas}
\showanswers
% \newpage
% \begin{questions}
%   \question $1+1= \mathord{?}$
%     \choice $0$ \choice $1$ \correctchoice $2$ \choice $3$ \choice $4$
% 
%   \question $\sin \frac{\pi}{2} = \mathord{?}$
%     \choice $0$ \correctchoice $1$ \choice $2$ \choice $\frac{\pi}{2}$
% 
%   \question What was the color of the bus driver's eyes?
%     \choice Green \choice Yellow \correctchoice Blue \choice Red
% \end{questions}
% 
% \showanswers

\end{document}
