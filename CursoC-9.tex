


\chapter{Variables estructuradas}

Como casi todos los lenguajes, el C provee varias maneras de estructurar, o combinar, variables simples en
alguna forma de agregación. Las variables estructuradas permiten manejar un conjunto de datos como una sola entidad.

\section{Arreglos}
La declaración
\begin{lstlisting}
tipo nombre[cant_elementos];
\end{lstlisting}

define un bloque llamado \code{nombre} de \code{cant_elementos} \textbf{objetos consecutivos} de tipo
\code{tipo}, lo que habitualmente recibe el nombre de \textbf{vector, array o arreglo}. Sus
elementos se acceden \textbf{indexando} el bloque con expresiones enteras entre
corchetes.

\begin{ejemplo}
Declaraciones y uso de arreglos.
\begin{lstlisting}
int dias[12];
\end{lstlisting}
\begin{lstlisting}
dias[0] = 31;
enero = dias[0];
febrero = dias[1];
a = dias[6 * b - 1];
\end{lstlisting}
\begin{lstlisting}
double saldo[10];
for(i=0; i<10; i++)
    saldo[i] = entradas[i] - salidas[i];
\end{lstlisting}
\end{ejemplo}


\importante{En C, los arreglos se indexan a partir de 0.}

\subsection{Inicialización de arreglos}

Al ser declarados, los arreglos pueden recibir una \textbf{inicialización}, que es una
lista de valores del tipo correspondiente, indicados entre llaves. Esta
inicialización puede ser completa o incompleta. Si se omite la dimensión del
arreglo, el compilador la infiere por la cantidad de valores de inicialización.

\begin{ejemplo}
Inicializaciones completas e incompletas.
\begin{lstlisting}
/* inicialización completa */
int dias[12] = { 31, 28, 31, 30, 31, 30, 31, 31, 30, 31, 30, 31 };   
\end{lstlisting}
\begin{lstlisting}
/* inicialización incompleta */
double saldo[10] = { 150.40, 170.20 };     
\end{lstlisting}
\begin{lstlisting}
long altura[] = { 3600, 3400, 3200, 6950 };  /* se infiere "long altura[4]" */
\end{lstlisting}
\end{ejemplo}

\subsection{Errores frecuentes con arreglos}

Los siguientes son errores de programación muy comunes al tratar con arreglos, y lamentablemente el lenguaje C no
provee ayuda para prevenirlos:
\begin{itemize}
\item \textbf{Indexación fuera de límites}
	
La dimensión dada en la declaración del arreglo dice cuántos elementos tiene.
Esto no quiere decir que exista un elemento del arreglo con ese índice (porque
se numeran a partir de 0).


\begin{ejemplo}
La última instrucción equivale a acceder a un elemento fuera de los límites del
arreglo. Este programa, aunque erróneo, \textbf{compilará correctamente}. 
\begin{lstlisting}
int tabla[5];

tabla [5] = 1;	/* ¡Error! el último elemento es tabla[4] */
\end{lstlisting}
\end{ejemplo}





\item \textbf{Asignación de arreglos}

Es frecuente confundir las operaciones de inicialización y de asignación. La
inicialización sólo es válida en el momento de la declaración: \textbf{no es legal
asignar un arreglo}. La asignación debe forzosamente hacerse elemento por
elemento.

\begin{ejemplo}
Notar la diferencia sintáctica entre inicialización y asignación.
\begin{lstlisting}
/* Inicializacion */
int tabla[5] = { 1, 3, 2, 3, 4 }; 	/* correcto */

/* Asignacion */
int tabla[5];
tabla[] = { 1, 3, 2, 3, 4 }; 		/* incorrecto */
\end{lstlisting}
La última sentencia no es compilable. Debe reemplazarse por:
\begin{lstlisting}
tabla[0] = 1;
tabla[1] = 3; ...etc
\end{lstlisting}
\end{ejemplo}
\end{itemize}
\importante{ %TODO llevar a seccion de punteros
Al
momento de ejecución, la conducta dependerá de cuestiones estructurales del
sistema operativo. En variantes modernas de UNIX puede resultar un error de
\textbf{violación de segmentación}, o \textbf{falla de segmentación}, lo que indica que el proceso
ha \textbf{rebasado los límites de su espacio asignado} y el sistema de protección del
sistema operativo ha terminado el proceso.

Sin embargo, en algunos otros casos, el error puede pasar inadvertido al
momento de ejecución, porque la dirección del acceso, a pesar de superar los límites del
arreglo, no cae fuera del espacio de memoria del proceso. En este caso la ejecución
proseguirá, pero \textbf{se habrá leído o escrito basura} en alguna zona impredecible
del espacio de memoria del proceso. 

Éste y otros casos de accesos no controlados a la memoria pueden tener consecuencias, también impredecibles, más adelante en la ejecución. Como la causa y la manifestación del error están separadas, esta clase de errores resulta muy difícil de diagnosticar.


%TODO [Acceso al Arreglo]
   El diagrama ilustra dos casos de acceso indebido a un elemento inexistente
  de un arreglo. Suponemos tener una declaración tal como int tabla[5], y una
 instrucción errónea que hace referencia al elemento tabla[5].
         En el primer caso, la variable estructurada tiene algún otro contenido
         contiguo dentro del espacio del proceso, y el acceso lee o escribe basura.
         En el segundo caso, el acceso cae fuera del espacio del proceso, y según la
         reacción del sistema operativo, puede ocurrir lo mismo que en el caso
         anterior, o bien el proceso puede ser terminado por la fuerza.
}


\section{Arreglos multidimensionales}
En C se pueden simular matrices y arreglos de más dimensiones creando \textbf{arreglos
cuyos elementos son arreglos}. La declaración:
\begin{lstlisting}
int matriz[3][4];
\end{lstlisting}
expresa un arreglo de tres posiciones cuyos elementos son arreglos de cuatro
posiciones. Una declaración con inicialización podría escribirse así:
\begin{lstlisting}
int matriz[3][4] = {
    {1, 2, 5, 7},
    {3, 0, 0, 1},
    {2, 8, 5, 4}};
\end{lstlisting}
y correspondería a una matriz de tres filas por cuatro columnas.
La primera dimensión de un arreglo multidimensional puede ser inferida:
\begin{lstlisting}
int matriz[][4] = {
    {1, 2, 5, 7},
    {3, 0, 0, 1},
    {2, 8, 5, 4}};
\end{lstlisting}
El recorrido de toda una matriz implica necesariamente un lazo doble, a dos
variables:
\begin{lstlisting}
for(i=0; i<3; i++)
    for(j=0; j<4; j++)
      a[i][j] = i  + j;	
\end{lstlisting}

\section{Estructuras y uniones}

\subsection{Estructuras}
Las \textbf{estructuras} de C permiten agrupar una cantidad de variables
simples, de tipos, en general, diferentes. Las estructuras y uniones aportan la
ventaja de que es posible manipular este conjunto de variables como un todo.

Es posible inicializar estructuras, asignar conjuntos de constantes a las
estructuras, asignar estructuras entre sí, pasarlas como argumentos reales a
funciones, y devolverlas como valor de salida de funciones. En particular, ésta
viene a ser la única manera de que una función devuelva más de un dato.

\begin{ejemplo}
Las declaraciones siguientes no definen variables, con espacio de
almacenamiento, sino que simplemente enuncian un nuevo tipo que puede usarse en
nuevas declaraciones de variables. 
\begin{lstlisting}
struct persona {
    long DNI;
    char nombre[40];
    int edad;
};

struct cliente {
    int num_cliente;
    struct persona p;
    double saldo;
};	
\end{lstlisting}
El nombre o \textit{tag} dado a la estructura es el
nombre del nuevo tipo. En las instrucciones siguientes se utilizan los tags
definidos anteriormente y se acceden a los diferentes miembros de las
estructuras.

\begin{lstlisting}
struct cliente c1, c2;

c1.num_cliente = 1001;
c1.p.DNI = 14233326;   /* acceso anidado */
c1.p.edad=40;

c2 = c1;               /* copia de estructuras */
struct persona p1 = {17698735, "Juan Perez", 30};
c2.p = p1;	
\end{lstlisting}
Una declaración con inicialización completa:
\begin{lstlisting}
struct cliente c3 = {
    1002,
    {17698735, "Juan Perez", 30},
    150.25 };
\end{lstlisting}
\end{ejemplo}

\begin{ejemplo}
Es legal declarar una variable \code{struct} junto con la enunciación de su
tipo, con o sin el tag asociado y con o sin inicialización.
\begin{lstlisting}
struct complejo { double real, imag; } c;
struct { double real, imag; } c;
struct complejo {
   double real, imag;
} c = { 20.5, -7.3 };
\end{lstlisting}
\end{ejemplo}

\begin{ejemplo}
Una función que recibe y devuelve estructuras:
\begin{lstlisting}
struct punto {
    int x, y;
};

struct punto puntomedio(struct punto p1, struct punto p2)
{
    struct punto z;
    z.x = (p1.x + p2.x) / 2;
    z.y = (p1.y + p2.y) / 2;
    return z;
}
\end{lstlisting}
\end{ejemplo}

\subsection{Uniones}
En una estructura, el compilador calcula la dirección de inicio de cada uno de
los miembros dentro de la estructura sumando los tamaños de los elementos de
datos. Una \textbf{unión} es un caso especial de estructura donde todos los miembros
\quotes{nacen} en el mismo lugar de origen de la estructura.
\begin{lstlisting}
union intchar {
    int i;
    char a[sizeof(int)];
};
\end{lstlisting}
Este ejemplo de unión contiene dos miembros: un entero y un arreglo de tantos
caracteres como bytes contiene un \code{int} en la arquitectura destino. Ambos
miembros, por la propiedad fundamental de los \code{unions}, quedan \quotes{superpuestos} en
memoria. El resultado es que podemos asignar un campo por un nombre y acceder
por el otro. 
\begin{ejemplo}
En este caso particular, podemos conocer qué valores recibe cada
byte de los que forman un \code{int}.
\begin{lstlisting}
union intchar k;
k.i = 30541;
b = k.a[2];
\end{lstlisting}
\end{ejemplo}
\nota{
      El diagrama ilustra los diferentes desplazamientos u Union y Estructura
offsets a los que se ubican los miembros en una union y en
  una struct, suponiendo una arquitectura subyacente donde
                              los ints miden cuatro bytes.
}

\subsection{Campos de bits}
Se pueden definir estructuras donde los miembros son agrupaciones de bits. Esta
construcción es especialmente útil en programación de sistemas donde se
necesita la máxima compactación de las estructuras de datos. Cada miembro de un
\textbf{campo de bits} es un \code{unsigned int} que lleva explícitamente un \quotes{ancho} indicando
la cantidad de bits que contiene.
\begin{lstlisting}
struct disp {
    unsigned int encendido : 1;
    unsigned int
        online : 1,
        estado : 4;
};
\end{lstlisting}
\begin{ejemplo}
Imaginemos, en base a la declaración anterior, un dispositivo mapeado en memoria con el cual
comunicarnos en base a un protocolo, también imaginario. Implementamos con un
campo de bits un registro de control \code{encendido}, que permite encenderlo o apagarlo,
consultar su disponibilidad (\code{online} o no), y hacer una lectura de un valor
de \code{estado} de cuatro bits (que entonces puede tomar valores entre 0 y 15). Todo
el registro de comunicación cabe en un byte.

Podríamos encender nuestro dispositivo imaginario, esperar a que se ponga
online, tomar el promedio de diez lecturas de estado y apagarlo, con las
instrucciones siguientes. Como se ve, no hay diferencia sintáctica de acceso con las
estructuras.
\begin{lstlisting}
struct disp {
    unsigned int encendido : 1;
    unsigned int
        online : 1,
        estado : 4;
};
\end{lstlisting}

\begin{lstlisting}
struct disp d;
d.encendido = 1;
while(!d.online);
for(p=0, i=0; i<10; i++)
    p += d.estado;
p /= 10;
d.encendido = 0;
\end{lstlisting}
\end{ejemplo}


\section{Ejercicios}
\begin{enumerate}
\item Escribir una declaración con inicialización de un arreglo de diez elementos
\code{double}, todos inicialmente iguales a $2.25$.
\item Escribir las sentencias para copiar un arreglo de cinco \code{longs} en otro.
\item Escribir las sentencias para obtener el producto escalar de dos vectores.
\item Escribir una función que devuelva la posición donde se halla el menor
elemento de un arreglo de \code{floats}.
\item Dado un vector de diez elementos, escribir todos los promedios de cuatro
elementos consecutivos.
\item Declarar una estructura punto conteniendo coordenadas x e y de tipo \code{double}.
Dar ejemplos de inicialización y de asignación.
\item Declarar una estructura \textbf{segmento} conteniendo dos estructuras \textbf{punto}. Dar
ejemplos de inicialización y de asignación. Dar una función que calcule su
longitud. Dar una función que reciba un segmento $S$ y un punto $p$, y devuelva \code{TRUE} si $p \in S$. 
\item ¿Cuál es el error en estas sentencias de inicialización?
\begin{enumerate} [label=\alph*.]

\item 
\begin{lstlisting}
struct alfa {
	int a, b;
};
alfa = { 10, 25 };
\end{lstlisting}

\item 
\begin{lstlisting}
struct alfa {
	int a, b;
};
alfa d = { 10, 25 };
\end{lstlisting}

\item 
\begin{lstlisting}
union dato {
	char dato_a[4];
	long dato_n;
} xdato = { "ABC", 1000 };
\end{lstlisting}
\end{enumerate}
\end{enumerate}

%TODO Ejercicios_Adicionales  Ejercicios_Avanzados

\begin{preguntas}

\question ¿Cuántos elementos tiene el arreglo \code{long trece[12];}?
\choice 11
\correctchoice 12
\choice 13

\question ¿Cuál es la declaración correcta para un arreglo de nueve caracteres?
\choice \code{int chars[9];}
\choice \code{alfa\=char[9];}
\correctchoice \code{char[9];}
\choice \code{char[9] alfa;}
\correctchoice \code{char alfa[9];}

\question ¿Cuántos elementos tiene el arreglo \code{int trece[12] \= \{1, 3, 5\};}? 
\choice 3
\choice 11
\correctchoice 12
\choice 13

\question ¿Cuántos elementos tiene el arreglo \code{int trece[] \= \{1, 3, 5\};}?
\correctchoice 3
\choice 11
\choice 12
\choice 13

\question Con la declaración del arreglo que sigue, ¿cuál de las sentencias es incorrecta? \\
\code{long trece[12] \= \{1, 5, 20L, 35\};}
\choice \code{trece[1]++;}
\correctchoice \code{trece[12]--;}
\choice \code{trece[1] \= trece[0];}
\correctchoice \code{trece[0];}
\choice \code{trece[11] \= 20L;}
\correctchoice \code{20L;}

\question ¿Cuál de estos segmentos de programa es incorrecto?
\choice \code{int alfa[3]; alfa[2++] \= 8;}
\correctchoice \code{int alfa[3]; alfa \= \{1, 2, 8\};}
\choice \code{int alfa[3]; c \= alfa[0]++;}
\choice \code{int alfa[3]; alfa[1] \= alfa[2];}

\end{preguntas}

