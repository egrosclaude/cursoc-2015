




\section{Errores más frecuentes}

\subsection{Punteros sin inicializar}
El utilizar un puntero sin inicializar parece estar entre las primeras causas
de errores en C.

\begin{ejemplo}
Si bien sintácticamente correctas, las líneas del ejemplo presentan un problema
muy común en C. Declaran un puntero a entero, \lstinline{p}, y almacenan un valor entero en
la dirección apuntada por el mismo. 

\begin{lstlisting}
/* Incorrecto */
   int *p;
   *p = 8;
\end{lstlisting}   

Como \lstinline{p} es un puntero, su contenido será interpretado como una dirección de memoria. El problema es que el contenido de \lstinline{p} es \textbf{impredecible}:
\begin{itemize}
	\item Si la variable \lstinline{p} es local, su clase de almacenamiento es \textbf{auto} y
por lo tanto contiene \textbf{basura}, salvo inicialización explícita. 
	\item Si \lstinline{p} es externa, \textbf{será inicializada a 0}. 
\end{itemize}

En el primer caso, esa dirección es aleatoria. En el segundo caso, será \lstinline{cero}, que en la mayoría de los sistemas operativos conocidos es una dirección \textbf{inaccesible} para los procesos no privilegiados.  

En cualquier caso, el programa compilará pero se encontrará con problemas de ejecución. Lo que falta es hacer que \lstinline{p} apunte a alguna zona válida.

Direcciones válidas, que pueden ser manipuladas mediante punteros, son las de
los objetos conocidos por el programa: \textbf{variables, estructuras, arreglos,
funciones, bloques de memoria asignados dinámicamente}, son todos objetos cuya
dirección ha sido \textbf{obtenida legítimamente}, ya sea al momento de carga o al
momento de ejecución. 
\end{ejemplo}

\importante {Si un puntero no está explícitamente inicializado a
alguna dirección válida dentro del espacio del programa, se estará escribiendo
en \textbf{alguna dirección potencialmente prohibida}. En el mejor de los casos, el
programa intentará escribir en el espacio de memoria de otro proceso y el
sistema operativo \textbf{lo terminará}. En casos más sutiles, el programa continuará
funcionando pero luego de \textbf{corromper algún área de memoria impredecible} dentro
del espacio del proceso. 

Este problema es por lo general difícil de detectar,
porque el efecto puede no mostrar ninguna relación con el origen del problema.
El error puede estar en una instrucción que se ejecuta pero corrompe la
memoria, y sin embargo manifestarse recién cuando se accede a esa zona de
memoria corrupta. Para este momento, el control del programa puede estar en un
punto arbitrariamente lejano de la instrucción que causó el problema.
}

\begin{ejemplo}
La solución al problema de los punteros sin inicializar es asegurarse, \textbf{siempre}, que los punteros apuntan a \textbf{lugares válidos
del programa}, asignándoles \textbf{direcciones de objetos conocidos}. 

\begin{lstlisting}
/* Correcto */
	int a;
	int *p;
	p = &a;
	*p = 8;	
\end{lstlisting}
\end{ejemplo}


\subsection{Confundir punteros con arreglos}
Es imprescindible comprender rigurosamente la diferencia entre arreglos y
punteros. Aunque son intercambiables en algunos contextos, suponer que son lo
mismo lleva a graves errores. Es frecuente confundirlos, y esta confusión es
explicable a partir de algunos hechos que se enumeran a continuación. No hay que
dejarse engañar por ellos.

\begin{enumerate}
	\item \textbf{Ambos se evalúan a direcciones.}
	
El nombre de un arreglo equivale a una dirección, y usar un puntero equivale a
usar la dirección que contiene. Es decir, tienen usos similares.

\begin{enumerate}[label=\alph*.]

\item 
\begin{lstlisting}
char formato[] = "%d %d\n";
printf(formato, 5, -1);
\end{lstlisting} 

\item 
\begin{lstlisting}
char *formato = "%d %d\n";
printf(formato, 5, -1);
\end{lstlisting} 

Aquí usamos como equivalentes a un arreglo y a un puntero, porque de cualquiera
de las dos maneras estamos expresando una dirección en la invocación a \lstinline{printf()}.
\end{enumerate}



\item \textbf{Como argumentos formales, son equivalentes.}

En una función, un argumento formal que sea una dirección puede ser declarado
como puntero o como arreglo, intercambiablemente. Convirtamos los ejemplos de
más arriba a funciones:

\begin{enumerate}[label=\alph*.]
\item 
\begin{lstlisting}
int fun(char *s, int x, int y)
{
       printf(s, x, y);
}
\end{lstlisting} 

\item 
\begin{lstlisting}
int fun(char s[], int x, int y)
{
        printf(s, x, y);
}
\end{lstlisting} 
\end{enumerate}

Ambas formas son válidas, porque lo único que estamos expresando es que un
argumento es una dirección; y, como se ha dicho, tanto punteros como nombres de
arreglos los representan. Además, cualquiera de las funciones escritas puede
usarse en cualquiera de las dos maneras siguientes, pasando punteros o arreglos
en la llamada:

\begin{enumerate}[label=\alph*.]
\item 
\begin{lstlisting}
char formato[] = "%d %d\n";
   fun(formato, 5, -1);
\end{lstlisting} 

\item 
\begin{lstlisting}
char *formato = "%d %d\n";
   fun(formato, 5, -1);
\end{lstlisting} 
\end{enumerate}


\item \textbf{Comparten operadores.}

Se pueden aplicar los mismos operadores de acceso a ambos; a saber, se puede
dereferenciar un arreglo (igual que un puntero) para acceder a un elemento y se
puede indexar un puntero (como un arreglo) para acceder a una posición dentro
del espacio al que apunta.

\begin{enumerate}[label=\alph*.]
\item 
\begin{lstlisting}
char cadena[] = "abcdefghijkl";
   char c;
   c = *(cadena + 4); /* c = 'e' */
\end{lstlisting} 

\item 
\begin{lstlisting}
char *cadena = "abcdefghijkl";
   char c;
   c = cadena[4];     /* c = 'e' */
\end{lstlisting} 
\end{enumerate}

\end{enumerate}



En muchas formas, entonces, punteros y arreglos pueden ser intercambiados
porque son dos maneras de acceder a direcciones de memoria, pero \textbf{un arreglo no
es un puntero}, y \textbf{ninguno de ellos es una dirección} (aunque las representan),
porque:

\begin{itemize}
	\item 
Un arreglo tiene memoria asignada para todos sus elementos (desde la carga
      del programa, para arreglos globales o \lstinline{static}, y desde la entrada a la
      función donde se lo declara, para los arreglos locales).
	\item 
Un puntero, en cambio, solamente contiene una dirección, que puede ser o
      no válida en el sentido de apuntar o no a un objeto existente en el
      espacio direccionable por el programa. La validez de la dirección
      contenida en un puntero es responsabilidad del programador.
\end{itemize}

\subsection{No analizar el nivel de indirección}
Una variable de un tipo básico cualquiera contiene un dato que puede ser
directamente utilizado en una expresión para hacer cálculos. En cambio, un puntero que
apunte a esa variable contiene su dirección; es una \textbf{referencia} al dato, y
necesita ser \textbf{dereferenciado} para acceder al dato. La variable y su puntero
tienen \textbf{diferente nivel de indirección}. 

\figura[12]{indir}{Diferentes niveles de indirección.}{indireccion.eps}

Un \lstinline{char}, un \lstinline{int}, un \lstinline{long}, un \lstinline{double}, una estructura de un tipo definido por el
usuario, tienen un mismo nivel de indirección, que podríamos llamar \quotes{el nivel
0}. Un puntero, una dirección, un nombre de arreglo, tienen un \quotes{nivel de
indirección 1}. La Fig. \ref{fig:indir} esquematiza situaciones donde los niveles de indirección son \textbf{0, 1, y 2}.

Aplicar el operador \lstinline{&} (dirección de) a algo aumenta su nivel de
indirección. Aplicar el operador \lstinline{*} (dereferenciación) lo decrementa.

\importante{
Al escribir una expresión, o hacer una asignación, o proveer
argumentos reales para una función, etc., es necesario que los niveles de
indirección de los elementos que componen la expresión sean \textbf{consistentes}, es
decir, que el resultado final de cada subexpresión tenga el nivel de
indirección adecuado.

\medskip
El concepto de consistencia de niveles de indirección es análogo al que se usa al escribir una ecuación con magnitudes físicas, donde ambos miembros de la ecuación deben tener el mismo sentido físico. Por ejemplo, si
$V=velocidad$, $E=espacio$, $T=tiempo$, no tiene sentido en Física escribir la igualdad $V=E*T$,
simplemente porque si \textbf{multiplicamos} metros por segundo no obtenemos $m/s$, que son
las unidades de $V$. Del mismo modo podemos verificar la consistencia de las
expresiones en C preguntándonos qué nivel de indirección debe tener cada
subexpresión.
}



\begin{ejemplo}
Este tipo de análisis es sumamente útil para prevenir errores de programación.
Conviene utilizarlo para dar una segunda mirada crítica a las expresiones que
escribimos.

\begin{lstlisting}
char *s = "cadena";
char *t;
char u;
t = s + 2; /* CORRECTO */
u = s;     /* INCORRECTO */
u = *s;    /* CORRECTO */
t = &u;    /* CORRECTO */	
\end{lstlisting}

La asignación \lstinline{t = s + 2} es correcta porque la suma de una dirección más un
entero está definida y devuelve una dirección; con lo cual la expresión
mantiene el mismo nivel de indirección en ambos miembros (\textbf{puntero = dirección}).

En cambio, la asignación \lstinline{u = s} intenta asignar una dirección (la dirección contenida en \lstinline{s}) a un
char. No se respeta el mismo nivel de indirección en ambos miembros de la
asignación (\textbf{puntero = char}), de modo que ésta es incorrecta y será rechazada por el compilador.

En las dos últimas asignaciones se usan los operadores de dirección y de
indirección para hacer consistentes los niveles de indirección de ambos
miembros:
\begin{lstlisting}
		u = *s;
		char = lo apuntado por(puntero a char);
		char = char;
\end{lstlisting}

\begin{lstlisting}
		t = &u;
		puntero a char = direccion de(char);
		direccion de char = direccion de char;
\end{lstlisting}

\end{ejemplo}

\section{Arreglos de punteros}
Una construcción especialmente útil es la de \textbf{arreglos de punteros a carácter}.
Esta construcción permite expresar una lista de rótulos y navegar por ellos con
la indexación natural de los arreglos.

\begin{ejemplo}
Aquí el tipo de los elementos del arreglo mes es puntero a carácter. Cada
elemento se inicializa en la declaración a una constante string.

\begin{lstlisting}
char *mes[] = { "Ene","Feb","Mar","Abr","May","Jun",
                "Jul","Ago","Sep","Oct","Nov","Dic" };

printf("Mes: %s\n",mes[6]);
\end{lstlisting}


\end{ejemplo}


\section{Estructuras referenciadas por punteros}
En el caso particular de \textbf{estructuras o uniones referenciadas por punteros}, la
notación para acceder a sus miembros cambia ligeramente, reemplazando el
operador \quotes{punto} por \quotes{\lstinline{->}}.
\begin{ejemplo}
Accedemos a los miembros de la estructura \lstinline{persona} en forma directa:
\begin{lstlisting}
struct persona p;
p.DNI = 14233326;
p.edad = 40;
\end{lstlisting}
En forma indirecta:
\begin{lstlisting}
struct persona p, *pp;
pp = &p;
pp->DNI = 14233326;
pp->edad = 40;
\end{lstlisting}
\end{ejemplo}


\section{Estructuras de datos recursivas}
Las estructuras de datos recursivas se expresan efectivamente creando miembros que sean punteros a
estructuras del mismo tipo.
\begin{lstlisting}
struct itemlista {
    double dato;
    struct itemlista *proximoitem;
}
\end{lstlisting}

\begin{lstlisting}
struct nodoarbol {
    int valor;
    struct nodoarbol *hijoizquierdo;
    struct nodoarbol *hermanoderecho;
}
\end{lstlisting}

En cambio, no es legal la composición de estructuras dentro de sí mismas:
\begin{lstlisting}
struct itemlista { /* INCORRECTO */
    double dato;
    struct itemlista proximoitem;
}
\end{lstlisting}

\section{Construcción de tipos}
Aunque la construcción de \textbf{tipos definidos por el usuario} no es una
característica directamente ligada a los punteros o a las variables
estructuradas, es un momento apropiado para introducirla. El lenguaje C admite la
generación de nuevos nombres para tipos (en particular, los tipos estructurados) mediante la primitiva
\lstinline{typedef}.

\begin{ejemplo}
Las declaraciones del ejemplo anterior se podrían reescribir más claramente de
la forma que sigue.
\begin{lstlisting}
typedef struct nodoarbol {
    int valor;
    struct nodoarbol *hijoizquierdo;
    struct nodoarbol *hermanoderecho;
}nodo;
typedef struct nodoarbol *nodop;	
\end{lstlisting}

Entonces, el tipo de un argumento de una función podría quedar expresado
sintéticamente como \lstinline{nodop}:
\begin{lstlisting}
nodop crearnodo(nodop padre);
\end{lstlisting}

La construcción con \lstinline{typedef} no es indispensable, pero aporta claridad al estilo
de programación, y, bien manejada, promueve la portabilidad.
\end{ejemplo}


\section{Asignación dinámica de memoria}
Se ha visto la necesidad de que los punteros apunten a direcciones válidas.
¿Qué hacer cuando la lógica del programa pide la creación de estructuras de
datos en forma dinámica? Los punteros son muy convenientes para manejarlas,
pero se debe asegurar que apunten a zonas de memoria legítimamente obtenidas
por el programa.

En C se tiene como \textbf{herramientas básicas de gestión dinámica de memoria} a las
funciones \lstinline{malloc()} y \lstinline{free()}. Con \lstinline{malloc()} pedimos una cantidad de bytes
contiguos que serán tomados del \textbf{heap}. La función \lstinline{malloc()} devuelve la dirección
del bloque de memoria asignado. Esta dirección debe reservarse en un puntero
para uso futuro y para liberarla con \lstinline{free()}.

\begin{ejemplo}
En lugar de hacer que \lstinline{p} apunte a un objeto existente al momento de compilación,
solicitamos tanta memoria como sea necesaria para alojar un entero y ponemos a
\lstinline{p} apuntando allí. Luego podemos hacer la asignación. Luego del uso se debe
liberar la zona apuntada por \lstinline{p}.
\begin{lstlisting}
/* Correcto */
int *p;
p = malloc(sizeof(int));
*p = 8;
free(p);
\end{lstlisting}
Para ser completamente correcto, el programa del ejemplo debería verificar que \lstinline{malloc()} no
devuelva \textbf{NULL} por ser imposible satisfacer el requerimiento de memoria. El
símbolo \lstinline{NULL} corresponde a la dirección 0, o, equivalentemente, al \textbf{puntero
nulo}, y nunca es una dirección utilizable.
\end{ejemplo}




\begin{ejemplo}
La propiedad de poder aplicar indexación a los punteros hace que, virtualmente,
el C sea capaz de proporcionarnos \textbf{arreglos dimensionables en tiempo de
ejecución}. En efecto:

\begin{lstlisting}
double *tabla;
tabla = malloc(k);
tabla[50] = 15.25;
\end{lstlisting}

Estas líneas son virtualmente equivalentes a un arreglo de $k$ elementos \lstinline{double},
donde $k$, por supuesto, puede ser calculado en tiempo de ejecución.

Una variante de \lstinline{malloc()} es la función \lstinline{calloc()}, que solicita una cantidad dada de
elementos de memoria, de un tamaño también dado, y además garantiza que todo el
bloque de memoria concedido esté \textbf{inicializado con ceros binarios}.

\begin{lstlisting}
float *lista;
int i;
lista = calloc(k, sizeof(float));
for(i=0; i<k; i++)
    lista[i] = fun(i);
\end{lstlisting}
\end{ejemplo}



\section{Punteros a funciones}
Así como se pueden tomar las direcciones de los elementos de datos, es posible
manipular las \textbf{direcciones iniciales de los segmentos de código} representados
por las funciones de un programa C, mediante punteros a funciones. Esta
característica es \textbf{sumamente poderosa}.

La declaración de un puntero a función tiene una sintaxis algo complicada: debe
indicar \textbf{el tipo devuelto} por la función y los \textbf{tipos de los argumentos}. El nombre del puntero, con el signo de \quotes{puntero} incluido, debe estar \textbf{entre paréntesis} en la declaración.

\begin{ejemplo}
Definición de un puntero llamado \lstinline{p}, a una función que recibe dos enteros y devuelve un entero:
\begin{lstlisting}
int (*p)(int x, int y);
\end{lstlisting}

o también:
\begin{lstlisting}
int (*p)(int, int);
\end{lstlisting}

Los paréntesis alrededor de \lstinline{*p} son importantes: sin ellos, se define una
\textbf{función que devuelve un puntero a entero}, que no es lo que se pretende.

Asignación del puntero \lstinline{p}:
\begin{lstlisting}
int fun(int x, int y)
{
    ...
}
p = fun;
\end{lstlisting}
Uso del puntero \lstinline{p} para invocar a la función \lstinline{fun} cuya dirección tiene asignada:
\begin{lstlisting}
a = (*p)(k1, 20 - k2);
\end{lstlisting}
\end{ejemplo}

\section{Aplicación de punteros a funciones}
La Biblioteca Standard contiene una función, \lstinline{qsort()}, que realiza el ordenamiento de una
tabla de datos mediante el método de \textbf{Quicksort}. Para que pueda ser
completamente flexible (para poder ordenar datos de cualquier naturaleza y según cualquier criterio), la
función acepta a su vez una \textbf{función provista por el usuario}, que determina el
\textbf{orden de dos elementos}. 

Es responsabilidad del usuario, entonces, definir
cuándo un elemento es mayor que el otro, a través de esta función de
comparación. La función de ordenamiento recibe un puntero a la función de comparación y la
invoca repetidamente.


La función de comparación sólo debe aceptar \lstinline{p1} y \lstinline{p2}, dos punteros a un par de
datos, y seguir el protocolo siguiente:

\medskip
\medskip
\begin{tabular}{c|c}
Si & Devolver \\
\hline
\lstinline$*p1 < *p2$ & un número menor que cero\\
\hline
\lstinline$*p1 == *p2$ & cero\\
\hline
\lstinline$*p1 < *p2$ & un número mayor que cero\\
\end{tabular}
\medskip
\medskip


\begin{ejemplo}
La declaración de argumentos formales como \lstinline{void *} expresa que esos argumentos pueden tratarse de
direcciones de objetos de cualquier tipo.
\begin{lstlisting}
#include <stdlib.h>

struct p {
    ...
    char nombre[40];
    double salario;
    ...
} lista[100];

int cmpSalario(const void *p1, const void *p2)
{
    return p1->salario - p2->salario;
}

int cmpNombre(const void *p1, const void *p2)
{
    return strncmp(p1->nombre, p2->nombre, 40);
}
\end{lstlisting}

Con estas definiciones, la tabla \textbf{lista} se puede ordenar por uno u otro campo de
la estructura con las sentencias:
\begin{lstlisting}
qsort(lista, 100, sizeof(struct p),cmpSalario);
\end{lstlisting}
\begin{lstlisting}
qsort(lista, 100, sizeof(struct p),cmpNombre);
\end{lstlisting}
\end{ejemplo}


\section{Punteros a punteros}
La indirección mediante punteros puede ser \textbf{doble, triple}, etc. Los punteros
dobles tienen aplicación en el manejo de conjuntos de strings o matrices
bidimensionales. 

Como en el caso de punteros a funciones, esto brinda una gran
potencia, pero a costa de complicar enormemente la notación y la programación,
por lo que se recomienda no abordar el tema en un curso introductorio y, sólo
una vez dominadas las técnicas y conceptos básicos de punteros, referirse a una
fuente como el libro de Kernighan y Ritchie, 2ª edición.

\begin{ejemplo}
Un ejemplo de uso de punteros dobles. 
\begin{itemize}
	\item En estas sentencias, el puntero \lstinline{q} recibe la dirección de \lstinline{a}; pero \lstinline{p} recibe la dirección de \lstinline{q}. Luego, \lstinline{p} contiene \quotes{la dirección de la dirección} de \lstinline{a}. 
	\item Similarmente, la expresión \lstinline{*p} significa \quotes{lo apuntado por \lstinline{p}}; pero entonces \lstinline{**p} significa \quotes{lo apuntado por lo apuntado por p}.
\end{itemize}
\begin{lstlisting}
int **p;    /* un puntero doble */
int *q;
int a;
q = &a;		/* q recibe la direccion de a */
p = &q;		/* p recibe la direccion de q */
			
**p = 8;    /* la variable a recibe un valor 8 */
\end{lstlisting}
\end{ejemplo}

\section{Una herramienta: gets()}
\label{sec:lafunciongets}

Para facilitar la práctica damos la descripción de otra función de Biblioteca
Standard.

La función \lstinline{gets()} pide al usuario una cadena terminada por \lstinline{ENTER}. Recibe como
argumento un espacio de memoria (expresado por una dirección) donde copiará los
caracteres tipeados. El fin de la cadena recibirá automáticamente un \textbf{cero final} para hacerla compatible con
las funciones de tratamiento de strings de la Biblioteca Standard.

La función \lstinline{gets()} debe recibir la dirección de un área
de memoria legal para el programa y donde no haya riesgo de sobreescribir
contenidos. Es importante comprender la teoría dada en este capítulo para
evitar el \textbf{uso de un puntero no inicializado}, una situación de error frecuente 
cuando se utilizan \lstinline{gets()} y funciones similares. 


\begin{ejemplo}

Creamos un área de memoria utilizable por gets() simplemente definiendo un arreglo de caracteres del tamaño deseado. 
\begin{lstlisting}
main()
{
    char arreglo[100];
    gets(arreglo);
}
\end{lstlisting}

El puntero del siguiente ejemplo no está inicializado. Por ser una variable local,
contiene \textbf{basura}, y por lo tanto apunta a un lugar impredecible.
\begin{lstlisting}
main()
{
    char *s;
    gets(s); /* Mal!!! */
}
\end{lstlisting}

Podemos corregir el ejemplo anterior proveyendo espacio legítimo a donde apunte \lstinline{s}, reservándolo
estáticamente mediante la declaración del arreglo.

\begin{lstlisting}
main()
{
    char arreglo[100];
    char *s;
    s = arreglo;
    gets(s); 		/* ¡Ahora sí! */
}
\end{lstlisting}

En el tercer ejemplo usamos asignación dinámica y liberación de memoria. En un
sentido estricto, en este caso particular no es necesaria la liberación, porque
la terminación del programa devuelve todas las estructuras creadas
dinámicamente; pero es útil habituarse a la disciplina de aparear cada
invocación de \lstinline{malloc()} con el correspondiente \lstinline{free()}.

\begin{lstlisting}
main()
{
    char *s;
    s = malloc(100); /* ¡Ahora sí! */
    gets(s);
    printf("ingresado: %s\n",s);
    free(s);
}
\end{lstlisting}
\end{ejemplo}

\section{Ejercicios}
\begin{enumerate}

\item ¿Qué objetos se declaran en las sentencias siguientes?
\begin{enumerate}[label=\alph*.]
	\item double (*nu)(int kappa);
	\item int (*xi)(int *rho);
	\item long phi();
	\item int *chi;
	\item int pi[3];
	\item long *beta[3];
	\item int *(gamma[3]);
	\item int (*delta)[3];
	\item void (*eta[5])(int *rho);
	\item int *mu(long delta);
\end{enumerate}

\item Construir una función que reciba un arreglo de punteros a string A y un
string B, y busque a B en el array A, devolviendo su índice en el array, o bien
$-1$ si no se halla.
\item Construir una función iterativa que imprima una cadena en forma inversa.
Muestre una versión y una recursiva.
\item Construir un programa que lea una sucesión de palabras y las busque en un
pequeño diccionario. Al finalizar debe imprimir la cuenta de ocurrencias de
cada palabra en el diccionario.
\item Construir un programa que lea una sucesión de palabras y las almacene en un
arreglo de punteros a carácter.
\item Ordenar lexicográficamente el arreglo de punteros del ejercicio 4.
\end{enumerate}

%TODO Ejercicios_Adicionales Ejercicios_Avanzados
