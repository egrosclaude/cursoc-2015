\chapter{Tipos de datos y expresiones}

En general, las \textbf{expresiones} en C se construyen conectando, mediante \textbf{operadores}, diversos elementos, tales
como \textbf{identificadores} de variables, \textbf{constantes} e invocaciones de \textbf{funciones}. Cada uno de estos
elementos tiene un valor al tiempo de ejecución, y debe ocupar -al menos temporariamente, mientras se calcula el resultado de la expresión- un
lugar en memoria. Al evaluar cada expresión, el compilador crea, para alojar cada subexpresión de las
que la constituyen, \textbf{objetos de datos}, que pueden pensarse como espacio de memoria reservado temporariamente para
contener valores. Al completar el cálculo de la expresión, el resultado nuevamente debe ser alojado en un objeto de datos propio. Estos espacios de memoria son de diferentes \quotes{tamaños} (cantidades de bits) de
acuerdo al \textbf{tipo de dato} de la subexpresión.

Así, las expresiones y subexpresiones en C asumen siempre un \textbf{tipo de datos}: alguno de los tipos básicos del lenguaje, o
uno definido por el usuario. Una expresión, según las necesidades, puede convertirse de un tipo a otro.
El compilador hace esto a veces en forma \textbf{automática}. Otras veces, el programador fuerza una
\textbf{conversión de tipo} para producir un determinado resultado.

\section{Declaración de variables}

Los \textbf{tipos básicos} son:
\begin{itemize}
	\item \lstinline{char} (un elemento del tamaño de un byte)
	\item \lstinline{int} (un número entero con signo)
	\item \lstinline{long} (un entero largo)
	\item \lstinline{float} (un número en punto flotante)
	\item \lstinline{double} (un número en punto flotante, doble precisión)
\end{itemize}

Cuando declaramos una variable o forzamos una conversión de tipo, utilizamos una \textbf{especificación de
tipo}. Ejemplos de declaración de variables:

\begin{lstlisting}
char a;
int alfa,beta;
float x1,x2;
\end{lstlisting}

Los \textbf{tipos enteros} (\lstinline{char}, \lstinline{int} y \lstinline{long}) admiten los modificadores \lstinline{signed} (con signo) y \lstinline{unsigned} (sin signo). Un objeto de datos \textbf{unsigned} utiliza todos sus bits para representar magnitud; un \textbf{signed} utiliza un bit para signo, en representación complemento a dos.

El modificador \lstinline{signed} sirve sobre todo para explicitar el signo de los chars. El default para un \lstinline{int} es
signed; el default para \lstinline{char} puede ser signed o unsigned, dependiendo del compilador.

\begin{lstlisting}
unsigned int edad;
signed char beta;
\end{lstlisting}


Un int puede afectarse con el modificador \lstinline{short} (corto).

\begin{lstlisting}
short i;
unsigned short k;
\end{lstlisting}


Cuando en una declaración aparece sólo el modificador unsigned o short, y no el tipo, \textbf{se asume int}. El
tipo entero se supone el tipo básico manejable por el procesador, y es el tipo por omisión en varias
otras situaciones. Por ejemplo, cuando no se especifica el \textbf{tipo del valor devuelto} por una función.

El modificador long puede aplicarse también a float y a double. Los tipos resultantes pueden tener más
precisión que los no modificados.

\begin{lstlisting}
long float e; 
long double pi;
\end{lstlisting}

\section{Tamaños de los objetos de datos}

El lenguaje C no define el tamaño de los objetos de datos de un tipo determinado. Es decir, un entero
puede ocupar 16 bits en una implementación del compilador, 32 en otra, o aun 64. Un \lstinline{long} puede tener, o no, más bits
que un \lstinline{int}. Un \lstinline{short} puede ser, o no, más corto que un \lstinline{int}. Según \textbf{K\&R}, lo único seguro es que "\textit{un \lstinline{short} no es mayor que un \lstinline{int}, que a su vez no es mayor que \lstinline{long}}".

Por supuesto, distintos tamaños en bits implican diferentes rangos de valores. Si deseamos \textbf{portar} un
programa, hecho bajo una implementación del compilador, a otra, no es posible asegurar a priori el
rango que tomará un tipo de datos. La fuente ideal para conocer los rangos de los diferentes tipos, en
una implementación determinada, es --además del manual del compilador-- el header \lstinline{limits.h} de la
Biblioteca Standard. Debe recordarse que cualquier suposición que hagamos sobre el rango o tamaño
de un objeto de datos afecta la portabilidad de un programa en C.

Las siguientes líneas son parte de un archivo \lstinline{limits.h} para una implementación en particular:

\begin{lstlisting}
/* Minimum and maximum values a 'signed short int' can hold. */
#define SHRT_MIN	(-32768)
#define SHRT_MAX	32767
/* Maximum value an 'unsigned short int' can hold. (Minimum is 0.) */
#define USHRT_MAX	65535
/* Minimum and maximum values a 'signed int' can hold. */
#define INT_MIN	(-INT_MAX - 1)
#define INT_MAX	2147483647
/* Maximum value an 'unsigned int' can hold. (Minimum is 0.) */
#ifdef __STDC__
#define UINT_MAX 	4294967295U
#else
#define UINT_MAX  	4294967295
#endif
/* Minimum and maximum values a 'signed long int' can hold. */
#ifdef __alpha__
#define LONG_MAX 	9223372036854775807L
#else
#define LONG_MAX	2147483647L
#endif
#define LONG_MIN 	(-LONG_MAX - 1L)
/* Maximum value an 'unsigned long int' can hold. (Minimum is 0.) */
# ifdef __alpha__
#define ULONG_MAX	18446744073709551615UL
#else
#ifdef __STDC__
#define ULONG_MAX 	4294967295UL
#else
#define ULONG_MAX 4294967295L
#endif
#endif
\end{lstlisting}

Cuando una operación sobre una variable provoca overflow, no se obtiene ninguna indicación de error.
El valor sufre \textbf{truncamiento} a la cantidad de bits que pueda alojar la variable.

Así, en una implementación donde los ints son de 16 bits, si tenemos en una variable entera el máximo
valor positivo:

\begin{lstlisting}
int a; 
a=32767; /* a=0111111111111111 binario */
a=a+1;
\end{lstlisting}

Al calcular el nuevo valor de \lstinline{a}, aparece un 1 en el bit más significativo, lo cual, según la representación de los enteros, lo transforma en un negativo (el menor negativo que soporta el tipo de datos, -32768).

Si el int es sin signo:

\begin{lstlisting}
unsigned a;
a=65535; /* maximo valor de unsigned int */
a=a+1; 
\end{lstlisting}

el incremento de \lstinline{a} provoca overflow, y el nuevo valor de \lstinline{a} se trunca a 16 bits, volviendo así a 0.

Siempre se puede saber el tamaño en bits de un tipo de datos aplicando el operador \lstinline{sizeof()} a una
variable o a la especificación de tipo.

\section{Operaciones con distintos tipos}

En una expresión en C pueden aparecer componentes de diferentes tipos. Durante la evaluación de una
expresión cuyas subexpresiones sean de \textbf{tipos diferentes}, deberá tener lugar una \textbf{conversión}, ya sea
implícita o explícita, para llevar ambos operandos a un \textbf{tipo de datos común} con el que se pueda
operar. La forma en que el compilador resuelve las conversiones implícitas a veces provoca algunas
sorpresas.

\subsection{Truncamiento en asignaciones}

Para empezar, una asignación de una expresión de un tipo dado a una variable de un tipo \quotes{menor} (en el sentido
del tamaño en bits de cada objeto de datos), no
sólo es permitida en C, sino que la conversión se hace en forma automática y generalmente sin ningún
mensaje de tiempo de compilación ni de ejecución. Por ejemplo,
\begin{lstlisting}
int a;
float b;
...
a=b;
\end{lstlisting}

En esta asignación tenemos miembros de diferentes tamaños. El resultado en \lstinline{a} será el truncamiento
del valor entero de \lstinline{b} a la cantidad de bits que permita un int. Es decir, se tomará la parte entera de \lstinline{b}, y
de ese valor se copiarán en el objeto de datos de \lstinline{a} tantos bits como quepan en un int, tomándose
los menos significativos.

Si el valor de \lstinline{b} es, por ejemplo, 20.5, \lstinline{a} valdrá finalmente 20, lo que es similar a aplicar una función
\quotes{parte entera} implícitamente, y no demasiado incongruente. Pero si la parte entera de \lstinline{b} excede el
rango de un entero (por ejemplo si \lstinline{b=99232.5} en una plataforma donde los enteros son de 16 bits), el resultado en \lstinline{a} no tendrá lógica aparente. En el primer caso, los bits menos significativos de \lstinline{b} que \quotes{caben} en \lstinline{a} conservan el valor de \lstinline{b}; en el segundo caso, no.

En la sentencia:
\begin{lstlisting}
a=19.27 * b;	
\end{lstlisting}
\lstinline{a} contendrá los \lstinline{sizeof(int)} bits menos significativos del resultado de evaluar la expresión de la
derecha, truncada sin decimales.

\subsection{Promoción automática de expresiones}

Por otra parte, se tienen las reglas de promoción automática de expresiones. Enunciadas en forma
aproximada (luego las daremos con más precisión), estas reglas dicen que el compilador hará
estrictamente las conversiones necesarias para llevar todos los operandos al tipo del \quotes{mayor} entre ellos. El
resultado de evaluar una operación aritmética será del tipo del \quotes{mayor} de sus operandos.

A veces, esto no es lo que se desea. Por ejemplo, dada la sentencia:
\begin{lstlisting}
a=3/2;
\end{lstlisting}

se tiene que tanto la constante 3 como la constante 2 son vistas por el compilador como ints; el
resultado de la división será también un entero (la parte entera de 3/2, o sea 1). Aun más llamativo es
el hecho de que si declaramos previamente:
\begin{lstlisting}
float a;
\end{lstlisting}
el resultado es casi el mismo: \lstinline{a} contendrá finalmente el valor float 1.0, porque el problema de
truncamiento se produce ya en la evaluación del miembro derecho de la asignación.


\subsection{Operador cast}
En el ejemplo anterior, ¿cómo recuperar el valor correcto, con decimales, de la división? Declarar a la variable \lstinline{a} como \textbf{float} es necesario,
pero no suficiente. Para que la expresión del miembro derecho sea \textbf{float} es necesario que \textbf{al menos uno}
de sus operandos sea \textbf{float}. Hay dos formas de lograr esto; la primera es escribir cualquiera de las
subexpresiones como constante en punto flotante (con punto decimal, o en notación exponencial):
\begin{lstlisting}
a=3./2;
\end{lstlisting}

La segunda consiste en forzar explícitamente una conversión de tipo, con un importante operador
llamado \textbf{cast}, de la siguiente manera.
\begin{lstlisting}
a=(float)3/2;
\end{lstlisting}

El operador \textbf{cast} es la aclaración, entre paréntesis, del tipo al cual queremos convertir la expresión (en este caso, la subexpresión \textbf{3}). Da lo mismo aplicarlo a cualquiera de las constantes. Sin embargo, lo siguiente \textbf{no funcionará}:
\begin{lstlisting}
a=(float)(3/2);
\end{lstlisting}

Aquí el operador \textbf{cast} se aplica a la expresión \textbf{ya evaluada como entero}, con lo que volvemos a tener
un valor \textbf{1.0}, float, en \lstinline{a}.

\subsection{Reglas de promoción en expresiones}

Son aplicadas por el compilador en el orden que se da más abajo (tomado de K\&R, 2a. ed.). Ésta es
una lista muy detallada de las comprobaciones y conversiones que tienen lugar. Para la mayoría de los
propósitos prácticos, basta tener en cuenta la regla de \textbf{llevar ambos operandos al tipo del \quotes{mayor}} de
ellos.

Entendemos por \quotes{promoción entera} el acto de llevar los objetos de tipo \lstinline{char}, \lstinline{enum} y \textbf{campos de bits} a \textbf{int}; o, si los bits de un int no alcanzan a representarlo, a \textbf{unsigned int}.

\importante{
\begin{enumerate}
\item Si cualquier operando es long double, se convierte el otro a long double.
\item Si no, si cualquier operando es double, se convierte el otro a double.
\item Si no, si cualquier operando es float, se convierte el otro a float.
\item Si no, se realiza promoción entera sobre ambos operandos.
\item Si cualquiera de ellos es unsigned long int, se convierte el otro a unsigned long int.
\item Si un operando es long int y el otro es unsigned int, el efecto depende de si un long int puede representar a todos los valores de un unsigned int.
\item Si es así, el unsigned int es convertido a long int.
\item Si no, ambos se convierten a unsigned long int.
\item Si no, si cualquier operando es long int, se convierte el otro a long int.
\item Si no, si cualquier operando es unsigned int, se convierte el otro a unsigned int.
\item Si no, ambos operandos son int.
\end{enumerate}
}

\section{Observaciones}
Nótese que \textbf{no existen} tipos \lstinline{boolean} ni \lstinline{string}. Más adelante veremos cómo manejar datos de
estas clases.

El tipo \textbf{char}, pese a su nombre, no está restringido a la representación de caracteres. Por el contrario,
un char \textbf{tiene entidad aritmética}. Almacena una cantidad \textbf{numérica} y puede intervenir en
operaciones matemáticas. En determinadas circunstancias, y sin perder estas propiedades, puede
ser interpretado como un carácter (el \textbf{carácter cuyo código ASCII contiene}).

En general, en C es conveniente habituarse a pensar en los datos separando la \textbf{representación} (la
forma como se almacena un objeto) de la \textbf{presentación} (la forma como se visualiza). Un mismo
patrón de bits almacenado en un objeto de datos puede ser visto como un número decimal, un
carácter, un número hexadecimal, octal, etc. La verdadera naturaleza del dato es la representación
de máquina, el patrón de bits almacenado en el objeto de datos.


\section{Una herramienta: printf()}
\label{sec:lafuncionprintf}
Con el objeto de facilitar la práctica, describimos aquí la función de Biblioteca Standard \lstinline{printf()}.

\begin{itemize}
	\item La función de salida \lstinline{printf()} lleva un \textbf{número variable de argumentos}.
	\item Su primer argumento siempre es una cadena o constante string, la \textbf{cadena de formato},
conteniendo texto que será impreso, más, opcionalmente, \textbf{especificaciones de conversión}.
	\item Las especificaciones de conversión comienzan con un signo \quotes{\lstinline$\%$}. Todo otro conjunto de
caracteres en la cadena de formato será impreso textualmente.
	\item Cada especificación de conversión determina la manera en que se imprimirán los restantes
argumentos de la función.
	\item Deben existir tantas especificaciones de conversión como argumentos luego de la cadena de
formato.
	\item Un mismo argumento de un tipo dado puede ser impreso o presentado de diferentes maneras
según la especificación de conversión que le corresponda en la cadena de formato (de aquí la
importancia de separar representación de presentación)
	\item Las especificaciones de conversión pueden estar afectadas por varios \textbf{modificadores} opcionales
que determinan, por ejemplo, el ancho del campo sobre el cual se escribirá el argumento, la
cantidad de decimales de un número, etc.
	\item Las principales especificaciones de conversión están dadas en el Cuadro \ref{tab:conv}.
\end{itemize}

\tabla{conv}{Especificaciones de conversión de printf().}{c|l}
{
\lstinline$\%d$ & entero, decimal\\
\lstinline$\%u$ & entero sin signo, decimal\\
\lstinline$\%l$ & long, decimal\\
\lstinline$\%c$ & carácter\\
\lstinline$\%s$ & cadena\\
\lstinline$\%f$ & float\\
\lstinline$\%lf$ & double\\
\lstinline$\%x$ & entero hexadecimal\\
}

\begin{ejemplo}
Este programa escribe algunos valores con dos especificaciones de formato distintas.

\begin{lstlisting}
main() {
	int i,j;
	for(i=65, j=1; i<70; i++, j++)
		printf("vuelta no. %d: i=%d, i=%c\n",j,i,i);
}
\end{lstlisting}
Salida del programa:
\begin{lstlisting}
vuelta no. 1: i=65, i=A
vuelta no. 2: i=66, i=B
vuelta no. 3: i=67, i=C
vuelta no. 4: i=68, i=D
vuelta no. 5: i=69, i=E
\end{lstlisting}
\end{ejemplo}
\begin{ejemplo}
El programa siguiente escribe el mismo valor en doble precisión pero con diferentes
modificadores del campo correspondiente, para incluir una cierta cantidad de decimales o alinear la impresión. 
\begin{lstlisting}
main() {
	double d;
	d=3.141519/2.71728182;
	printf("d=%lf\n",d);
	printf("d=%20lf\n",d);
	printf("d=%20.10lf\n",d);
	printf("d=%20.5lf\n",d);
	printf("d=%.10lf\n",d);
}	
\end{lstlisting}
Salida del programa:
\begin{lstlisting}
d=1.156126
d=            1.156126
d=        1.1561255726
d=             1.15613
d=1.1561255726
\end{lstlisting}
\end{ejemplo}

\section{Ejercicios}

\begin{enumerate}
	\item ¿Cuáles de entre estas declaraciones contienen errores?
		\begin{multicols}{2}
		\begin{enumerate}[label=\alph*.]
			\item integer a;
			\item short i,j,k;
			\item long float (h);
			\item double long d3;
			\item unsigned float n;
			\item char 2j;
			\item int MY;
			\item float ancho, alto, long;
			\item bool i;
		\end{enumerate}
		\end{multicols}
	\item Dé declaraciones de variables con tipos de datos adecuados para almacenar:
	\begin{enumerate}[label=\alph*.]
		\item La edad de una persona.
		\item Un número de DNI.
		\item La distancia, en Km, entre dos puntos cualesquiera del globo terrestre.
		\item El precio de un artículo doméstico.
		\item El valor de la constante PI expresada con 20 decimales.
	\end{enumerate}
	\item Prepare un programa con variables conteniendo los valores máximos de cada tipo entero, para
comprobar el resultado de incrementarlas en una unidad. Imprima los valores de cada variable antes y
después del incremento. Incluya \textbf{unsigneds}.
	\item Lo mismo, pero dando a las variables los valores mínimos posibles, e imprimiéndolas antes y
después de decrementarlas en una unidad.
	\item Averigüe los tamaños de todos los tipos básicos en su sistema aplicando el operador \lstinline{sizeof()}.
	\item Si se asigna la expresión $(3-5)$ a un \lstinline{unsigned short}, ¿cuál es el resultado? ¿Depende de qué formato de conversión utilicemos para imprimirlo?
	\item ¿Qué hace falta corregir para que la variable \lstinline{r} contenga la división exacta de \lstinline{a} y \lstinline{b}?
	\begin{lstlisting}
int a, b;
float r;
a = 5;
b = 2;
r = a / b;		
	\end{lstlisting}
	\item ¿Qué resultado puede esperarse del siguiente fragmento de código?
	\begin{lstlisting}
int a, b, c, d;
a = 1;
b = 2;
c = a / b;
d = a / c;
	\end{lstlisting}
	\item ¿Cuál es el resultado del siguiente fragmento de código? Anticipe su respuesta en base a lo dicho en
esta unidad y luego confírmela mediante un programa.
	\begin{lstlisting}
printf("%d\n", 20/3);
printf("%f\n", 20/3);
printf("%f\n", 20/3.);
printf("%d\n", 10%3);
printf("%d\n", 3.1416);
printf("%f\n", (double)20/3);
printf("%f\n", (int)3.1416);
printf("%d\n", (int)3.1416);
	\end{lstlisting}
\item Escribir un programa que multiplique e imprima $100000 * 100000$. ¿De qué tamaño son los ints
en su sistema?
\item Convertir una moneda a otra sabiendo el valor de cambio. Dar el valor a dos decimales.
\item Escriba y corra un programa que permita saber si los chars en su sistema son signed o unsigned.
\item Escriba y corra un programa que asigne el valor 255 a un char, a un unsigned char y a un signed
char, y muestre los valores almacenados. Repita la experiencia con el valor \lstinline$-1$ y luego con \lstinline$'\377'$.
Explicar el resultado.
\item Copiar y compilar el siguiente programa. Explicar el resultado.
	\begin{lstlisting}
main() {
	double x;
	int i;
	i = 1400;
	x = i; /* conversion de int a double */
	printf("x = %10.6le\ti = %d\n",x,i);
	x = 14.999;
	i = x; /* conversion de double a int */
	printf("x = %10.6le\ti = %d\n",x,i);
	x = 1.0e+60;
	i = x;
	printf("x = %10.6le\ti = %d\n",x,i);
}
	\end{lstlisting}

\item Escriba un programa que analice la variable \lstinline{v} conteniendo el valor 347 y produzca la salida:
	\begin{lstlisting}
3 centenas
4 decenas
7 unidades
\end{lstlisting}
(y, por supuesto, salidas acordes si \lstinline{v} toma otros valores).
	\item Sumando los dígitos de un entero escrito en notación decimal se puede averiguar si es divisible por
3 (se constata si la suma de los dígitos lo es). ¿Esto vale para números escritos en otras bases? ¿Cómo
se puede averiguar esto?
	\item Indicar el resultado final de los siguientes cálculos
\begin{enumerate}[label=\alph*.]
\item \lstinline{int a; float b = 12.2; a = b;}
\item \lstinline{int a, b; a = 9; b = 2; a /= b;}
\item \lstinline{long a, b; a = 9; b = 2; a /= b;}
\item \lstinline{float a; int b, c; b = 9; c = 2; a = b/c;}
\item \lstinline{float a; int b, c; b = 9; c = 2; a = (float)(b/c);}
\item \lstinline{float a; int b, c; b = 9; c = 2; a = (float)b/c;}
\item \lstinline{short a, b, c; b = -2; c = 3; a = b * c;}
\item \lstinline{short a, b, c; b = -2; c = 3; a = (unsigned)b * c;}
\end{enumerate}
\item Aplicar operador cast donde sea necesario para obtener resultados apropiados:
\begin{enumerate}[label=\alph*.] 
	\item 
	\begin{lstlisting}
int a; long b; float c;
a = 1; b = 2; c = a / b;
	\end{lstlisting} 

	\item 
	\begin{lstlisting}
long a; 
int b,c;
b = 1000; c = 1000;
a = b * c;
	\end{lstlisting}
\end{enumerate}
\end{enumerate}
