


\chapter{Apuntadores y Direcciones}

El tema de esta unidad es el más complejo del lenguaje C y por este motivo se
han separado los contenidos en dos partes (llamadas 10 y 10b).

La memoria del computador está organizada como un vector o arreglo
unidimensional. Los índices en este arreglo son las direcciones de memoria.
Este arreglo puede accederse indexando a cada byte individualmente, y en
particular a cada estructura de datos del programa, mediante su dirección de
comienzo.

Para manipular direcciones se utilizan en C variables especiales llamadas
apuntadores o punteros, que son aquellas capaces de contener direcciones. En la
declaración de un apuntador se especifica el tipo de los objetos de datos cuya
dirección contendrá.

La notación:
\begin{lstlisting}
char *p;
\end{lstlisting}
es la declaración de una variable puntero a carácter. El contenido de la
variable \lstinline{p} puede ser, en principio, cualquiera dentro del rango de direcciones
de la máquina subyacente al programa. Una vez habiendo recibido un valor, se
dice que la variable \lstinline{p} apunta a algún objeto en memoria.

Esquemáticamente representamos la situación de una variable que contiene una
dirección (y por lo tanto \quotes{apunta a esa dirección}) según el diagrama siguiente
(mas\_información). La posición 1 de la memoria aloja un puntero que actualmente
apunta a la posición 5.

%[arreglo memoria]

El tema de apuntadores (o punteros) y direcciones es crucial en la programación
en C, y parece ser el origen más frecuente de errores. Programas con mala
lógica de acceso a memoria pueden ser declarados válidos por el compilador: su
compilación puede ser exitosa y sin embargo ser completamente erróneos en
ejecución. Esta es una de las críticas más frecuentes al lenguaje C, aunque en
rigor de verdad, el problema no es del lenguaje, sino del programador con una
mala comprensión de las cuestiones del lenguaje relacionadas con memoria.

Es fundamental, para no cometer estos errores, comprender en profundidad los
conceptos de direcciones y punteros, así como la sintaxis de las
declaraciones de punteros, para asegurarnos de que escribimos lo que se pretende
lograr.

\section{Operadores especiales}
Para manipular punteros se hacen necesarios dos operadores especiales:

\begin{tabular}{c|l}
%\lstinline{\&} & Operador \quotes{dirección de} o de referencia\\
%\lstinline{\*} & Operador de indirección o de dereferenciación\\
\end{tabular}


\begin{itemize}
	\item El operador \textbf{dirección} devuelve \textbf{la dirección de un objeto}. La construcción
siguiente:
\begin{lstlisting}
p = &a;
\end{lstlisting}
puede leerse: \quotes{asignar a \texttt{p} la dirección de \texttt{a}}.
\item El operador de \textbf{indirección}, o de \textbf{dereferenciación}, surte el efecto contrario:
accede al \textbf{objeto apuntado por} una dirección. La construcción
\begin{lstlisting}
a = *p;
\end{lstlisting}
puede leerse como \quotes{a es igual a lo apuntado por p}.
\end{itemize}


Para obtener el efecto lógicamente esperado, en las expresiones anteriores
\lstinline{p} deberá ser un \textbf{puntero}, capaz de recibir y entregar una dirección.

En general, si el contenido de la variable \texttt{p} es igual a la dirección de \lstinline{a}, es decir, \lstinline{&a}, la expresión de dereferenciación \lstinline{*p} puede aparecer
en cualquier contexto en el que apareciera \lstinline{a}. En particular, es legal asignar
indirectamente a través de un apuntador. 

\begin{ejemplo}
Las instrucciones
\begin{lstlisting}
int a, *p;
p = &a;
*p = 1;
\end{lstlisting}
equivalen a
\begin{lstlisting}
a = 1;
\end{lstlisting}
\end{ejemplo}

\section{Aritmética de punteros}
Son operaciones legales asignar punteros entre sí, sumar algebraicamente un
entero a un puntero y restar dos punteros. Las consecuencias de cada operación
se esquematizan en las figuras siguientes.

\subsection{Asignación entre punteros}
Luego de asignar un puntero a otro, ambos apuntan al mismo objeto. Cualquier
modificación al objeto apuntado por uno se refleja al accederlo mediante el
otro puntero.

\subsection{Suma de enteros a punteros}
La suma algebraica de una dirección más un entero es nuevamente una dirección.
El sentido de la operación es desplazar el punto de llegada del apuntador
(hacia arriba o hacia abajo en memoria) en tantas unidades como diga el entero,
con la particularidad de que el resultado final es dependiente del tamaño del
objeto apuntado. Esto es en general lo que desea el programador al aplicar un
incremento a un apuntador.

Es decir que sumar (o restar) una unidad a un puntero, lo incrementa
(decrementa) en tantos bytes como mida el objeto al cual apunta.
Por ejemplo, para punteros a carácter, la instrucción \lstinline{p++} incrementa el valor
del puntero en $1$, que es el \lstinline{sizeof()} de los \lstinline{chars}; pero si \lstinline{p} es un puntero a
\lstinline{long}, en una arquitectura donde los \lstinline{longs} miden cuatro bytes, \lstinline{p++} incrementa el
valor de \lstinline{p} en $4$ (y \lstinline{p} queda apuntando \quotes{un \lstinline{long} más allá} en la memoria). 

\importante{El
resultado al tiempo de ejecución para la expresión \lstinline{p + i}, donde \lstinline{p} es un
puntero a \lstinline{tipo} e \lstinline{i} es un entero, es siempre \lstinline{p + i * sizeof(tipo)}.
}

\subsection{Resta de punteros}
El sentido de una resta de punteros (o, equivalentemente, de una diferencia de
direcciones) es obtener el tamaño del área de memoria comprendida entre los
objetos apuntados por ambos punteros. La resta tendrá sentido únicamente si se
hace entre variables que apuntan a objetos del mismo tipo.

Nuevamente se aplica la lógica del punto anterior: el resultado obtenido
quedará expresado en unidades del tamaño del objeto apuntado. Es decir, si una
diferencia entre punteros a long da 3, debe entenderse el resultado como
equivalente a 3 longs, y por lo tanto a 3*sizeof(long) bytes.


\section{Punteros y arreglos}
Una consecuencia de que sea posible sumar enteros a punteros es que se puede
simular el recorrido de un arreglo mediante el incremento sucesivo de un
puntero. La operación de acceder a un elemento del arreglo es equivalente a
obtener el objeto apuntado por el puntero. Las sentencias:
\begin{lstlisting}
int *p;
int a[10];
p = &a[0];
\end{lstlisting}

Habilitan al programador para acceder a cada elemento del arreglo a mediante
aritmética sobre el puntero p. Como el nombre de un arreglo se evalúa a su
dirección inicial, la última sentencia también puede escribirse simplemente
así:
\begin{lstlisting}
p = a;
\end{lstlisting}

\begin{ejemplo}
Supongamos dadas las definiciones siguientes:
\begin{lstlisting}
int alfa[] = { 2, 4, 6, 7, 4, 2, 3, 1 };
int *p, *q;
int b;
\end{lstlisting}
Con estas definiciones, veamos algunas manipulaciones de arreglos y punteros. 

\begin{lstlisting}
p = alfa;        /* el nombre de un arreglo
                    equivale a su direccion    */
\end{lstlisting}

\begin{lstlisting}
*p = 3;          /* equivalente a alfa[0] = 3  */
*(p+2) = 4;      /* equivalente a alfa[2] = 4  */
b = *p;          /* equiv. a b = alfa[0]       */
*(p+3) = *(p+6); /* sobreescribe el 7 con el 3 */
\end{lstlisting}

\begin{lstlisting}
q = alfa + 2;    /* apunta al tercer elemento  */
\end{lstlisting}

\begin{lstlisting}
printf("%d\n",*q);                /* imprime 4 */
printf ("%d\n",q - p);            /* imprime 2 */
\end{lstlisting}

\begin{lstlisting}
p += q;          /* ERROR - la suma de punteros
                              no esta definida */
\end{lstlisting}
Los dos segmentos de código siguientes hacen exactamente la misma tarea pero de maneras diferentes.

% \begin{tabular}{l|l}
% \hline\\
% \codecell{
% int i;
% long tabla[10];
% for(i = 0; i < 10; i++)
%     suma += tabla[i];
% }
% &
% \codecell{
% long *p;
% long tabla[10];
% for(p = tabla; p < tabla+10; p++)
%     suma += *p;
% }\\
% \hline\\
% \end{tabular}
\end{ejemplo}


\section{Punteros y cadenas de texto}

Un caso típico del uso de punteros ocurre cuando se necesita
trabajar con \textbf{cadenas de texto}, a veces llamadas \textbf{strings}, o \textbf{constantes string}. 

Como se vio en la sección \ref{sec:constantesstring}, las constantes string son todas aquellas secuencias de caracteres (eventualmente la secuencia vacía) en el texto del programa, que aparecen entre comillas. La representación interna de las constantes string durante la compilación, con el \textbf{cero final}, y con la referencia a la constante reemplazada por su dirección, se trasladará al programa compilado y aparecerá en la memoria, en la zona de datos estáticos, al momento de ejecución del programa. 

Como la constante string es reemplazada por la dirección de su primer carácter, la asignación o inicialización de una constante string a un puntero es legal, y almacena en el puntero dicha dirección. En la inicialización de punteros, las constantes string son un caso especialmente frecuente.

\begin{ejemplo}
Un puntero a carácter inicializado con una constante string.
\begin{lstlisting}
char *s = "Lenguaje C";
\end{lstlisting}
\end{ejemplo}

\begin{ejemplo}
La inicialización
de \texttt{s} y la asignación de \texttt{t} cargan a ambas variables con las direcciones del primer
carácter, o direcciones iniciales, de las cadenas respectivas.
\begin{lstlisting}
char *s = "Esta es una cadena";
char *t;
t = "Esta es otra cadena";
\end{lstlisting}
\end{ejemplo}

Representamos en el diagrama el carácter 0 final (que no es imprimible) con el
símbolo \textcurrency. La expresión en C de este carácter es simplemente 0 (un entero) o
\lstinline{'\0'} (una constante carácter cuyo código ASCII es cero).

La función de biblioteca standard \lstinline{printf()} permite imprimir una cadena con el
especificador de conversión \lstinline{%s}.

\begin{ejemplo}
Las líneas siguientes;
\begin{lstlisting}
char *s = "Cadena de prueba";
char *t;
t = s + 7;
printf("%s\n", s);
printf("%s\n", t);
\end{lstlisting}
O bien, equivalentemente:
\begin{lstlisting}
char *s = "Cadena de prueba";
printf("%s\n", s);
printf("%s\n", s + 7);
\end{lstlisting}

imprimen:
\begin{lstlisting}
Cadena de prueba
de prueba
\end{lstlisting}
\end{ejemplo}

\begin{ejemplo}
Una función que recorre una cadena ASCIIZ buscando un carácter y devuelve la
primera dirección donde se lo halló, o bien el puntero nulo (NULL).
\begin{lstlisting}
char *donde(char *p, char c)
{
    for( ; *p != 0; p++)
        if(*p == c)
             return p;
    return NULL;
}

main()
{
    char *cadena = "Buscando exactamente esto";
    char *s;
    s = donde(cadena, 'e');
    if(s != NULL)
        printf("%s\n", s);
}
\end{lstlisting}

El ejemplo de uso imprime:
\begin{lstlisting}
exactamente esto
\end{lstlisting}
\end{ejemplo}

\section{Pasaje por referencia}
En C, donde todo pasaje de parámetros a funciones se realiza \textbf{por valor}, los punteros
brindan una manera de entregar a las funciones \textbf{referencias a objetos}. 

\importante{
El pasaje
por referencia permite que una función pueda \textbf{modificar} un objeto que es local a
otra función.
Un pasaje por referencia de un objeto implica entregar a la función \textbf{la dirección del objeto}.
}

\begin{ejemplo}
Modificación de un objeto externo a una función.
La función \texttt{f2()} debe poner a cero una variable entera que es externa a ella, por lo cual el argumento
formal \texttt{h} debe ser la dirección de un entero.
\begin{lstlisting}
void f2(int *h)
{
    *h = 0;
}

int f1()
{
    int j,k;
    int *p;

    p = &j;
    f2(p);		/* le pasamos una direccion */
    f2(&k);		/* y tambien aqui           */
}
\end{lstlisting}
\end{ejemplo}

\begin{ejemplo}
Uso incorrecto de argumentos pasados por valor.
\begin{lstlisting}
void swap(int x, int y) /* incorrecta */
{
    int temp;
    temp = x;
    x = y;
    y= temp;
}
\end{lstlisting}
La función \texttt{swap()}, que podría ser usada por un algoritmo de ordenamiento para
intercambiar los valores de dos variables, está \textbf{incorrectamente escrita}, ya que
los valores que intercambia son los de sus argumentos, que vienen a estar al
nivel de variables locales. El uso de la función swap() no tendrá efecto en el
exterior de la misma. La versión correcta debe escribirse con pasaje por
referencia:
\begin{lstlisting}
void swap(int *x, int *y) /* correcta */
{
    int temp;
    temp = *x;
    *x = *y;
    *y = temp;
}
\end{lstlisting}

La invocación de \texttt{swap()} debe hacerse con las direcciones de los objetos a
intercambiar:
\begin{lstlisting}
int a, b;
swap(&a, &b);
\end{lstlisting}
\end{ejemplo}

\section{Punteros y argumentos de funciones}
En las funciones que reciben direcciones, los argumentos formales pueden tener
cualquiera de dos notaciones: como punteros, o como arreglos. No importa qué objeto 
sea exactamente el argumento real (arreglo o puntero): en ambos casos, la función únicamente
\textbf{recibe una dirección} y no sabe, ni le importa, cuál es la naturaleza real del objeto exterior a
ella.

\begin{ejemplo}
La función que busca un carácter en una cadena, vista más arriba, puede
escribirse correctamente así, cambiando el tipo del argumento formal. El uso es
exactamente el mismo que antes, sin cambios en la función que llama.
\begin{lstlisting}
char *donde(char p[], char c)
{
    int i;
    for(i=0 ; p[i] != 0; i++)
        if(p[i] == c)
             return p+i;
    return NULL;
}
\end{lstlisting}
Nótese que dentro del cuerpo de la función podemos seguir utilizando la
notación de punteros si lo deseamos, aun con la declaración del argumento
formal como arreglo.
\begin{lstlisting}
char *donde(char p[], char c) /* el argumento p es un arreglo */
{
    for( ; *p != 0; p++)      /* pero se le puede aplicar el  */
        if(*p == c)			  /* operador de dereferenciacion */
             return p;
    return NULL;
}
\end{lstlisting}
\end{ejemplo}

Del mismo modo, si quisiéramos, podríamos representar los argumentos como
punteros y manipular los datos con indexación. Todo esto se debe, por un lado,
a que las notaciones \texttt{*p} y \texttt{p[]}, para argumentos formales, expresan únicamente que
el argumento es una dirección; y por otro lado, a la equivalencia entre las
formas de acceso mediante apuntadores y mediante índices de arreglos.

¡Esto no quiere decir que punteros y arreglos sean lo mismo! Véanse las
observaciones en la próxima unidad.

\section{Ejercicios}

\begin{enumerate}
	\item Dado el programa siguiente, ¿a dónde apunta \texttt{k1}?
\begin{lstlisting}
main()
{
    int k;
    int *k1;
}
\end{lstlisting}
	\item Dado el programa siguiente, ¿a dónde apunta \texttt{m1}?
\begin{lstlisting}
int *m1;
main()
{
    ...
}	
\end{lstlisting}
	\item ¿Cuánto espacio de almacenamiento ocupa un arreglo de diez enteros? ¿Cuánto
espacio de almacenamiento ocupa un puntero a entero?
	\item Declarar variables \texttt{long} llamadas a, b y c, y punteros a \texttt{long} llamados p, q y r; y dar
las sentencias en C para realizar las operaciones siguientes. Para cada caso,
esquematizar el estado final de la memoria.
	\begin{enumerate}[label=\alph*.]
		\item Cargar p con la dirección de a. Si ahora escribimos \lstinline{*p = 1000}, ¿qué
      ocurre?
		\item Cargar r con el contenido de p. Si ahora escribimos \lstinline{*r = 1000}, ¿qué
      ocurre?
		\item Cargar q con la dirección de b, y usar q para almacenar una constante \texttt{4L}
      en el espacio de b.
		\item Cargar en c la suma de a y b, pero sin escribir la expresión $a+b$.
		\item Almacenar en c la suma de a y b pero haciendo todos los accesos a las
      variables en forma indirecta.
	\end{enumerate}
\item Compilar y ejecutar:
\begin{enumerate}[label=\alph*.]
		\item 
\begin{lstlisting}
main()
{
    char *a = "Ejemplo";
    printf("%s\n",a);
}
\end{lstlisting}

\item 
\begin{lstlisting}
main()
{
    char *a;
    printf("%s\n",a);
}
\end{lstlisting}		

\item 
\begin{lstlisting}
main()
{
    char *a = "Ejemplo";
    char *p;
    p = a;
    printf("%s\n", p);
}
\end{lstlisting}	
\end{enumerate}



	\item ¿Qué imprimirán estas sentencias?
	\begin{enumerate}[label=\alph*.]
\item \lstinline{char *s = "ABCDEFG";}
\item \lstinline{printf("%s\n", s);}
\item \lstinline{printf("%s\n", s + 0);}
\item \lstinline{printf("%s\n", s + 1);}
\item \lstinline{printf("%s\n", s + 6);}
\item \lstinline{printf("%s\n", s + 7);}
\item \lstinline{printf("%s\n", s + 8);}
\end{enumerate}
	\item ¿Son correctas estas sentencias? Bosqueje un diagrama del estado final de la
memoria para aquellas que lo sean.
	\begin{enumerate}[label=\alph*.]
\item \lstinline{char *a = "Uno";}
\item \lstinline{char *a, b; a = "Uno"; b = "Dos";}
\item \lstinline{char *a,*b ; a = "Uno"; b = a;}
\item \lstinline{char *a,*b ; a = "Uno"; b = *a;}
\item \lstinline{char *a,*b ; a = "Uno"; *b = a;}
\item \lstinline{char *a = "Dos"; *a = 'T';}
\item \lstinline{char *a = "Dos"; a = "T";}
\item \lstinline{char *a = "Dos"; *(a + 1) = 'i'; *(a + 2) = 'a';}
\item \lstinline{char *a, *b ; b = a;}
	\end{enumerate}
	\item Escribir funciones para:
	\begin{enumerate}[label=\alph*.]
\item Calcular la longitud de una cadena.
\item Dado un carácter determinado y una cadena, devolver la primera posición
      de la cadena en la que se lo encuentre, o bien $-1$ si no se halla.
\item Buscar una subcadena en otra, devolviendo un puntero a la posición donde
      se la halle.
	\end{enumerate}

	\item Escribir una función para reemplazar en una cadena todas las ocurrencias de
un carácter dado por otro, suponiendo:
	\begin{enumerate}[label=\alph*.]
\item Que no interesa conservar la cadena original, sino que se reemplazarán
      los caracteres sobre la misma cadena.
\item Que se pretende obtener una segunda copia, modificada, de la cadena
      original, sin destruirla.
	\end{enumerate}

	\item Escribir funciones para:
	\begin{enumerate}[label=\alph*.]
\item Rellenar una cadena con un carácter dado, hasta que se encuentre el $0$
      final, o hasta alcanzar $n$ iteraciones.
\item Pasar una cadena a mayúsculas o minúsculas.
	\end{enumerate}

	\item Reescriba dos de las funciones escritas en 8 y dos de las escritas en 10
usando la notación opuesta (cambiando punteros por arreglos).

\end{enumerate}
%TODO Ejercicios_Adicionales Ejercicios_Avanzados

