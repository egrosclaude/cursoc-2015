


\chapter{Entrada/Salida Standard}

El concepto de \textbf{Entrada/Salida} en C replica el de su ambiente nativo, el sistema
operativo UNIX, donde todos los archivos son vistos como una sucesión de bytes,
prescindiendo completamente de su contenido, organización o forma de acceso.
Además, en UNIX los dispositivos de entrada o salida llevan asociados \textbf{archivos
lógicos}, que son puntos de entrada implementados en software a dichos
dispositivos. 

Toda la comunicación entre un programa y el mundo externo, ya
sean archivos físicos o lógicos, se hace mediante las mismas funciones comunes.
Para la programación en C, la abstracción de un
archivo es simplemente un flujo de bytes o stream, que se maneja con
operaciones primitivas comunes y con independencia de cuál sea el origen y su destino de ese flujo.

Si bien el C no contiene palabras reservadas de entrada/salida (\textbf{E/S}), la
Biblioteca Standard sí provee un rico conjunto de funciones de E/S, tan amplio
que suele provocar confusión en quienes se aproximan por primera vez al
lenguaje. Primero veremos un resumen de las funciones de E/S standard
y luego nos concentraremos en la E/S sobre archivos. Aunque la información dada
aquí es suficiente para intentar la creación de programas simples, la E/S es un
tema notablemente complejo, y es aconsejable tener a mano el manual de las
funciones C de nuestro sistema. Para mejor orientarnos, agregamos un mapa de esta unidad en el Cuadro \ref{tab:mapaes}.


\begin{table}[htbp]
\centering
\begin{tabular}{|c|c|c|c|}
\hline
\multicolumn{4}{|c|}{Funciones de Entrada/Salida}\\
\hline
		E/S standard (\ref{sec:esstandard})& \multicolumn{3}{c|}{Sobre archivos (\ref{sec:esarchivos})}\\
		\hline
		De caracteres (\ref{subsec:esstandardcaract})& \multicolumn{2}{c|}{ANSI C (\ref{subsec:esarchivosansic})}  & POSIX (\ref{subsec:esarchivosposix})\\\cline{2-4}
		De líneas (\ref{subsec:esstandardlineas})& De caracteres (\ref{subsubsec:esarchivosansiccaract})& De acceso directo (\ref{subsec:esarchivosansicdirect})& \\
		Con formato (\ref{subsec:esstandardformato})& De líneas (\ref{subsubsec:esarchivosansiclineas})& & \\
		Sobre strings (\ref{subsec:esstandardstrings})& Con formato (\ref{subsubsec:esarchivosansicformato})& & \\
\hline
\end{tabular}
\caption{Clasificación de las funciones de E/S en C.}
\label{tab:mapaes}
\end{table}


\section{Funciones de E/S Standard}
\label{sec:esstandard}
Los programas C reciben tres \textbf{canales de comunicación} con el ambiente abiertos
antes de comenzar su ejecución. El uso típico de estos canales de comunicación
es la lectura del teclado y la impresión sobre pantalla, aunque, si el sistema
operativo lo soporta, la existencia de estos canales también hace posibles las poderosas técnicas de la
\textbf{redirección} y el \textbf{piping}. 

%TODO diagrama proceso con tres vias standard

La Biblioteca Standard provee funciones mínimas para estos usos,
quedando fuera de consideración algunas características indispensables en
programas de producción, como seguridad, validación de datos, o la posibilidad de
organizar la salida en pantalla. Por ejemplo, no hay una forma canónica de
borrar la pantalla en C, ya que ésta es una función que depende fuertemente de
la plataforma y del ambiente donde se ejecute el programa. Las características faltantes en la
E/S standard se compensan recurriendo a bibliotecas de terceras partes.

\section{E/S standard de caracteres}
\label{subsec:esstandardcaract}
Las funciones de E/S standard de caracteres son \lstinline{getchar()} y \lstinline{putchar()}. Las funciones \lstinline{getchar()} y \lstinline{putchar()} leen de teclado e imprimen, respectivamente, \textbf{un carácter por vez}.

\begin{ejemplo}
Un programa que copia la entrada en la salida, carácter a
carácter. Puede usarse, con redirección, para crear o copiar archivos, como un
clon del comando \lstinline{cat} de UNIX.

\begin{lstlisting}
#include <stdio.h>
main()
{
    int a;
    while((a = getchar()) != EOF)
        putchar(a);
}
\end{lstlisting}
\end{ejemplo}

\section{E/S standard de líneas}
\label{subsec:esstandardlineas}
Las funciones \lstinline{gets()} y \lstinline{puts()} leen de teclado e imprimen, respectivamente,
\textbf{líneas de caracteres} terminadas por la señal de fin de línea \lstinline{\n}. La función
\lstinline{gets()} debe recibir como argumento \textbf{la dirección de un buffer o zona de memoria}
donde depositar los caracteres provenientes de entrada standard. Éstos pueden ser tipeados por 
el usuario, o, gracias a la redirección, provenir de archivos, o ser
resultado de la ejecución --eventualmente concurrente-- de otros programas. 

Es un error muy frecuente ofrecer a \lstinline{gets()} un puntero
no inicializado. La función \lstinline{gets()} ha sido descripta en la unidad sobre
apuntadores y direcciones (ver Sección \ref{sec:lafunciongets}).


\begin{ejemplo}
El mismo programa, pero orientado a copiar un stream de texto línea por línea.
La constante \lstinline{BUFSIZ} está definida en \lstinline{stdio.h} y es el tamaño del buffer de estas
funciones. Se puede sugerir esta elección para el buffer del programa, salvo
que haya motivos para proporcionar otro tamaño.

\begin{lstlisting}
#include <stdio.h>
main()
{
    char area[BUFSIZ];
    while(gets(area) != NULL)
        puts(area);
}
\end{lstlisting}
\end{ejemplo}

La función \lstinline{gets()} elimina el \lstinline{\n} final con que termina la línea antes de
almacenarla en su buffer. La función \lstinline{puts()} lo repone.

\importante{
Como el tamaño del buffer no es argumento para la función
\lstinline{gets()}, ésta no conoce los límites del área de memoria de que dispone para
dejar los resultados de una operación de entrada, y por lo tanto no puede hacer
verificación en tiempo de ejecución. Podría ocurrir que una línea de entrada
superara el tamaño del buffer: entonces esta entrada corromperá algún otro
contenido del espacio del programa. 

Esta condición se conoce como \textit{buffer overflow}, y el comportamiento en este caso queda indefinido; dando lugar,
inclusive, a problemas de seguridad. Por este motivo \lstinline{gets()} no es utilizada en
programas de producción.
}

\section{E/S standard con formato}
\label{subsec:esstandardformato}
Las funciones \lstinline{printf()} y \lstinline{scanf()} permiten imprimir e ingresar, respectivamente,
\textbf{conjuntos de datos en formato legible, descriptos por cadenas de formato}. Las
cadenas se componen de especificaciones de conversión y son simétricamente las
mismas para ambas funciones. La función \lstinline{printf()} y las cadenas de formato han
sido descriptas en la unidad correspondiente a tipos de datos (ver sección \ref{sec:lafuncionprintf}).

Inversamente a \lstinline{printf()}, la función \lstinline{scanf()} buscará en la entrada standard
\textbf{patrones de caracteres} que estén de acuerdo con las especificaciones de conversión. Generará
representaciones internas para los datos leídos, y los almacenará en variables.
Para esto debe recibir las \textbf{direcciones} de dichas variables donde almacenar los
elementos detectados en la entrada standard. Es un error frecuente ofrecerle,
como argumentos, \textbf{las variables, y no las referencias} a las mismas.


\begin{ejemplo}
Leer de entrada standard un valor entero y un \lstinline{long}, e imprimirlos. Controlar que efectivamente se haya logrado leer ambos valores antes de imprimir las variables.
\begin{lstlisting}
main() 
{
    int a, long b;
    if(scanf("%d %ld", &a, &b) != 2)
        exit(1);
    printf("a=%d, b=%ld\n", a, b);
}
\end{lstlisting}
\end{ejemplo}

\importante{
El valor devuelto por \lstinline{scanf()} es la \textbf{cantidad de datos leídos y convertidos
exitosamente}, y debería siempre comprobarse que es igual a lo esperado.
}

El uso de \lstinline{scanf()} es con frecuencia problemático. La función \lstinline{scanf()} consumirá
toda la entrada posible, pero se detendrá al encontrar un error (una entrada
que no corresponda a lo descripto por la especificación de conversión) y \textbf{dejará
el resto de la entrada sin procesar}, en el buffer de entrada. Si luego otra
función de entrada intenta leer, se encontrará con esta entrada no consumida,
lo cual puede dar origen a problemas de ejecución difíciles de diagnosticar. El
error parecerá producirse en una instrucción \textbf{posterior} a la invocación de \lstinline{scanf()} culpable. 

Por esta razón suele ser difícil combinar el uso de \lstinline{scanf()} con el de 
otras funciones de entrada/salida. Además, no hay manera directa de validar que
la entrada quede en el rango del tipo de datos destino.

El uso más recomendable de \lstinline{scanf()} es cuando se la utiliza para leer, mediante
redirección, un flujo \textbf{generado automáticamente} por otro programa (y que, por lo
tanto, tiene una gramática rigurosa y conocida).

\section{E/S standard sobre strings}
\label{subsec:esstandardstrings}

La misma lógica de las funciones de E/S con formato sirve para que otras
funciones lean variables con formato de un string, o impriman variables
formateadas sobre una cadena. El efecto de \lstinline{sprintf()} sobre su cadena argumento
es el mismo que tendría \lstinline{printf()} sobre salida standard. Por su parte la función 
\lstinline{sscanf()} lee de un string en memoria, conteniendo datos en formato legible, y los
recoge en representación binaria en variables, lo mismo que si \lstinline{scanf()} los
hubiera leído de entrada standard.

\begin{ejemplo}
\begin{lstlisting}
main()
{
    char area[1024];
    int a; long b;
    
	printf("1: %d %ld\n", -6534, 1273632);
    sprintf(area, "%d %ld", -6534, 1273632);
	printf("2: %s\n",area);
    sscanf(area, "%d %ld", &a, &b);
    printf("3: %d %ld\n", a, b);
}
\end{lstlisting}

El resultado debería ser:
\begin{lstlisting}
1: -6534 1273632
2: -6534 1273632
3: -6534 1273632
\end{lstlisting}
\end{ejemplo}

\chapter{E/S sobre archivos}
\label{sec:esarchivos}

Diferentes sistemas operativos tienen diferentes \textbf{sistemas de archivos} y
diferentes \textbf{modelos de archivos}. Los sistemas de archivos son los conjuntos de
funciones particulares que cada sistema ofrece para acceder a los archivos y a
la estructura de directorios que soporta. Los modelos de archivos son aquellas
convenciones de formato u organización que son particulares de un sistema
operativo o plataforma.

Por ejemplo, tanto DOS como UNIX soportan la noción de archivo de texto, pero
con algunas diferencias a nivel de modelo de archivos. En ambos, un archivo de
texto es una secuencia de líneas de texto, donde cada línea es una secuencia de
caracteres terminado en fin de línea; pero en DOS, la convención de fin de
línea es un par de caracteres (\textbf{CR, LF}) (equivalentes a ASCII 13 y 10, respectivamente) mientras
que para UNIX, un fin de línea equivale solamente a \textbf{LF} (ASCII 10). Además DOS, a diferencia de UNIX,
almacena un carácter especial (\textbf{EOF} o ASCII 26) al final de los archivos 
de texto para señalar el fin del archivo.


Por otro lado, diferentes sistemas de archivo proveen diferentes vistas sobre
diferentes implementaciones. Un sistema operativo puede soportar o no la noción
de directorio, o la de links múltiples; o puede fijar determinadas condiciones
sobre los nombres de archivos, todo esto en función de la organización íntima
de sus estructuras de datos.

\section{Estándares ANSI y POSIX}
Siendo uno de sus objetivos de diseño el favorecer la producción de
programas portables, el lenguaje C contempla la forma de resolver estos problemas de
manera fácil para los programadores. Las funciones de entrada/salida sobre
archivos de la Biblioteca Standard están divididas en dos grandes regiones: el
conjunto de funciones del C standard, también llamadas funciones de 
entrada/salida \textbf{bufferizada}, definidas por ANSI, y las 
\textbf{funciones POSIX}, también llamadas
funciones de entrada/salida \textbf{de bajo nivel}.

Las funciones ANSI C tienen el objetivo de ocultar a los programas las
particularidades de la plataforma, haciéndolos plenamente portables a pesar de
las diferencias conceptuales y de implementación de entrada/salida entre los
diferentes sistemas operativos. Estas funciones resuelven, por ejemplo, el
clásico problema de las diferentes convenciones sobre los delimitadores de un
archivo de texto. Es decir, están orientadas a resolver los problemas de
incompatibilidad inherentes al modelo de archivos. En cambio, el esfuerzo de
estandarización de POSIX apunta a establecer (aunque no solamente en lo
referente a los archivos) una interfaz uniforme entre compilador y sistema
operativo, proveyendo primitivas de acceso a los archivos con un comportamiento
claramente determinado, independientemente de cuál sea el sistema operativo
subyacente. Así, las funciones POSIX resuelven problemas de consistencia entre
diferentes sistemas de archivos.

Las funciones del ANSI C son las más comúnmente utilizadas por el programador,
pero se apoyan en funcionalidad suministrada por las funciones POSIX (de nivel
más bajo), que también están disponibles y son las recomendadas cuando las
restricciones del problema exceden a las funciones ANSI. La característica
fundamental de las funciones ANSI es la entrada/salida bufferizada. Por
ejemplo, una operación de escritura solicitada por una instrucción del programa
no se efectiviza inmediatamente sino que se realiza sobre un buffer intermedio,
administrado por las funciones de biblioteca standard y con su política propia
de \textit{flushing} o descarga al dispositivo. 

Las funciones de entrada/salida bufferizada reciben argumentos a imprimir y
los van depositando en un buffer o zona de memoria intermedia.
Cuando el buffer se llena, o cuando aparece un carácter de fin de línea, el
buffer se descarga al dispositivo, escribiéndose los contenidos del buffer en
pantalla, disco, etc. En cambio, las funciones POSIX hacen E/S 
directa a los dispositivos (o al menos, al sistema de E/S del sistema
operativo) y por esto son las preferidas para la programación de drivers,
servidores, etc., donde la performance y otros detalles finos deban ser
controlados más directamente por el programa.

\section{Funciones ANSI C de E/S sobre archivos}
\label{subsec:esarchivosansic}
Las funciones ANSI realizan todas las operaciones sobre archivos por medio de
una estructura o bloque de control cuyo tipo en C se llama \lstinline{FILE}. Esta
estructura está definida en el header \lstinline{stdio.h} y contiene, entre otras cosas,
punteros a buffers para escritura y lectura. La primera operación necesaria es
la \textbf{apertura} del archivo, que construye una estructura \lstinline{FILE}, la inicializa con
valores adecuados y \textbf{devuelve un apuntador} a la misma (es decir, un valor de tipo \lstinline{FILE *}). El apuntador servirá para
referenciar esa estructura durante todo el trabajo con el archivo y hasta que deba ser
cerrado.

En la apertura del archivo corresponde indicar el \textbf{modo de acceso} (la clase de
operaciones que se van a hacer sobre él). Como algunos sistemas operativos
(notoriamente, el DOS) distinguen entre archivos de texto y binarios, el ANSI C
provee dos formas de apertura, para indicar cómo se va a tratar el archivo.
Cuando un archivo se abre en \textbf{modo de texto}, durante las operaciones de lectura
y escritura se aplicarán las conversiones de fines de línea y de fin de archivo
propias de la plataforma. Para los archivos abiertos en \textbf{modo binario}, no se
aplicarán conversiones.

En sistemas conformes a POSIX, como UNIX, no hay realmente diferencia entre los
dos modos de apertura. Si se desea especificar una apertura de archivo en \textbf{modo
binario} (para asegurar la portabilidad) se añade una \textbf{b} a la especificación de
modo (por ejemplo, como en \lstinline{"wb+"}). En el Cuadro \ref{tab:modos} se resumen las especificaciones de modos de acceso en apertura de archivos. 

\begin{table}[hbtp]
\centering
\begin{tabular}{c|l}
Modo & Operación \\
\hline
\begin{codecell}
"r"
\end{codecell} & Abre un archivo que ya existe para lectura. \\
				& La lectura se realiza al inicio del archivo.\\
\hline
\begin{codecell}
"w"
\end{codecell} & Se crea un nuevo archivo para escribir. \\
				& Si el archivo ya existe, se inicializa y se sobreescribe.\\
\hline
\begin{codecell}
"a" 
\end{codecell}& Abre un archivo que ya existe, para agregar información al final.\\ 
				& Sólo se puede escribir a partir del final del archivo.\\
\hline
\begin{codecell}
"r+"
\end{codecell}& Abre un archivo que ya existe, para actualizarlo \\
				& (tanto para lectura como para escritura).\\
\hline
\begin{codecell}
"w+"
\end{codecell}& Crea un nuevo archivo para actualizarlo (lectura y escritura); \\
				& si ya existe, lo sobreescribe.\\
\hline
\begin{codecell}
"a+"
\end{codecell}& Abre un archivo para añadir información al final. \\
				& Si no existe, lo crea.\\
\end{tabular}
\caption{Modos de acceso de archivos en E/S ANSI}
\label{tab:modos}
\end{table}

Como en la entrada/salida standard, para manejar archivos tenemos funciones
para E/S \textbf{de caracteres, de líneas y con formato}.

\section{Funciones ANSI C de caracteres sobre archivos}
\label{subsubsec:esarchivosansiccaract}
Las funciones que leen y escriben sobre archivos \textbf{un carácter por vez} son \lstinline{fgetc()} y \lstinline{fputc()}. Ejemplo que copia un archivo:

\begin{lstlisting}
#include <stdio.h>
main()
{
    FILE *fp1, *fp2;
    int a;
    if(((fp1 = fopen("ejemplo.txt","r")) == NULL) ||
                    ((fp2 = fopen("copia.txt","w")) == NULL))
        exit(1);
    while((a = fgetc(fp1)) != EOF)
        fputc(a, fp2);
    fclose(fp1);
    fclose(fp2);
}
\end{lstlisting}

Hay otras funciones dentro de esta categoría, como \lstinline{ungetc()}, que \textbf{devuelve un
carácter} al flujo de donde se leyó.

\section{Funciones ANSI C de líneas sobre archivos}
\label{subsubsec:esarchivosansiclineas}
Mismo ejemplo anterior, en base a líneas, usando las funciones \lstinline{fgets()} y \lstinline{fputs()}. La
declaración de variables \lstinline{FILE *} y las sentencias de apertura y cierre de
archivos son idénticas al caso anterior. Enunciamos solamente el lazo
principal.
    
\begin{lstlisting}
    char area[BUFSIZ];
    ...
    while(fgets(area, BUFSIZ, fp1) != NULL)
        fputs(area, fp2);
\end{lstlisting}

\section{Funciones ANSI C con formato sobre archivos}
\label{subsubsec:esarchivosansicformato}

Existen funciones \lstinline{fprintf()} y \lstinline{fscanf()}, casi idénticas a \lstinline{printf()} y \lstinline{scanf()},
pero donde se especifica el stream de entrada o de salida. Las funciones \lstinline{printf()} y \lstinline{scanf()} pueden verse como el caso particular de las primeras donde el
stream es \textbf{stdout} o \textbf{stdin}, respectivamente.

\begin{ejemplo}
Leer dos variables de un archivo y escribirlas, en el
mismo formato, en un segundo archivo.
\begin{lstlisting}
#include <stdio.h>
main()
{
    FILE *fp1, *fp2;
    int a; long b;
    if(((fp1 = fopen("ejemplo.txt","r")) == NULL) ||
                    ((fp2 = fopen("copia.txt","w")) == NULL))
        exit(1);
    if(fscanf(fp1, "%d %ld", &a, &b) != 2)
        exit(1);
    fprintf(fp2, "%d %ld\n", a, b);
    fclose(fp1);
    fclose(fp2);
}
\end{lstlisting}
\end{ejemplo}



\section{Funciones ANSI C de acceso directo}
\label{subsec:esarchivosansicdirect}

Un conjunto muy útil de funciones ANSI permite el \textbf{acceso directo, aleatorio, o
random} sobre archivos, saltando por encima del modelo de E/S secuencial que
domina al resto de las funciones. Las funciones básicas de acceso directo son
\lstinline{fread()} y \lstinline{fwrite()}, que leen bloques de un tamaño dado y en una cantidad dada.

El acceso directo a un archivo se logra manteniendo un dato en su estructura \lstinline{FILE}, 
llamado el \textbf{puntero de lectura/escritura},
correspondiente al punto dentro del archivo donde se efectuará la próxima lectura o escritura  
cuando el programa lo requiera. Cada lectura o escritura de una cierta cantidad de bytes
avanza este puntero en esa cantidad de bytes, dejando el puntero listo para la siguiente 
operación a continuación de la última.
 
Sin embargo, el valor del puntero de lectura/escritura puede consultarse o modificarse arbitrariamente, con funciones específicas. Cambiando explícitamente el valor de este dato, podemos hacerlo  
apuntar a cualquier dirección dada, u \textit{offset}, dentro del archivo. Si hacemos que el puntero de escritura sea 
igual al tamaño en bytes del archivo, la próxima escritura se hará a partir del final del
archivo. Si hacemos que sea igual a 0, la próxima lectura (o escritura) se hará a partir del comienzo del archivo. Del mismo modo se puede modificar un trozo de información, de cualquier tamaño dado, en cualquier lugar del archivo. 

Las funciones ANSI de acceso directo utilizan el puntero de lectura/escritura. Son ideales para lectura y escritura directa de estructuras de datos, ya sea que se trate de elementos individuales u organizados en arreglos. Al poder
posicionarse el \textbf{puntero de lectura o escritura} en zonas arbitrarias de los
archivos, se logra la capacidad de \textbf{E/S por registros con acceso aleatorio}.


\begin{ejemplo}
En el ejemplo suponemos que se han grabado en el archivo varios registros cuyo
formato está representado por la estructura de la variable \lstinline{datos}. La función
\lstinline{fseek()} posiciona el puntero de lectura en el \textit{offset} \lstinline{10 * sizeof(...)} (que debe
ser un \lstinline{long}), significando que necesitamos acceder al \textbf{registro lógico} número 10 del
archivo. A continuación se leen tantos bytes como mide un elemento de datos.

Cambiando el tercer argumento de \lstinline{fread()} podemos leer en un solo acceso un
vector completo de estas estructuras en lugar de un elemento individualmente.
Luego de cambiar un valor del registro se lo vuelve a grabar, esta vez en un
offset distinto (correspondiente al registro lógico 5).
\begin{lstlisting}
struct registro {
    int dato1;
    long dato2;
} datos;
...
fseek(fp, 10L * sizeof(struct registro), SEEK_SET);
fread(&datos, sizeof(struct registro), 1, fp);
datos.dato1 = 1;
fseek(fp, 5L * sizeof(struct registro), SEEK_SET);
fwrite(&datos, sizeof(struct registro), 1, fp);
\end{lstlisting}
\end{ejemplo}

\subsection{Constantes de posicionamiento}
\label{subsubsec:posicionamiento}
Para posicionar el puntero de lectura/escritura en un offset determinado,
existe la función \lstinline{fseek()}. El origen del desplazamiento se expresa 
con las constantes \lstinline{SEEK_SET}, \lstinline{SEEK_END} y \lstinline{SEEK_CUR}.

La constante \lstinline{SEEK_SET} indica que el posicionamiento solicitado debe entenderse
como absoluto a partir del principio del archivo. La constante \lstinline{SEEK_CUR}
indica posicionamiento a partir del offset actual, y \lstinline{SEEK_END}, a partir del fin del
archivo. El offset proporcionado puede ser negativo.

\subsection{Sincronización de E/S}
Una restricción importante de la E/S en ANSI C es que no se pueden mezclar
instrucciones de entrada y de salida sin que intervenga una operación
intermedia de posicionamiento. Es decir, una sucesión de invocaciones a \lstinline{fwrite()} 
puede ser seguida de uno o más \lstinline{fread()}, pero únicamente luego de un \lstinline{fseek()}
entre ambas. La operación de posicionamiento resincroniza ambos punteros y su
ausencia puede hacer que se recupere basura.

Este posicionamiento puede ser nulo, como por ejemplo en \lstinline{fseek(fp, 0L, SEEK_CUR)}
que no varía en absoluto la posición de los punteros, pero realiza la
sincronización buscada.

\section{Resumen de funciones ANSI C de E/S}
Podemos resumir lo visto hasta aquí con los prototipos de las funciones ANSI de
E/S en el Cuadro \ref{tab:resumenesansicarchivos}.

\begin{table}[hbtp]
\centering
\begin{tabular}{l|l|l}
	\hline
			& E/S Standard			& E/S sobre archivos \\
    \hline
    \multirow{2}{*}{De caracteres}  & 
\begin{codecell}
int getchar();	
\end{codecell}
& 
\begin{codecell}
int fgetc(FILE *stream); 
\end{codecell}
\\
    								&
\begin{codecell}
int putchar(int c) 
\end{codecell}
& 
\begin{codecell}
int fputc(int c, FILE *stream); 
\end{codecell}
\\
    \hline
    \multirow{2}{*}{De líneas}  & 
\begin{codecell}
char *gets(char *s);
\end{codecell}
	& 
\begin{codecell}
char *fgets(char *s, 
int n, FILE *stream);
\end{codecell}
\\
    								&
\begin{codecell}
int puts(const char *s);
\end{codecell}
  & 
\begin{codecell}
int fputs(const char *s, 
FILE *stream);
\end{codecell}
 \\
    
    \hline	
\multirow{2}{*}{Con formato}  	& 
\begin{codecell}
int printf(const 
char *format, ...);
\end{codecell}
	& 
\begin{codecell}
int fprintf(FILE *stream, 
const char *format, ...);
\end{codecell}
\\
    								&  
\begin{codecell}
int scanf(const 
char *format, ...);
\end{codecell}
   & 
\begin{codecell}
int fscanf(FILE *stream, 
const char *format, ...);
\end{codecell}
  \\
    \hline
\multirow{2}{*}{Sobre strings}  	& 
\begin{codecell}
int sprintf(char *s, const 
char *format, ...);
\end{codecell}
	&  \\
    								&   
\begin{codecell}
int sscanf(char *s, const 
char *format, ...);
\end{codecell}
   &    \\
    \hline
\multirow{2}{*}{De acceso directo}  	& 	& 
\begin{codecell}
size_t fread(void *ptr, size_t 
size, size_t nobj, FILE *stream);
\end{codecell}
\\
    								&      &  
\begin{codecell}
size_t fwrite(const void *ptr, 
size_t size, size_t nobj, 
FILE *stream);
\end{codecell}
\\
\end{tabular}
\caption{Funciones de E/S ANSI C para archivos.}
\label{tab:resumenesansicarchivos}
\end{table}

%  _____________________________________________________________________________
% |_____________|E/S_Standard___________________|E/S_sobre_archivos_____________|
% |De caracteres|int getchar();                 |int fgetc(FILE *stream);       |
% |_____________|int_putchar(int_c);____________|int_fputc(int_c,_FILE_*stream);|
% |             |                               |char *fgets(char *s, int n,    |
% |De líneas    |char *gets(char *s);           |FILE *stream);                 |
% |             |int puts(const char *s);       |int fputs(const char *s, FILE  |
% |_____________|_______________________________|*stream);______________________|
% |             |int printf(const char *format, |int fprintf(FILE *stream, const|
% |Con formato  |...);                          |char *format, ...);            |
% |             |int scanf(const char *format,  |int fscanf(FILE *stream, const |
% |_____________|...);__________________________|char_*format,_...);____________|
% |             |int sprintf(char *s, const char|                               |
% |Sobre strings|*format, ...);                 |                               |
% |             |int sscanf(char *s, const char |                               |
% |_____________|*format,_...);_________________|_______________________________|
% |             |                               |size_t fread(void *ptr, size_t |
% |             |                               |size, size_t nobj, FILE        |
% |De acceso    |                               |*stream);                      |
% |directo      |                               |size_t fwrite(const void *ptr, |
% |             |                               |size_t size, size_t nobj, FILE |
% |_____________|_______________________________|*stream);______________________|
% 

\section{Funciones POSIX de E/S sobre archivos}
\label{subsec:esarchivosposix}

Las funciones POSIX son la interfaz directa del programa con las llamadas al
sistema, o \textbf{system calls}. Los programas que utilicen funciones POSIX deben incluir los headers de
biblioteca standard \lstinline{unistd.h} y \lstinline{fcntl.h}.
 

\subsection{Apertura de archivos}
Las funciones POSIX que operan sobre archivos lo hacen
a través de descriptores de archivos. Estos pertenecen a una tabla de archivos
abiertos que tiene cada proceso o programa en ejecución y se corresponden con
estructuras de datos del sistema operativo para manejar la escritura y la
lectura en los archivos propiamente dichos.

La tabla de archivos abiertos de un proceso se inicia con tres archivos
abiertos (correspondientes a los streams de \textbf{entrada standard}, \textbf{salida standard} y
\textbf{salida standard de errores}), que reciben los descriptores números 0, 1 y 2
respectivamente. Estos archivos son automáticamente provistos por el ambiente donde se ejecuta el 
proceso, y al comenzar a ejecutarse los encuentra ya abiertos.

Además de los tres descriptores standard abiertos por defecto, los programas abren otros archivos 
usando la función \lstinline{open()}. 
Los archivos que se abran subsiguientemente irán ocupando los
siguientes descriptores de archivo libres en esta tabla, y por lo tanto reciben
números de descriptor de 3 en adelante. Cada nueva operación de apertura exitosa de un
archivo devuelve un nuevo descriptor. Este número de descriptor se utiliza
durante el resto de la actividad sobre el archivo.

Tanto las funciones ANSI C para archivos como las funciones POSIX de archivos
manejan referencias obtenidas mediante la apertura y utilizadas durante toda la
relación del programa con el archivo, pero las referencias son diferentes. La
referencia al bloque de control utilizada por las funciones ANSI es de tipo
\lstinline{FILE *}, mientras que el descriptor de archivo POSIX es un int. Por este motivo
no se pueden mezclar las llamadas a funciones de uno y otro grupo.

Sin embargo, sí es cierto que las estructuras de tipo \lstinline{FILE}, referenciadas por
un \lstinline{FILE *}, como se dijo antes, se apoyan en funcionalidad aportada por
funciones POSIX, y por lo tanto contienen un descriptor de archivo. Si se tiene
un archivo abierto mediante una función POSIX, es posible, dado su descriptor,
obtener directamente el stream bufferizado correspondiente para manipularlo con
funciones ANSI. Esto se logra con la función \lstinline{fdopen()}.



\subsubsection{Flags}
Las funciones de E/S POSIX permiten controlar varios aspectos de la entrada/salida sobre dispositivos o archivos. Existe un conjunto de \textit{flags}, u opciones, que modifican la conducta de cada operación.   
Los flags se expresan como nombres simbólicos para constantes de bits, y se especifican en la apertura del archivo como segundo argumento de la función \lstinline{open()}. Todas las combinaciones válidas de flags pueden expresarse como un \lstinline{OR} de bits (operador \lstinline{|}) que acumula las constantes de bits correspondientes. Resumimos los flags más importantes existentes para este segundo argumento de \lstinline{open()} en el Cuadro \ref{tab:flags}.


 \begin{table} 
 \centering 
 \small
 \begin{tabular}{l|l}
\begin{codecell}
O_RDONLY
\end{codecell} & El archivo se abre para lectura solamente\\
\begin{codecell}
O_RDWR
\end{codecell}   & Se abre para escritura solamente\\
\begin{codecell}
O_APPEND
\end{codecell} & El archivo puede ser leído o agregársele contenido\\
\begin{codecell}
O_CREAT
\end{codecell}  & Si el archivo no existe, se lo crea\\
\begin{codecell}
O_EXCL
\end{codecell}   & Si ya existe, se vuelve con indicación de error\\
\begin{codecell}
O_WRONLY
\end{codecell} & Se abre para escritura solamente\\
\begin{codecell}
O_TRUNC
\end{codecell}  & Si existe, se destruye antes de crearlo\\

 \end{tabular}
 \caption{Flags de apertura de archivos en funciones POSIX.}
 \label{tab:flags} 
 \end{table} 
 
\begin{ejemplo}
Este programa copia un archivo \lstinline{ejemplo.txt} sobre otro llamado \lstinline{copia.txt}, usando funciones POSIX \lstinline{read()} y \lstinline{write()} aplicadas a los descriptores obtenidos con \lstinline{open()}.
\begin{itemize}
	\item Abre el primer archivo en modo de sólo lectura con el \textit{flag}, u opción, \lstinline{O_RDONLY}.
	\item Como necesita escribir sobre el segundo, utiliza el flag
\lstinline{O_WRONLY}. 
	\item Además, para el segundo archivo, especifica otros flags que van
agregados al primero, y que son \lstinline{O_CREAT} (si no existe, se lo crea) y \lstinline{O_TRUNC} (si
ya existe, se borran todos sus contenidos).
\end{itemize}
\begin{lstlisting}
#include <unistd.h>
#include <fcntl.h>
main()
{
    char area[1024];
    int fd1, fd2, bytes;
    if((fd1 = open("ejemplo.txt", O_RDONLY)) < 0)
        exit(1);
    if((fd2 = open("copia.txt", O_WRONLY|O_CREAT|O_TRUNC, 0660)) < 0)
        exit(1);
    while(bytes = read(fd1, area, sizeof(area)))
        write(fd2, area, bytes);
    close(fd1);
    close(fd2);
}
\end{lstlisting}
\end{ejemplo}


\subsubsection{Permisos de los archivos}
El tercer argumento de \lstinline{open()} tiene sentido sólo al \textbf{crear} un archivo. Sirve
para especificar los permisos con los que será creado, siempre según el
concepto de UNIX de permisos de \textbf{lectura, escritura y ejecución}, distribuidos en
\textbf{clases de usuarios}. Para los sistemas operativos que no cuentan con estas
nociones, el tercer argumento simplemente se ignora, pero se mantiene en la
interfaz POSIX para asegurar la portabilidad de los programas. Nótese que las funciones ANSI C no permiten la especificación de estos permisos de creación.

Como para los flags, hay algunas constantes de bits útiles, que se detallan en el Cuadro \ref{tab:permisos}.

%TODO S_IREAD; S_IWRITE; S_IEXEC son obsoletas y por compatibilidad con BSD
%  ____________________________________________________
% |_________________|Dueño____________|Grupo__|Otros__|
% |Lectura__________|S_IRUSR_(S_IREAD)_|S_IRGRP|S_IROTH|
% |Escritura________|S_IWUSR_(S_IWRITE)|S_IWGRP|S_IWOTH|
% |Ejecución_______|S_IXUSR_(S_IEXEC)_|S_IXGRP|S_IXOTH|
% |Los_tres_permisos|S_IRWXU___________|S_IRWXG|S_IRWXO|
% 


\begin{table}
\centering
\begin{tabular}{l|l|l|l}
	& Dueño & Grupo & Otros \\
\hline
Lectura & 
\begin{codecell}
S_IRUSR
\end{codecell}
& 
\begin{codecell}
S_IRGRP
\end{codecell}
 & 
\begin{codecell}
S_IROTH
\end{codecell}
 \\
Escritura & 
\begin{codecell}
S_IWUSR
\end{codecell}
& 
\begin{codecell}
S_IWGRP
\end{codecell}
 & 
\begin{codecell}
S_IWOTH
\end{codecell}
 \\
Ejecución & 
\begin{codecell}
S_IXUSR
\end{codecell}
 & 
\begin{codecell}
S_IXGRP
\end{codecell}
 & 
\begin{codecell}
S_IXOTH
\end{codecell}
 \\
Los tres permisos & 
\begin{codecell}
S_IRWXU
\end{codecell}
 & 
\begin{codecell}
S_IRWXG 
\end{codecell}
& 
\begin{codecell}
S_IRWXO
\end{codecell}
\\
\end{tabular}
 \caption{Flags de permisos de archivos en funciones POSIX.}
 \label{tab:permisos} 
 \end{table} 



\begin{ejemplo}
Si el archivo \lstinline{prueba.dat} no existe, se lo crea; se abre para lectura y
escritura y con todos los permisos para el creador, pero sólo con permiso de
lectura para el grupo del dueño. El resto de los usuarios no tiene ningún
permiso sobre el archivo.
\begin{lstlisting}
int fd = open("prueba.dat", O_RDWR|O_CREAT, S_IRWXU|S_IRGRP);	
\end{lstlisting}
\end{ejemplo}


\subsection{Posicionamiento en archivos}
Para posicionar el puntero de lectura/escritura en un offset determinado,
existe la función \lstinline{lseek()}. El origen del desplazamiento se expresa, como en las
funciones ANSI C de acceso directo, con las constantes \lstinline{SEEK_SET}, \lstinline{SEEK_END} y
\lstinline{SEEK_CUR} (ver \ref{subsubsec:posicionamiento}).

\begin{ejemplo}
Repetimos el ejemplo dado para las funciones ANSI donde se lee el registro
lógico número 10 y tras una modificación se lo copia en el registro lógico 5,
esta vez con funciones POSIX.
\begin{lstlisting}
lseek(fd, 10L * sizeof(struct registro), SEEK_SET);
read(fd, &datos, sizeof(struct registro));
datos.dato1 = 1;
lseek(fd, 5L * sizeof(struct registro), SEEK_SET);
write(fd, &datos, sizeof(struct registro));
\end{lstlisting}
\end{ejemplo}



\section{Ejercicios}
\begin{enumerate}
\item Escribir una función que copie la entrada en la salida pero eliminando las
vocales.
\item Escribir una función que reemplace los caracteres no imprimibles por
caracteres punto.
\item Construir un programa que imprima su entrada invertida, carácter por carácter.
\item Construir un programa que imprima su entrada invertida, línea por línea.
\item Construir un programa que cuente la cantidad de palabras de un archivo,
separadas por blancos, tabuladores o fin de línea.
\item Construir un programa que cuente la cantidad de caracteres y de líneas de un
archivo.
\item Construir un programa que permita eliminar de un archivo las líneas que
contengan una cadena dada.
\item Escribir una función que reciba como argumento dos enteros y devuelva un
string de formato conteniendo una máscara de formato apropiada para imprimir un
número en punto flotante. Por ejemplo, si se le dan como argumentos 7 y 2,
deberá devolver el string \lstinline{"%7.2f"}. Aplicar la función para imprimir números en
punto flotante.
\item Escribir sobre un archivo una variable int con valor 1 y una variable long
con valor 2. Hacerlo primero con funciones de E/S con formato, y luego con
funciones de acceso directo. Examinar en cada caso el resultado visualizando el
archivo y opcionalmente con un comando como \lstinline{od -bc}.
\item Defina una estructura básica simple para un registro, a su gusto. Puede ser
un registro de información personal, bibliográfica, etc. Construya funciones
para leer estos datos del teclado e imprimirlos en pantalla. Luego, usando
funciones ANSI C, construya funciones para leer y escribir una de estas
estructuras en un archivo, dado un número de registro lógico determinado.
\item Repita el ejercicio anterior reemplazando las funciones ANSI C por funciones
POSIX.
\item Construya programas que utilicen las funciones anteriores, ANSI C o POSIX,
y ofrezcan un menú de operaciones de administración: cargar un dato en el
archivo, imprimir los datos contenidos en una posición determinada, listar la
base generada completa, eliminar un registro, etc.
\end{enumerate}


%TODO Ejercicios_Adicionales Ejercicios_Avanzados

