



\chapter{La Biblioteca Standard}
La Biblioteca Standard no forma parte del C, estrictamente hablando, pero todos
los compiladores contienen una implementación, a veces con agregados o pequeñas
variantes. Desde la oficialización del ANSI C, los contenidos de la Biblioteca
Standard se han estabilizado y se puede contar con el mismo conjunto de
funciones en todas las plataformas.

Las funciones de la biblioteca standard están agrupadas en varias categorías.
Para poder utilizar cualquiera de las funciones de cada categoría, es necesario
incluir en el fuente el header asociado con la categoría. Esto no
implica incluir los textos de las funciones en la unidad de traducción, sino
simplemente incorporar los prototipos de las funciones de la biblioteca. Es
decir, incluir un header de Biblioteca Standard no \textbf{define} las funciones que se
van a usar, sino que las \textbf{declara}. La resolución de las referencias a las
funciones o variables de Biblioteca Standard quedan pendientes hasta la
linkedición.

Las categorías y los headers de la biblioteca standard son los siguientes. Una
implementación de C puede aportar muchísimos otros headers más específicos. Con
esta información no pretendemos reemplazar al manual del compilador sino
orientar a los primeros pasos en el uso de la Biblioteca Standard.

\tabla{catbs}{Categorías de funciones de la Biblioteca Standard.}{2}{

Categoría & header \\
\hrule
Macros de diagnóstico de errores	&	<assert.h> \\
Macros de clasificación de caracteres	&	<ctype.h> \\
Variables y funciones relacionadas con condiciones de error & <errno.h>\\
Características de la representación en punto flotante	& <float.h>\\
Rangos de tipos de datos, dependientes de la plataforma & <limits.h>\\
Definiciones relacionadas con el idioma y lugar de uso & <locale.h>\\
Funciones matemáticas & <math.h>\\
Saltos no locales & <setjmp.h>\\
Manejo de señales & <signal.h>\\
Listas de argumentos variables & <stdarg.h>\\
Definiciones de algunos tipos y constantes comunes & <stddef.h>\\
Entrada/salida & <stdio.h>\\
Varias funciones útiles & <stdlib.h>\\
Operaciones sobre cadenas & <string.h>\\
Funciones de fecha y hora & <time.h>\\
}

Ya hemos visitado la mayoría de las funciones de la categoría de funciones de E/S. Nos
quedan por explorar algunas otras de importancia.

\section{Funciones de strings (<stdio.h>)}
No existiendo un tipo de datos string en C, se lo implementa como un arreglo de
caracteres, dado por su dirección inicial, y terminado en el carácter especial
'\0'. Todas las funciones de strings de BS hacen uso de este protocolo de fin
de string. Muchas de ellas han sido implementada en las prácticas de capítulos
anteriores.

Aquí las más importantes. Consultar también manual de las funciones \code{strspn()},
\code{strcspn()}, \code{strpbrk()}, \code{strstr()}, \code{strtok()}.
 ____________________________________________________________________________________________________
|prototipo|significado_______________________________________________________________|ejemplo________|
|char     |Copia la cadena ct sobre s, incluyendo el '\0'. Devuelve s.               |char alfa[10]; |
|*strcpy  |                                                                          |strcpy(alfa,   |
|(s,_ct)__|__________________________________________________________________________|"cadena");_____|
|char     |Copia ct sobre s hasta n caracteres. Rellena con '\0' hasta el final si ct|char alfa[10]; |
|*strncpy |tiene menos de n caracteres.                                              |strncpy(alfa,  |
|(s, ct,  |                                                                          |"cadena", 4);  |
|n)       |                                                                          |strncpy(alfa,  |
|         |                                                                          |otro, sizeof   |
|_________|__________________________________________________________________________|(alfa));_______|
|char     |Concatena la cadena ct al final de s. Debe garantizarse espacio para      |char alfa[10] =|
|*strcat  |realizar la operación.                                                   |"abc";         |
|(s, ct)  |                                                                          |strcat(alfa,   |
|_________|__________________________________________________________________________|"def");________|
|char     |Idem hasta n caracteres.                                                  |               |
|*strncat |                                                                          |               |
|(s,_ct)__|__________________________________________________________________________|_______________|
|int      |Compara las cadenas. Devuelve <0 si cs<ct, 0 si cs==ct, >0 si cs>ct.      |if(strcmp      |
|strcmp   |                                                                          |(alfa,"abcdef")|
|(cs, ct) |                                                                          |== 0)          |
|         |                                                                          |    printf     |
|_________|__________________________________________________________________________|("iguales\n");_|
|int      |Idem hasta n caracteres.                                                  |               |
|strncmp  |                                                                          |               |
|(cs,_ct)_|__________________________________________________________________________|_______________|
|char     |Devuelve un apuntador a la primera ocurrencia del carácter c en la cadena|char *p;       |
|*strchr  |cs, o bien NULL si no se lo halla.                                        |char r[] =     |
|(cs, c)  |                                                                          |"casualidad";  |
|         |                                                                          |p = strchr(r,  |
|         |                                                                          |'s');          |
|         |                                                                          |strncpy(p,     |
|_________|__________________________________________________________________________|"us",_2);______|
|char     |Idem la última ocurrencia.                                               |               |
|*strrchr |                                                                          |               |
|(cs,_c)__|__________________________________________________________________________|_______________|
|size_t   |Devuelve la longitud de la cadena, sin contar el '\0' final.              |for(i=0; i <   |
|strlen   |                                                                          |strlen(s); i++)|
|(cs)     |                                                                          |    printf     |
|_________|__________________________________________________________________________|("%c\n",_s[i]);|

Listas de argumentos variables (<stdarg.h>)
Es posible definir funciones que reciban una cantidad arbitraria de parámetros
reales. Para esto se prepara un encabezado de la función con los parámetros
reales fijos que se desee y se indican los restantes, variables, mediante
puntos suspensivos. Se recuperan los demás con macros especiales definidas en
este header.
Lamentablemente estas macros no permiten la creación de funciones sin
argumentos fijos. Existe otro paquete de argumentos variables, definido en
varargs.h, que sí lo permite; pero que no está comprendido en el estándar ANSI
C y que no es compatible con stdarg.h.
**** Ejemplo ****
    #include <stdarg.h>
    int sumar(int cuantos, ...)
    {
        va_list ap;
        int suma=0;

        va_start(ap, cuantos);
        for(i=0; i<cuantos; i++)
            suma += va_arg(ap, int);
        va_end(ap);
        return suma;
   }
Que se utilizaría como:
    main()
    {
        printf("Resultado 1: %d\n", sumar(3, 4, 5, 6));
        printf("Resultado 2: %d\n", sumar(2, 100, 2336));
    }

Funciones de tratamiento de errores (<errno.h> y <assert.h>)
Esta zona de la Biblioteca provee indispensables herramientas de debugging. La
variable externa errnoes un entero que toma un valor de acuerdo a condiciones
de error provocadas por cualquiera de las funciones de la BS, y de acuerdo a
una catalogación de errores que depende de la función. Si una función ANSI C
devuelve un valor indicador de error (como NULL donde debería devolver un
puntero, o negativo donde debería devolver un positivo), la variable errno
contendrá más explicación sobre el motivo del error. Se consulta con las
funciones strerror()o perror(). La función perror() admite una cadena
arbitraria para indicar, por ejemplo, el lugar del programa donde se produce el
error. Imprimirá esta cadena más la descripción del problema.
La macro assertsirve para detener la ejecución cuando se alcanza un estado
imposible para la lógica del programa. Para usarla adecuadamente es necesario
identificar invariantes en el programa (condiciones que no deban jamás ser
falsas). Si al evaluarse la macro resulta que su condición argumento es falsa,
assert() aborta el programa indicando nombre del archivo fuente y línea donde
estaba originalmente la llamada. Un programa en producción no debería fallar
debido a assert.
**** Ejemplos ****
    #include <errno.h>
    if(open("noexiste", O_RDONLY) < 0)
        perror("Error en apertura");


    #include <assert.h>
    ...
    assert(restantes >= 0);

Funciones de fecha y hora (<time.h>)
Existen dos tipos definidos en el header time.h para manejar datos de fechas.
Por un lado, se tiene el tipo struct tm, que contiene los siguientes elementos
que describen un momento en el tiempo:
struct tm {
int tm_sec,    /* segundos 0..59 */
    tm_min,    /* minutos 0..59 */
    tm_hour,   /* horas 0..23 */
    tm_mday,   /* día del mes 1..31 */
    tm_mon,    /* meses desde enero 0..11 */
    tm_year,   /* años desde 1900 */
    tm_wday,   /* días desde el domingo 0..6 */
    tm_yday,   /* días desde enero 0..365 */
    tm_isdst;  /* flag de ahorro diurno de luz */
};
Por otro lado, existe un segundo formato de representación interna de fechas,
time_t, que es simplemente un entero conteniendo la cantidad de segundos desde
el principio de la era UNIX ("the epoch"), acaecido el 1/1/1970 a la hora 0
UTC. Este formato es el usado por la función time() que da la hora actual.
Este formato entero puede convertirse a struct tm y viceversa con funciones
definidas en esta zona de la BS, como mktime() y gmtime().
El contenido de una estructura tm se puede imprimir con gran variedad de
formatos con la función strftime(), que acepta una cantidad de especificaciones
al estilo de printf(). Las funciones ctime() y asctime()son más sencillas.
Devuelven una cadena conteniendo una fecha en formato normalizado (como el que
aparece en los mensajes de correo electrónico). La primera recibe un puntero a
time_t; la segunda, un puntero a struct tm.
**** Ejemplo ****
    time_t t;
    struct tm *stm;

    t = time(NULL);             /* recoge la hora actual */
    printf("%s\n",ctime(&amp;t));   /* imprime en formato standard */

    char area[100];
    stm = gmtime(&amp;t);           /* convierte t a struct tm */
    strftime(area,sizeof(area), /* prepara string segun formato del usuario */
        "%A %b %d %H",stm);
    printf("%s\n",area);        /* lo imprime */

Funciones matemáticas (<math.h>)
Aquí se encuentran las habituales funciones aritméticas avanzadas,
trigonométricas y logarítmicas.
 ____________________________________________________________________________________
|sin(x)_|seno_de_x__________________________________________________________|________|
|cos(x)_|coseno_de_x________________________________________________________|________|
|tan(x)_|tangente_de_x______________________________________________________|________|
|asin(x)|arco seno de x                                                     |Devuelve|
|       |                                                                   |valores |
|       |                                                                   |en el   |
|       |                                                                   |rango [-|
|       |                                                                   |pi/2,   |
|_______|___________________________________________________________________|pi/2]___|
|acos(x)|arco coseno de x                                                   |En el   |
|       |                                                                   |rango   |
|_______|___________________________________________________________________|[0,_pi]_|
|atan(x)|arco tangente de x                                                 |Devuelve|
|       |                                                                   |valores |
|       |                                                                   |en el   |
|       |                                                                   |rango [-|
|       |                                                                   |pi/2,   |
|_______|___________________________________________________________________|pi/2]___|
|atan2  |arco tangente de y/x                                               |En el   |
|(y,x)  |                                                                   |rango [-|
|_______|___________________________________________________________________|pi,_pi]_|
|sinh(x)|seno_hiperbólico_de_x_____________________________________________|________|
|cosh(x)|coseno_hiperbólico_de_x___________________________________________|________|
|tanh(x)|tangente_hiperbólica_de_x_________________________________________|________|
|exp(x)_|función_exponencial_de_base_e_____________________________________|________|
|log(x)_|logaritmo_natural__________________________________________________|________|
|log10  |logaritmo de base decimal                                          |        |
|(x)____|___________________________________________________________________|________|
|pow    |xy                                                                 |Para x=0|
|(x,y)  |                                                                   |debe ser|
|       |                                                                   |y>0. Si |
|       |                                                                   |x<0,    |
|       |                                                                   |debe ser|
|_______|___________________________________________________________________|y_entero|
|sqrt(x)|raíz cuadrada de x                                                |Debe ser|
|_______|___________________________________________________________________|x>=0____|
|ceil(x)|menor entero no menor que x                                        |Devuelve|
|       |                                                                   |un      |
|_______|___________________________________________________________________|double__|
|floor  |mayor entero no mayor que x                                        |Devuelve|
|(x)    |                                                                   |un      |
|_______|___________________________________________________________________|double__|
|fabs(x)|Valor_absoluto_de_x________________________________________________|________|
|ldexp  |Devuelve x * 2n                                                    |        |
|(x,n)__|___________________________________________________________________|________|
|frexp  |Divide a x en una potencia entera de 2, que se almacena en el lugar|        |
|(x,exp)|apuntado_por_exp,_y_devuelve_una_mantisa_en_el_intervalo_[½,_1].__|________|
|modf(x,|Divide a x en parte entera fraccionaria. El argumento ip debe ser  |        |
|ip)____|un_puntero_a_entero._______________________________________________|________|
|fmod   |Residuo de punto flotante de x/y.                                  |        |
|(x,y)__|___________________________________________________________________|________|


***** Funciones utilitarias (<stdlib.h>) *****
El header stdlib.h agrupa las declaraciones de varias funciones no relacionadas
entre sí, y que sirven a varios fines. Solamente las nombramos y encarecemos la
lectura del manual.
    * Funciones de conversión: las funciones atoi(), atol(), atof(), strtol(),
      strtod(), strtoul(), toman cadenas representando números y generan los
      elementos de datos del tipo correspondiente.
    * Se pueden generar números aleatorios con rand() y srand().
    * Aquí también se declaran las funciones de asignación de memoria como
      malloc(), calloc(), realloc(), free().
    * Para manejar datos en memoria con eficiencia se puede recurrir a qsort()y
      bsearch(), que ordenan una tabla y realizan búsqueda binaria en la tabla
      ordenada.
Clasificación de caracteres (<ctype.h>)
El header ctype.h contiene declaraciones de macros para averiguar la
pertenencia de un carácter a determinados conjuntos. Son todas booleanas salvo
las últimas que devuelven ints.
 ______________________________________________________________________________
|macro___|devuelve_TRUE_cuando_________________________________________________|
|isalnum |isalpha(c) o isdigit(c) son TRUE                                     |
|(c)_____|_____________________________________________________________________|
|isalpha |isupper(c) o islower(c) son TRUE                                     |
|(c)_____|_____________________________________________________________________|
|iscntrl |c es un carácter de control                                         |
|(c)_____|_____________________________________________________________________|
|isdigit |c es un dígito decimal                                              |
|(c)_____|_____________________________________________________________________|
|isgraph |c es un carácter imprimible y no isspace(c)                         |
|(c)_____|_____________________________________________________________________|
|islower |c es una letra minúscula                                            |
|(c)_____|_____________________________________________________________________|
|isprint |c es un carácter imprimible, incluyendo el caso en que isspace(c) es|
|(c)_____|TRUE_________________________________________________________________|
|ispunct |c es imprimible pero no espacio, letra ni dígito                    |
|(c)_____|_____________________________________________________________________|
|isspace |c es espacio, fin de línea, tabulador                               |
|(c)_____|_____________________________________________________________________|
|isupper |c es letra mayúscula                                                |
|(c)_____|_____________________________________________________________________|
|isxdigit|c es dígito hexadecimal                                             |
|(c)_____|_____________________________________________________________________|
|tolower |devuelve c en minúscula                                             |
|(c)_____|_____________________________________________________________________|
|toupper |devuelve c en mayúscula                                             |
|(c)_____|_____________________________________________________________________|


===============================================================================
Ejercicios
1. Utilizar la función de cantidad variable de argumentos definida más arriba
para obtener los promedios de los 2, 3, ..., n primeros elementos de un
arreglo.
2. Construir una función de lista variable de argumentos que efectúe la
concatenación de una cantidad arbitraria de cadenas en una zona de memoria
provista por la función que llama.
3. Construir una función de cantidad variable de argumentos que sirva para
imprimir, con un formato especificado, mensajes de debugging, conteniendo
nombres y valores de variables.
4. Construir un programa que separe la entrada standard en palabras, usando las
macros de clasificación de caracteres. Debe considerar como delimitadores a los
caracteres espacio, tabulador, signos de puntuación, etc.
5. Dadas dos fechas y horas del día, calcular su diferencia. Utilizar las
funciones de BS para convertir a tipos de datos convenientes e imprimir la
diferencia en años, meses, días, horas, etc.
6. Generar fechas al azar dentro de un período de tiempo dado.

===============================================================================





