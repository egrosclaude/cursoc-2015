
\documentclass[11pt,a4paper]{book}
\usepackage[utf8x]{inputenc}
\usepackage[T1]{fontenc}
\usepackage[spanish]{babel}
\usepackage{amsmath}
\usepackage{amssymb,amsfonts,textcomp}
\usepackage{color}
\usepackage{array}
\usepackage{multirow}
\usepackage{hhline}
\usepackage{hyperref}
\usepackage{float}
\usepackage{xkeyval}
\usepackage[pdftex]{graphicx}
\usepackage[yyyymmdd,hhmmss]{datetime}
\usepackage[usenames,dvipsnames]{xcolor}
\usepackage{appendix}
\usepackage{listings}
\usepackage{enumitem}
\usepackage{fancyvrb}


\usepackage{amsthm} %load after amsmath

\newtheoremstyle{ejemplo}% hname
{11pt}% Space above
{3pt}% Space below
{\footnotesize}% Body font
{}% Indent amount
{\bfseries}% Theorem head font
{}% Punctuation after theorem head
{\newline}% Space after theorem head
{}% Theorem head spec (can be left empty, meaning ‘normal’)

\theoremstyle{ejemplo}
\newtheorem{ejemplo}{Ejemplo}[section]


\usepackage{multicol}


%\usepackage{wasysym}
\definecolor{dkgreen}{rgb}{0,0.6,0}
\definecolor{gray}{rgb}{0.5,0.5,0.5}
\definecolor{mauve}{rgb}{0.58,0,0.82}
\definecolor{lstbackground}{rgb}{0.95,0.95,0.95}
\definecolor{white}{rgb}{100,100,100}
% \lstset{frame=tb,
% 	backgroundcolor=\color{lstbackground},
%   language=Bash,
%   aboveskip=3mm,
%   belowskip=3mm,
%   showstringspaces=false,
%   columns=flexible,
%   basicstyle={\small\ttfamily},
%   numbers=none,
%   numberstyle=\tiny\color{gray},
%   keywordstyle=\color{blue},
%   %commentstyle=\color{dkgreen},
%   stringstyle=\color{mauve},
%   breaklines=true,
%   breakatwhitespace=true,
%   extendedchars=true,
%   tabsize=3
% }
\lstset{
  backgroundcolor=\color{lstbackground},
% language=Bash,
  aboveskip=3mm,
  belowskip=3mm,
  showstringspaces=false,
  columns=flexible,
  basicstyle={\small\ttfamily\bfseries},
  numbers=none,
%  xleftmargin=0.5cm,
%  xrightmargin=0.5cm,
%  frame=lr,framesep=0.5cm,framerule=0pt,
%  numberstyle=\tiny\color{gray},
%  keywordstyle=\color{blue},
%  commentstyle=\color{dkgreen},
%  stringstyle=\color{mauve},
  breaklines=true,
  breakatwhitespace=true,
  tabsize=4
}

% \lstnewenvironment{codecell}
%   	{\lstset{
% 	  aboveskip=3mm,
% 	  belowskip=3mm,
% 	  showstringspaces=false,
% 	  columns=flexible,
% 	  basicstyle={\small\ttfamily},
% 	  numbers=none,
% 	  breaklines=true,
% 	  breakatwhitespace=true,
% 	  tabsize=4
% 	  }
% 	} {}
% 

\lstnewenvironment{mylistings}
  {\lstset{language=C++,
    backgroundcolor=\color{listingscolor}, % set backgroundcolor
    basicstyle=\footnotesize,% basic font setting
    }%
  }
  {}

\lstnewenvironment{codecell}
  {\lstset{basicstyle={\small\ttfamily},% basic font setting
    backgroundcolor=\color{white}}%
  }
  {}



\usepackage[font=small,skip=1cm]{caption}

%\DeclareCaptionFont{black}{ \color{black} }
%\DeclareCaptionFormat{listing}{
%  \colorbox[cmyk]{0.93, 0.95, 0.95,0.01 }{
%    \parbox{\textwidth}{\hspace{15pt}#1#2#3}
%  }
%}
%\captionsetup[lstlisting]{ format=listing, labelfont=black, textfont=black, singlelinecheck=false, margin=0pt, font={bf,footnotesize} }
%\captionsetup[lstlisting]{format=plain, font={footnotesize}}
%\captionsetup[figure]{format=plain,font={footnotesize}}
% ...


%\renewcommand{\lstlistingname}{Code}
\usepackage{verbatim}

%\addto\captionsspanish {%
%	\def\appendixname{Apéndices}
%}
% Outline numbering
\setcounter{secnumdepth}{1}
% Reset section numbering between parts
\makeatletter
\@addtoreset{section}{chapter}
\makeatother  
% List styles
\newcommand\liststyleLi{%
\renewcommand\labelitemi{\tiny${\blacksquare}$}
\renewcommand\labelitemii{\tiny${\square}$}
\renewcommand\labelitemiii{\tiny${\circ}$}
\renewcommand\labelitemiv{\tiny${\circ}$}
}
\newcommand\liststyleLii{%
\renewcommand\labelitemi{{\textbullet}}
\renewcommand\labelitemii{${\circ}$}
\renewcommand\labelitemiii{${\blacksquare}$}
\renewcommand\labelitemiv{{\textbullet}}
}
\newcommand\liststyleLiii{%
\renewcommand\labelitemi{{\textbullet}}
\renewcommand\labelitemii{${\circ}$}
\renewcommand\labelitemiii{${\blacksquare}$}
\renewcommand\labelitemiv{{\textbullet}}
}

\liststyleLi

% Page layout (geometry)
\setlength\voffset{-1in}
\setlength\hoffset{-1in}
\setlength\topmargin{2cm}
\setlength\oddsidemargin{2cm}
\setlength\textheight{23.246668cm}
\setlength\textwidth{17.006cm}
\setlength\footskip{26.144882pt}
\setlength\headheight{1.016cm}
\setlength\headsep{0.508cm}
% Footnote rule
\setlength{\skip\footins}{0.119cm}
\renewcommand\footnoterule{\vspace*{-0.018cm}\setlength\leftskip{0pt}\setlength\rightskip{0pt plus 1fil}\noindent\textcolor{black}{\rule{0.25\columnwidth}{0.018cm}}\vspace*{0.101cm}}
% Pages styles
\makeatletter
\newcommand\ps@Standard{
  \renewcommand\@oddhead{{\raggedleft Cabecera \ } {\raggedright \thepage{}}}
  \renewcommand\@evenhead{\@oddhead}
  \renewcommand\@oddfoot{}
  \renewcommand\@evenfoot{\@oddfoot}
  \renewcommand\thepage{\arabic{page}}
}

% \pagestyle{Standard}
\usepackage{fancyhdr}
\pagestyle{fancy}


%%--------------------------------------
% F O N T S 
% \usepackage{dejavu}
%\usepackage{librebaskerville}
%\usepackage{sans}
%\usepackage{libertine}
%\usepackage{lmodern}
%\usepackage{opensans}
%\usepackage{helvet}
%\usepackage{times}

\usepackage{lmodern}
\renewcommand*\familydefault{\sfdefault} %% Only if the base font of the document is to be sans serif


%% LaTeX Preamble - Font choices
%% Each block selects new math, roman (serif), sans serif, and typewriter fonts.
%% Delete or comment out all but one to make your choice.

% Fourier for math | Utopia (scaled) for rm | Helvetica for ss | Latin Modern for tt
%\usepackage{fourier} % math & rm
%\usepackage[scaled=0.875]{helvet} % ss
%\renewcommand{\ttdefault}{lmtt} %tt

% Latin Modern (similar to CM with more characters)
%\usepackage{lmodern} % math, rm, ss, tt
%\usepackage[T1]{fontenc}

% Palatino for rm and math | Helvetica for ss | Courier for tt
%\usepackage{mathpazo} % math & rm
%\linespread{1.05}        % Palatino needs more leading (space between lines)
%\usepackage[scaled]{helvet} % ss
%\usepackage{courier} % tt
%\normalfont
%\usepackage[T1]{fontenc}

% Euler for math | Palatino for rm | Helvetica for ss | Courier for tt
%\renewcommand{\rmdefault}{ppl} % rm
%\linespread{1.05}        % Palatino needs more leading
%\usepackage[scaled]{helvet} % ss
%\usepackage{courier} % tt
%\usepackage{euler} % math
%\usepackage{eulervm} % a better implementation of the euler package (not in gwTeX)

%\normalfont
%\usepackage[T1]{fontenc}

% Times for rm and math | Helvetica for ss | Courier for tt
%\usepackage{mathptmx} % rm & math
%\usepackage[scaled=0.90]{helvet} % ss
%\usepackage{courier} % tt
%\normalfont
%\usepackage[T1]{fontenc}

% !! COMMERICAL FONT !! Lucida Bright (w/expert package)
%\usepackage[T1]{fontenc}
%\usepackage[expert,vargreek,altbullet]{lucidabr}

%% END Font choices
%%---------------------------------------------
% \renewcommand*\familydefault{\sfdefault}
% \pagestyle{Standard}
\usepackage{mdframed}


% footnotes configuration
\makeatletter
\renewcommand\thefootnote{\arabic{footnote}}
\makeatother
\usepackage{graphicx}

\usepackage{xkeyval}
\usepackage{pifont}
\usepackage{xcolor}
\newcommand{\revisar}[1]{{\color{red}[#1]}}
%\newcommand{\nota}[1]{{\color{red}[#1]}}
%\newcommand{\revisar}[1]{}

\newcommand{\borrador}{
\revisar{\today, \currenttime  -  Material en preparación, se ruega no imprimir mientras aparezca esta nota}
}




\newcommand{\nota}[1]{}

\newcommand{\nonota}[1]{#1}

\newcommand{\quotes}[1]{``#1''}

   
\newcommand{\shade}[1]{\textcolor{black!50}{#1}}

% ancho opcional, por defecto 15cm
% \figura{copyleft}{Símbolo de Copyleft}{copyleft.png}
% \figura[6]{copyleft}{Símbolo de Copyleft}{copyleft.png}
\newcommand{\figura}[4][15]{
 \begin{figure}[htb] 
 \centering 
 \includegraphics[width=#1cm]{./img/#4}
 \caption{#3}
 \label{fig:#2} 
 \end{figure} 
}



% tabla{label}{caption}{columns}{contents}
\newcommand{\tabla}[4]{
 \begin{table} 
 \centering 
 \small
 \begin{tabular}{#3}
 #4
 \end{tabular}
 \caption{#2}
 \label{tab:#1} 
 \end{table} 
}

\usepackage{sectsty}
%\usepackage[compact]{titlesec} 
\definecolor{bl}{rgb}{0.0,0.2,0.6} 
\allsectionsfont{\color{bl}\scshape\selectfont}


\newcommand{\recuadro}[1]{
\begin{minipage}[c]{0.84\textwidth}
\begin{mdframed}
#1
\end{mdframed}
\end{minipage}
}

\newcommand{\code}[1]{\lstinline$#1$}

%\usepackage{C}
%\clearpage\clearpage\setcounter{page}{1}\pagestyle{HTML}

\section{Tipos de datos y expresiones}

Las expresiones se construyen, en general, conectando mediante operadores elementos diversos, tales
como identificadores de variables, constantes e invocaciones de funciones. Cada uno de estos
elementos debe ocupar -al menos temporariamente, mientras se calcula el resultado de la expresión- un
lugar en memoria. Al evaluar cada expresión, el compilador crea, para alojar cada subexpresión de las
que la constituyen, objetos de datos, que pueden pensarse como espacio de memoria reservado para
contener valores. Estos espacios de memoria son de diferentes "tamaños" (cantidades de bits) de
acuerdo al tipo de dato de la subexpresión.

Así, las expresiones en C asumen siempre un tipo de datos: alguno de los tipos básicos del lenguaje, o
uno definido por el usuario. Una expresión, según las necesidades, puede convertirse de un tipo a otro.
El compilador hace esto a veces en forma automática. Otras veces, el programador fuerza una
conversión de tipo para producir un determinado resultado.

\section{Declaración de variables}

Los tipos básicos son:
\begin{itemize}
	\item \texttt{char} (un elemento de tamaño "byte")
	\item \texttt{int} (un número entero con signo)
	\item \texttt{long} (un entero largo)
	\item \texttt{float} (un número en punto flotante)
	\item \texttt{double} (un número en punto flotante, doble precisión)
\end{itemize}

Cuando declaramos una variable o forzamos una conversión de tipo, utilizamos una especificación de
tipo. Ejemplos de declaración de variables:

\texttt{char a;
int alfa,beta;
float x1,x2;}

Los tipos enteros (char, int y long) admiten los modificadores signed (con signo) y unsigned (sin
signo). Un objeto de datos unsigned utiliza todos sus bits para representar magnitud; un signed utiliza
un bit para signo, en representación complemento a dos.

El modificador signed sirve sobre todo para explicitar el signo de los chars. El default para un int es
signed; el default para char puede ser signed o unsigned, dependiendo del compilador.

unsigned int edad;
signed char beta;

Un int puede afectarse con el modificador short (corto).
short i;
unsigned short k;

- 16 -

Lenguaje C

Tamaños de los objetos de datos

Cuando en una declaración aparece sólo el modificador unsigned o short, y no el tipo, se asume int. El
tipo entero se supone el tipo básico manejable por el procesador, y es el tipo por omisión en varias
otras situaciones. Por ejemplo, cuando no se especifica el tipo del valor devuelto por una función.
El modificador long puede aplicarse también a float y a double. Los tipos resultantes pueden tener más
precisión que los no modificados.

long float e; long double pi;

Tamaños de los objetos de datos

El lenguaje C no define el tamaño de los objetos de datos de un tipo determinado. Es decir, un entero
puede ocupar 16 bits en una implementación, 32 en otra, o aun 64. Un long puede tener o no más bits
que un int. Un short puede ser o no más corto que un int. Según K&R, lo único seguro es que "un
short no es mayor que un int, que a su vez no es mayor que long".

Por supuesto, distintos tamaños en bits implican diferentes rangos de valores. Si deseamos portar un
programa, hecho bajo una implementación del compilador, a otra, no es posible asegurar a priori el
rango que tomará un tipo de datos. La fuente ideal para conocer los rangos de los diferentes tipos, en
una implementación determinada, es -además del manual del compilador- el header limits.h de la
biblioteca standard. Debe recordarse que cualquier suposición que hagamos sobre el rango o tamaño
de un objeto de datos afecta la portabilidad de un programa en C.

Las siguientes líneas son parte de un archivo limits.h para una implementación en particular:

/* Minimum and maximum values a ‘signed short int’ can hold. */
# define SHRT_MIN
(-32768)
# define SHRT_MAX
32767
/* Maximum value an ‘unsigned short int’ can hold. (Minimum is 0.) */
# define USHRT_MAX
65535
/* Minimum and maximum values a ‘signed int’ can hold. */
# define INT_MIN
(-INT_MAX - 1)
# define INT_MAX
2147483647
/* Maximum value an ‘unsigned int’ can hold. (Minimum is 0.) */
# ifdef __STDC__
# define UINT_MAX
# 4294967295U
# else
# define UINT_MAX
 4294967295
endif
/* Minimum and maximum values a ‘signed long int’ can hold. */
# ifdef __alpha__
# define LONG_MAX
# 9223372036854775807L
# else
# define LONG_MAX
# 2147483647L
 endif
define LONG_MIN
(-LONG_MAX - 1L)
/* Maximum value an ‘unsigned long int’ can hold. (Minimum is 0.) */
# ifdef __alpha__
#
define ULONG_MAX
18446744073709551615UL
# else
#
ifdef __STDC__
#
define ULONG_MAX
4294967295UL
- 17 -
Operaciones con expresiones de distintos tipos
#
#
#
#
else
define ULONG_MAX
endif
endif
Eduardo Grosclaude 2001
4294967295L

Cuando una operación sobre una variable provoca overflow, no se obtiene ninguna indicación de error.
El valor sufre truncamiento a la cantidad de bits que pueda alojar la variable.

Así, en una implementación donde los ints son de 16 bits, si tenemos en una variable entera el máximo
valor positivo:

int a;
a=32767;
a=a+1;
/*
a=0111111111111111 binario */
Al calcular el nuevo valor de a, aparece un 1 en el bit más significativo, lo que lo transforma en un
negativo (el menor negativo que soporta el tipo de datos, -32768).
Si el int es sin signo:
unsigned a;
a=65535;
a=a+1;
/* maximo valor de unsigned int */

el nuevo valor de a se trunca a 16 bits, volviendo a 0.
Siempre se puede saber el tamaño en bits de un tipo de datos aplicando el operador sizeof() a una
variable o a la especificación de tipo.

Operaciones con expresiones de distintos tipos

En una expresión en C pueden aparecer componentes de diferentes tipos. Durante la evaluación de una
expresión donde sus subexpresiones sean de tipos diferentes, deberá tener lugar una conversión, ya sea
implícita o explícita, para llevar ambos operandos a un tipo de datos común con el que se pueda
operar. La forma en que el compilador resuelve las conversiones implícitas a veces provoca algunas
sorpresas.

Truncamiento en asignaciones

Para empezar, una asignación de una expresión de un tipo dado a una variable de un tipo menor no
sólo es permitida en C sino que la conversión se hace en forma automática y generalmente sin ningún
mensaje de tiempo de compilación ni de ejecución. Por ejemplo,
int a;
float b;
...
a=b;

En esta asignación tenemos miembros de diferentes tamaños. El resultado en a será el truncamiento
del valor entero de b a la cantidad de bits que permita un int. Es decir, se tomará la parte entera de b y
de esa expresión se copiarán en el objeto de datos de a tantos bits como quepan en un int, tomándose
los menos significativos.

- 18 -
Lenguaje C

Promoción automática de expresiones

Si el valor de b es, por ejemplo, 20.5, a terminará valiendo 20, lo que es similar a aplicar una función
"parte entera" implícitamente, y no demasiado incongruente. Pero si la parte entera de b excede el
rango de un entero (por ejemplo si b=99232.5 con enteros de 16 bits), el resultado en a no tendrá
lógica aparente. En el primer caso, los bits menos significativos de b que "caben" en a conservan el
valor de b; en el segundo caso, no.

En la sentencia:
a=19.27 * b;
a contendrá los sizeof(int) bits menos significativos del resultado de evaluar la expresión de la
derecha, truncada sin decimales.

Promoción automática de expresiones

Por otra parte, se tienen las reglas de promoción automática de expresiones. Enunciadas en forma
aproximada (más adelante las damos con precisión), estas reglas dicen que el compilador hará
estrictamente las conversiones necesarias para llevar todos los operandos al tipo del mayor. El
resultado de evaluar una operación aritmética será del tipo del mayor de sus operandos, en el sentido
del tamaño en bits de cada objeto de datos.

A veces esto no es lo deseado. Por ejemplo, dada la sentencia:
a=3/2;
se tiene que tanto la constante 3 como la constante 2 son vistas por el compilador como ints; el
resultado de la división será también un entero (la parte entera de 3/2, o sea 1). Aun más llamativo es
el hecho de que si declaramos previamente:
float a;
el resultado es casi el mismo: a terminará conteniendo el valor float 1.0, porque el problema de
truncamiento se produce ya en la evaluación del miembro derecho de la asignación.

Cómo recuperar el valor correcto, con decimales, de la división? Declarar a como float es necesario
pero no suficiente. Para que la expresión del miembro derecho sea float es necesario que al menos uno
de sus operandos sea float. Hay dos formas de lograr esto; la primera es escribir cualquiera de las
constantes como constante en punto flotante (con punto decimal, o en notación exponencial):
a=3./2;

Operador cast
La segunda consiste en forzar explícitamente una conversión de tipo, con un importante operador
llamado cast, así:
a=(float)3/2;
Da lo mismo aplicar el operador cast a cualquiera de las constantes. Sin embargo, lo siguiente no es
útil:
a=(float)(3/2);

- 19 -

Reglas de promoción en expresiones
Eduardo Grosclaude 2001

Aquí el operador cast se aplicará a la expresión ya evaluada como entero, con lo que volvemos a tener
un 1.0 float en a.

Reglas de promoción en expresiones

Son aplicadas por el compilador en el orden que se da más abajo (tomado de K&R, 2a. ed.). Esta es
una lista muy detallada de las comprobaciones y conversiones que tienen lugar; para la mayoría de los
propósitos prácticos basta tener en cuenta la regla de llevar ambos operandos al tipo del mayor de
ellos.

Entendemos por "promoción entera" el acto de llevar los objetos de tipo char, enum y campos de bits a
int; o, si los bits de un int no alcanzan a representarlo, a unsigned int.
1. Si cualquier operando es long double, se convierte el otro a long double
2. Si no, si cualquier operando es double, se convierte el otro a double
3. Si no, si cualquier operando es float, se convierte el otro a float
4. Si no, se realiza promoción entera sobre ambos operandos.
5. Si cualquiera de ellos es unsigned long int, se convierte el otro a unsigned long int.
6. Si un operando es long int y el otro es unsigned int, el efecto depende de si un long int puede
representar a todos los valores de un unsigned int.
7. Si es así, el unsigned int es convertido a long int.
8. Si no, ambos se convierten a unsigned long int.
9. Si no, si cualquier operando es long int, se convierte el otro a long int.
10. Si no, si cualquier operando es unsigned int, se convierte el otro a unsigned int.
11. Si no, ambos operandos son int.

Observaciones
Nótese que no existen tipos boolean ni string. Más adelante veremos cómo manejar datos de
estas clases.
El tipo char no está restringido a la representación de caracteres, como en Pascal. Por el contrario,
un char tiene entidad aritmética, almacena una cantidad numérica y puede intervenir en
operaciones matemáticas. En determinadas circunstancias, y sin perder estas propiedades, puede
ser interpretado como un carácter (el carácter cuyo código ASCII contiene).
En general, en C es conveniente habituarse a pensar en los datos separando la representación (la
forma como se almacena un objeto) de la presentación (la forma como se visualiza). Un mismo
patrón de bits almacenado en un objeto de datos puede ser visto como un número decimal, un
carácter, un número hexadecimal, octal, etc. La verdadera naturaleza del dato es la representación
de máquina, el patrón de bits almacenado.
- 20 -
Lenguaje C

Una herramienta: printf()
Una herramienta: printf()
Con el objeto de facilitar la práctica, describimos aquí la función de biblioteca standard printf().
La función de salida printf() lleva un número variable de argumentos.
Su primer argumento siempre es una cadena o constante string, la cadena de formato,
conteniendo texto que será impreso, más, opcionalmente, especificaciones de conversión.
Las especificaciones de conversión comienzan con un signo š%Š. Todo otro conjunto de
caracteres en la cadena de formato será impreso textualmente.
Cada especificación de conversión determina la manera en que se imprimirán los restantes
argumentos de la función.
Deben existir tantas especificaciones de conversión como argumentos luego de la cadena de
formato.

Un mismo argumento de un tipo dado puede ser impreso o presentado de diferentes maneras
según la especificación de conversión que le corresponda en la cadena de formato (de aquí la
importancia de separar representación de presentación)
Las especificaciones de conversión pueden estar afectadas por varios modificadores opcionales
que determinan, por ejemplo, el ancho del campo sobre el cual se escribirá el argumento, la
cantidad de decimales de un número, etc.

Las principales especificaciones de conversión son:
%d entero, decimal
%u entero sin signo, decimal
%l long, decimal
%c carácter
%s cadena
%f float
%lf double
%x entero hexadecimal
Ejemplos
Este programa escribe algunos valores con dos especificaciones distintas.
- 21 -


Ejercicios
Eduardo Grosclaude 2001
main()
{
int i,j;
for(i=65, j=1; i<70; i++, j++)
printf("vuelta no. %d: i=%d, i=%c\n",j,i,i);
}
Salida del programa:
vuelta
vuelta
vuelta
vuelta
vuelta
no.
no.
no.
no.
no.
1:
2:
3:
4:
5:
i=65,
i=66,
i=67,
i=68,
i=69,
i=A
i=B
i=C
i=D
i=E
El programa siguiente escribe el mismo valor en doble precisión pero con diferentes
modificadores del campo correspondiente.
main()
{
double d;
d=3.141519/2.71728182;
printf("d=%lf\n",d);
printf("d=%20lf\n",d);
printf("d=%20.10lf\n",d);
printf("d=%.10lf\n",d);
}
Salida:
d=1.156126
d=
1.156126
d=
1.1561255726
d=1.1561255726
Ejercicios
1. ¿Cuáles de entre estas declaraciones contienen errores?
a.
b.
c.
d.
e.
f.
g.
h.
i.
integer a;
short i,j,k;
long float (h);
double long d3;
unsigned float n;
char 2j;
int MY;
float ancho, alto, long;
bool i;
2. Dé declaraciones de variables con tipos de datos adecuados para almacenar:
a. la edad de una persona
- 22 -
Lenguaje C
Ejercicios
b. un número de DNI
c. la distancia, en Km, entre dos puntos cualesquiera del globo terrestre
d. el precio de un artículo doméstico
e. el valor de la constante PI expresada con 20 decimales
3. Prepare un programa con variables conteniendo los valores máximos de cada tipo entero, para
comprobar el resultado de incrementarlas en una unidad. Imprima los valores de cada variable antes y
después del incremento. Incluya unsigneds.
4. Lo mismo, pero dando a las variables los valores mínimos posibles, e imprimiéndolas antes y
después de decrementarlas en una unidad.
5. Averigüe los tamaños de todos los tipos básicos en su sistema aplicando el operador sizeof().
6. Si se asigna la expresión (3-5) a un unsigned short, ¿cuál es el resultado?
7. ¿Qué hace falta corregir para que la variable r contenga la división exacta de a y b?
int a, b;
float r;
a = 5;
b = 2;
r = a / b;
8. ¿Qué resultado puede esperarse del siguiente fragmento de código?
int
a =
b =
c =
d =
a, b, c, d;
1;
2;
a / b;
a / c;
9. ¿Cuál es el resultado del siguiente fragmento de código? Anticipe su respuesta en base a lo dicho en
esta unidad y luego confírmela mediante un programa.
printf("%d\n",
printf("%f\n",
printf("%f\n",
printf("%d\n",
printf("%d\n",
printf("%f\n",
printf("%f\n",
printf("%d\n",
20/3);
20/3);
20/3.);
10%3);
3.1416);
(double)20/3);
(int)3.1416);
(int)3.1416);
10. Escribir un programa que multiplique e imprima 100000 por 100000. ¿De qué tamaño son los ints
en su sistema?
11. Convertir una moneda a otra sabiendo el valor de cambio. Dar el valor a dos decimales.
12. Escriba y corra un programa que permita saber si los chars en su sistema son signed o unsigned.
- 23 -
Ejercicios
Eduardo Grosclaude 2001
13. Escriba y corra un programa que asigne el valor 255 a un char, a un unsigned char y a un signed
char, y muestre los valores almacenados. Repita la experiencia con el valor -1 y luego con ’\377’.
Explicar el resultado.
14. Copiar y compilar el siguiente programa. Explicar el resultado.
main()
{
double
int
x;
i;
i = 1400;
x = i; /* conversión int/double */
printf("x = %10.6le\ti = %d\n",x,i);
x = 14.999;
i = x; /* conversión double/int */
printf("x = %10.6le\ti = %d\n",x,i);
x = 1.0e+60;
i = x;
printf("x = %10.6le\ti = %d\n",x,i);
}
15. Escriba un programa que analice la variable v conteniendo el valor 347 y produzca la salida:
3 centenas
4 decenas
7 unidades
(y, por supuesto, salidas acordes si v toma otros valores).
16. Sumando los dígitos de un entero escrito en notación decimal se puede averiguar si es divisible por
3 (se constata si la suma de los dígitos lo es). ¿Esto vale para números escritos en otras bases? ¿Cómo
se puede averiguar esto?
17. Indicar el resultado final de los siguientes cálculos
a.
b.
c.
d.
e.
f.
g.
h.
int a; float b = 12.2; a = b;
int a, b; a = 9; b = 2; a /= b;
long a, b; a = 9; b = 2; a /= b;
float a; int b, c; b = 9; c = 2; a = b/c;
float a; int b, c; b = 9; c = 2; a = (float)(b/c);
float a; int b, c; b = 9; c = 2; a = (float)b/c;
short a, b, c; b = -2; c = 3; a = b * c;
short a, b, c; b = -2; c = 3; a = (unsigned)b * c;
18. Aplicar operador cast donde sea necesario para obtener resutlados apropiados:
int a; long b; float c;
a = 1; b = 2; c = a / b;
long a; int b,c;
b = 1000; c = 1000;
a = b * c;
