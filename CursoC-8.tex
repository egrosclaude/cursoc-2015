



\chapter{Funciones}

Una unidad de traducción en C contiene un conjunto de funciones. Si entre ellas
existe una con el nombre especial \textbf{main}, entonces esa unidad de traducción puede
dar origen a un programa ejecutable, y el comienzo de la función main será el
punto de entrada al programa.

\section{Declaración y definición de funciones} 

Los tipos de datos de los parámetros recibidos y del resultado que devuelve la
función quedan especificados en su cabecera. El valor devuelto se expresa
mediante \textbf{return}:

\begin{ejemplo}
Una función que recibe un int y un long, y devuelve un int.
\begin{lstlisting}
int fun1(int alfa, long beta)
{
	...
}	
\end{lstlisting}

Una función que recibe dos doubles y devuelve un double.
\begin{lstlisting}
double sumar(double x, double y)
{
	...
	return x+y;
}	
\end{lstlisting}
\end{ejemplo}

El caso especial de una función que no desea devolver ningún valor se
especifica con el modificador \textbf{void}, y en tal caso un return, si lo hay, no debe
tener argumento. Los paréntesis son necesarios aunque la función no lleve
parámetros, y en ese caso es recomendable indicarlo con un parámetro void:

\begin{ejemplo}
Una función que recibe un int, y no devuelve ningún valor.
\begin{lstlisting}
void procesar(int k)
{
	...
}	
\end{lstlisting}

Una función que devuelve un int, y no recibe argumentos.
\begin{lstlisting}
int hora(void)
{
	...
}	
\end{lstlisting}
\end{ejemplo}

Una función puede ser declarada de un tipo cualquiera y sin embargo no contar
con una instrucción return. En ese caso su valor de retorno queda
indeterminado. Además, la función que llama a otra puede utilizar o ignorar el
valor devuelto, a voluntad, sin provocar errores.

\begin{ejemplo}
En el caso siguiente se recoge basura en la variable \texttt{a}, ya que \texttt{fun2} no devuelve
ningún valor pese a ser declarada como de tipo entero.
\begin{lstlisting}
int fun2(int x)
{
    ...
    return;
}

    ...
    a=fun2(1);
\end{lstlisting}
\end{ejemplo}


El cuerpo de la función, y en general cualquier cuerpo de instrucciones entre
llaves, es considerado un bloque. Las \textbf{variables locales} son aquellas declaradas
dentro del cuerpo de una función, y su declaración debe aparecer antes de
cualquier sentencia ejecutable. Es legal ubicar la declaración de variables
como la primera sección dentro de cualquier bloque, aun cuando ya se hayan
incluido sentencias ejecutables. Sin embargo, no es legal declarar funciones
dentro de funciones. 

\begin{ejemplo}
La variable \texttt{v} declarada dentro del bloque
interno opaca a la declarada al principio de la función.
\begin{lstlisting}
int fun3()
{
    int j,k,v;

     for(i=0; i<10; i++) {
        double v;
        ...
}
\end{lstlisting}
\end{ejemplo}

	

\section{Prototipos de funciones}
En general, como ocurre con las variables, el uso de una función debe estar
precedido por su declaración. Sin embargo, el compilador trata el caso de las
funciones con un poco más de flexibilidad. Un uso de variable sin declaración
es ilegal, mientras que un uso de función sin definición obliga al compilador a
suponer ciertos hechos, pero permite proseguir la compilación.

La suposición que hará el compilador, en la primera instancia en que se utilice
una función y en ausencia de una definición previa, es que el resultado y los
parámetros de la función son del tipo más \quotes{simple} que pueda representarlos.
Esto vale tanto para las funciones escritas por el usuario como para las mismas
funciones de la biblioteca standard. 

Así, si se intenta calcular $e^5$:
\begin{lstlisting}
main()
{
    double a;
    a=exp(5);
}
\end{lstlisting}

Nada permite al compilador suponer que la función exp() debe devolver algo
distinto de un entero (el hecho de que se esté asignando su valor a un double
no es informativo, dada la conversión automática de expresiones que hace el C).
Además, el argumento 5 puede tomarse a primera vista como int, pudiendo ser que
en la definición real de la función se haya especificado como double, o alguna
otra elección de tipo.

En cualquier caso, esto es problemático, porque la comunicación de parámetros
entre funciones, normalmente, se hace mediante el stack del programa, donde los
objetos se almacenan como sucesiones de bytes. La función llamada intentará
recuperar del stack los bytes necesarios para \quotes{armar} los objetos que necesita,
mientras que la función que llama le ha dejado en el mismo stack menos
información de la esperada. El programa compilará correctamente pero los datos
pasados a y desde la función serán truncamientos de los valores deseados.

\section{Redeclaración de funciones}
Otro problema, relacionado con el anterior, es el que ocurre si permitimos que
el compilador construya esa declaración provisoria y luego, en la misma unidad
de traducción, damos la definición de la función, y ésta no concuerda con la
imaginada por el compilador. La compilación abortará con error de
\quotes{redeclaración de función}.
La forma de advertir al compilador de los tipos correctos antes del uso de la
función es, o bien, definirla (proporcionando su fuente), o incluir su
prototipo:
\begin{lstlisting}
double exp(double x); /* prototipo de exp() */
main()
{
    double a;
    a=exp(5);
}
\end{lstlisting}

En el caso particular de las funciones de biblioteca standard, cada grupo de
funciones cuenta con su header conteniendo estas declaraciones, que podemos
utilizar para ahorrarnos tipeo. Para las matemáticas, utilizamos el header
math.h:
\begin{lstlisting}
#include <math.h>
main()
{
    double a;
    a=exp(5);
}
\end{lstlisting}

Un problema más serio que el de la redeclaración de funciones es cuando una
función es compilada en una unidad de traducción separada A y luego se la
utiliza, desde una función en otra unidad de traducción B, pero con una
declaración incorrecta, ya sea porque se ha suministrado un prototipo erróneo o
porque no se ha suministrado ningún prototipo explícito. y el implícito, que
puede inferir el compilador, no es el correcto. En este caso la compilación y
la linkedición tendrán lugar sin errores, pero la conducta al momento de
ejecución depende de la diferencia entre ambos prototipos, el real y el
inferido.

\section{Recursividad}
Las funciones en C pueden ser recursivas, es decir, pueden invocarse a sí
mismas directa o indirectamente. Siguiendo el principio de que las estructuras
de programación deben replicar la estructura de los datos, la recursividad de
funciones es una manera ideal de tratar las estructuras de datos recursivas,
como árboles, listas, etc.
\begin{lstlisting}
int factorial(int x)
{
    if(x==0)
        return 1;
    return x * factorial(x-1);
}
\end{lstlisting}


\section{Ejercicios}
\begin{enumerate}
	\item Escribir una función que reciba tres argumentos enteros y devuelva un
entero, su suma.
\item Escribir una función que reciba dos argumentos enteros y devuelva un long,
su producto.
\item Escribir una función que reciba dos argumentos enteros a y b, y utilice a
las dos anteriores para calcular:
\begin{lstlisting}
(a * b + b * 5 + 2) * (a + b + 1)
\end{lstlisting}
\item Escribir un programa que utilice la función anterior para realizar el
cálculo con a=7 y b=3.
\item ¿Qué está mal en estos ejemplos?
\begin{enumerate}[label=\alph*.]
\item 
\begin{lstlisting}
int f1(int x, int y);
{
    int z;
    z = x - y;
    return z;
}
\end{lstlisting}

\item 
\begin{lstlisting}
void f2(int k)
{
    return k + 3;
}
\end{lstlisting}

\item 
\begin{lstlisting}
int f3(long k)
{
    return (k < 0) ? -1 : 1;
}
printf("%d\n",f3(8));
\end{lstlisting}
\end{enumerate}
\item Escribir una función que reciba dos argumentos, uno de tipo int y el otro de
tipo char. La función debe repetir la impresión del char tantas veces como lo
diga el otro argumento. Escribir un programa para probar la función.

\end{enumerate}

%TODO Ejercicios_Adicionales Ejercicios_Avanzados
